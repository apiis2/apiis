\documentclass[10pt,a4paper,DIV14]{scrartcl}
\usepackage{german}
\usepackage{umlaut}
\usepackage{longtable}
\usepackage{fancyvrb}

\pagestyle{myheadings}
\markright{\today}
\begin{document}

\centerline{\Large \bf Roundtable 2. Day}
\vspace{5mm}

\section{Present the own Material}

Main informations could be:
\begin{itemize}
\item global informations (species, traits)
\item number of data sources and short description
\item number of raw files and structure (like pipe seperated or not,
  multiple header\ldots)
\item number of dataelements in the raw files
\item if there different namespaces (numbering systems) in the data
\item have one animal more than one identification
\end{itemize}

\section{Model}

\begin{itemize}
\item describe the added tables and their relations
\item present the normalised structure as xfig
\item show business rules % run show_rules, show_tab.pl
\end{itemize}


\centerline{\Large \bf Roundtable 3. Day}
\vspace{5mm}

\section{Codes / Unit}

\begin{itemize}
\item which codes / units are present on the material
\item change the modelfile (use db\_ and set ForeignKey)
\end{itemize}


\section{Animal Identification}

\begin{itemize}
\item number of different namespaces
\item which elements have to be concatenated to be unique
\item display identified unknown or invalid IDs
\item display duplicated animal IDs
\end{itemize}


\end{document}





