%\section{Interfaces to the Model File}

Usually you have to provide the model file\index{Model file}, you
want to use with your \apiis{} program. This is done with
\verb+$apiis->join_model('my_model_file');+.
\verb+join_model()+\index{\verb+$apiis+!\verb+->join_model+} reads in
the model file, parses it and creates public interfaces to all the
information via an object, which is added to the \verb+$apiis+ structure
via \verb+Apiis::Init::_add_object+\index{\verb+$apiis+!\verb+->_add_object+}.
The entry point for these methods is
\verb+$apiis->Model+\index{\verb+$apiis+!\verb+->Model+}.

An auxiliary module to find the model file in the \verb+$APIIS_HOME+
hierarchy is \verb+Apiis::CheckFile+\index{Apiis::CheckFile},
which scans some likely directories for candidates. It also can be used for searching
other configuration files like those for forms or reports.

The model file contains both some basic informations and -- primarily --
the complete database structure\index{database structure}, including business rules.

The basic information can be accessed by these public
methods:\index{\verb+$apiis+!\verb+->Model+!public methods}

\smallskip
\begin{tabular}{rl|l}
\multicolumn{3}{l}{\texttt{\$apiis->Model->}}       \\
                      &\verb+basename+         & basename of the model file \\
                      &\verb+check_level+      & get/set check level \\
                      &\verb+db_driver+        & database driver \\
                      &\verb+db_host+          & database host machine \\
                      &\verb+db_name+          & database name \\
                      &\verb+db_password+      & database password of current user \\
                      &\verb+db_port+          & database system port \\
                      &\verb+db_user+          & database system (meta) user \\
                      &\verb+ext+              & extension of model file name \\
                      &\verb+fullname+         & full name of model file \\
                      &\verb+max_check_level+  & maximum check level reached \\
                      &\verb+path+             & path to model file \\
                      &\verb+table+            & object for detailed table information \\
                      &\verb+tables+           & array(ref) of all tablenames \\
\end{tabular}
\medskip

For all tables in the model file, a table object is created which contains the
information for this table. To get all columns of table 'animal' you could write:

\begin{verbatim}
   my $table_obj = $apiis->Model->table('animal');
   printf "Columns of table animal are: %s\n",
      join(', ', $table_obj->columns);
\end{verbatim}

A more intuitive and shorter way is:

\begin{verbatim}
   printf "Columns of table animal are: %s\n",
      join(', ', $apiis->Model->table('animal')->columns);
\end{verbatim}

The methods of the table object are:\index{\verb+$apiis+!\verb+->Model+!\verb+->table()+!public methods}

\smallskip
\begin{tabular}{rl|l}
\multicolumn{3}{l}{\texttt{\$apiis->Model->table(<name>)->}}       \\
                                     &\verb+check+       & check rules of a column \\
                                     &\verb+cols+        & columns of this table \\
                                     &\verb+columns+     & columns of this table \\
                                     &\verb+datatype+    & meta-datatype of a column \\
                                     &\verb+default+     & default value of a column \\
                                     &\verb+description+ & description of a column \\
                                     &\verb+foreignkey+  & foreign key definition of a column \\
                                     &\verb+index+       & indices for this table \\
                                     &\verb+indexes+     & indices for this table \\
                                     &\verb+indices+     & indices for this table \\
                                     &\verb+length+      & default length of a column (forms) \\
                                     &\verb+modify+      & modify rules of a column \\
                                     &\verb+name+        & name of this table \\
                                     &\verb+primarykey+  & primary key definitions of this table \\
                                     &\verb+rowid+       & get/set rowid of this record \\
                                     &\verb+sequence+    & sequences defined for this table \\
                                     &\verb+sequences+   & sequences defined for this table \\
                                     &\verb+triggers+    & trigger definitions for this table \\
\end{tabular}
\medskip

Most of the mentioned methods will be used rarely as they are implemented in the
newer record object in a more consistent way. Some of them are even outdated, like
the read/write method
\begin{verbatim}
$apiis->Model->table( $mytable )->rowid( $thisrowid );
\end{verbatim}

A future rewrite of the model file interface will have a syntax similar to the record
objects. Instead of 
\begin{verbatim}
$apiis->Model->table($thistable)->foreignkey->($thiscolumn);
\end{verbatim}
it  likely will look like
\begin{verbatim}
$apiis->Model->table($thistable)->column($thiscolumn)->foreignkey;
\end{verbatim}
This will give a a consistent syntax according to this record object example:
\begin{verbatim}
$record_obj->column($thiscolumn)->foreignkey;
\end{verbatim}

% vim: expandtab:tw=100
