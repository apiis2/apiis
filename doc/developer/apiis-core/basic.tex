% \section{Basic Initialization}

Each program in the \apiis{} suite has to start with the 
preinitialization:

\begin{verbatim}
#!/usr/bin/env perl
##############################################################################
# $Id: basic.tex,v 1.3 2005/03/08 11:02:34 heli Exp $
# Short description about this program
##############################################################################

BEGIN {
   use Env qw( APIIS_HOME );
   die "\n\tAPIIS_HOME is not set!\n\n" unless $APIIS_HOME;
   push @INC, "$APIIS_HOME/lib";
}

use strict;
use warnings;
use Apiis;
Apiis->initialize( VERSION => '$Revision: 1.3 $' );
\end{verbatim}

The \verb+BEGIN+ block checks at compile time, if the environment variable
\verb+$APIIS_HOME+ is set.
\verb+$APIIS_HOME+ is the main starting point for the complete code base of
\apiis. \verb+$APIIS_HOME/lib+ is included into the library search path of
Perl.

\verb+use Apiis.pm;+ loads the \apiis-system into your main namespace and
provides the method initialize().
Usually you should pass the version of your program to \verb+initialize+.
When you commit this program to CVS, the cvs-tag \verb+$Revision: 1.3 $+ will
be replaced by the current version and it will look like
\verb+$Revision: 1.3 $+. You can recall the version later with
\verb+$apiis->version+\index{\verb+$apiis+!\verb+->version+}.
If you don't use cvs then just hardcode the version:
\verb+Apiis->initialize( VERSION => '0.71' );+

Every program which exceeds the intention of being a quick and dirty script for
a small daily job should contain the following two lines before you start coding:

\begin{verbatim}
   use strict;
   use warnings;
\end{verbatim}

You should not ask somebody else to help you track down some bugs in your code unless
you used \verb+strict+ and \verb+warnings+!

\subsection{initialize}\index{initialize}

Besides performing some basic checks the main task of
\verb+initialize+ is the creation of the global
\verb+$apiis+-structure, represented by the object reference \verb+$apiis+.

\subsection{Apiis::Init}
The successfull creation of the \verb+$apiis+ object implies some basic
initializing before and adds various items:
\begin{itemize}
   \item the base configuration from \verb+$APIIS_HOME/apiisrc+
   \item the I18N/L10N part
   \item methods to extend the \verb+$apiis+ structure modularly
   \item methods for logging to files or the system syslog
   \item the public interfaces to set an error status, check this status and
   document errors in detail
   \item handy methods to access the current day and time 
\end{itemize}

The main setup for error handling is done in the package \verb+Apiis::Errors+.

On the main level, the \verb+$apiis+ object provides these public
methods\index{\verb+$apiis+!public methods}:

\smallskip
\begin{tabular}{rl|l}
\verb+$apiis->+& \verb+APIIS_HOME+       & path to \verb+$APIIS_HOME+  \\
               & \verb+APIIS_LOCAL+      & path to \verb+$APIIS_LOCAL+ \\
               & \verb+browser+          & your favourite browser \\
               & \verb+check_status+     & checks the error status \\
               & \verb+code_table+       & usually table 'codes' \\
               & \verb+date_format+      & the date format (US or EU) \\
               & \verb+entry_views+      & hashref of entry-views \\
               & \verb+errors+           & array(reference) of error objects \\
               & \verb+exists_database+  & is the database joined to \verb+$apiis+ \\
               & \verb+exists_form+      & is a form joined to \verb+$apiis+\\
               & \verb+exists_model+     & is the model file joined to \verb+$apiis+ \\
               & \verb+fileselector+     & choose a Tk fileselector \\
               & \verb+HOME+             & path to \verb+$HOME+ \\
               & \verb+join_database+    & join the DataBase object into \verb+$apiis+ \\
               & \verb+join_form+        & join a named Form object into \verb+$apiis+ \\
               & \verb+join_model+       & join the Model object into \verb+$apiis+ \\
               & \verb+language+         & choose user language \\
               & \verb+localtime+        & unformatted timestamp \\
               & \verb+now+              & formatted timestamp \\
               & \verb+programname+      & name of the invoking program \\
               & \verb+reserved_strings+ & reserved strings for data \\
               & \verb+status+           & error status \\
               & \verb+today+            & formatted date \\
               & \verb+user+             & current user \\
               & \verb+version+          & version of the invoking program \\
\end{tabular}

\smallskip
See the POD-documentation for a detailed reference of these methods.

% vim: expandtab:tw=100
