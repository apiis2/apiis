\subsection{usage\label{usage}\index{usage}}


get\_value( 'position cfix position2', $\backslash$@lineref, format )

\subsection{description\label{description}\index{description}}


return value from datafile at specific position   @lineref is the line from datafile
return: value if position known as code (from collect\_codes1.pl)
return db\_code after check file codes.chg for possible changing this code
same for unit and table\_id
MUST be used if fields can be empty: as a consequence the variable is not an
empty string but a NULL value!!!

\subsection{configuration\label{configuration}\index{configuration}}
\subsubsection*{position\label{position}\index{position}}


describe the position on input line. could be more than once and also
fixed parts are possible (see the c-option).

\subsubsection*{format\label{format}\index{format}}


define either an date format, a simple test if the return value is
an number or lower resp. upper case any character.

\paragraph*{[dmyj] plus any character which split the date (.-/...)\label{_dmyj_plus_any_character_which_split_the_date_-_}\index{[dmyj] plus any character which split the date (.-/...)}}


date format use the function getdate(). this allow an specific format
like 'dd.mm.yyyy' or 'ddmmjj' or use the number of elements in the
data. for further details see function getdate().

\paragraph*{n\label{n}\index{n}}


test is a number else ignore the value. also change from ',' to '.' if exist.

\paragraph*{cl\label{cl}\index{cl}}


lower case characters.

\paragraph*{cu\label{cu}\index{cu}}


upper case characters.

\subsection{used in\label{used_in}\index{used in}}


loading historic data (load\_data.pl)

\subsection{usage\label{usage}\index{usage}}


getdate(date) or getdate(date, format)

\subsection{return\label{return}\index{return}}
\begin{verbatim}
 formated date, status, err_msg
\end{verbatim}
\subsection{description\label{description}\index{description}}


getdate should be simplify the handling of dates in incomming
datastreams.



only the following dates are possible:

\begin{enumerate}

\item 799 -$>$3-Juli-1999
\item 799 -$>$ 12-Juli-1999
\item -$>$ 12-Dezember-1999
\item [12.07.99$|$12-07-99$|$12:07:99] -$>$ 12-Juli-1999\end{enumerate}


or using a format as second parameter like 'dd.mm.jjjj or 'yyy.tt.mm'
or 'ttmmjj' and so on



getdate('19984/02','yyyymm/tt') =$>$ 2-April-1998

\subsection{xfig\_lib.pm\label{xfig_lib_pm}\index{xfig\ lib.pm}}


Library for model2xfig

\subsubsection*{DESCRIPTION\label{xfig_lib_pm_DESCRIPTION}\index{xfig lib pm!DESCRIPTION}}


xfig\_lib initialized basic variables, sets default values and defines subroutines
for creating the FIG-header, for compounding objects, drawing lines, arrowlines,
boxes and text.

\subsubsection*{SUBROUTINES\label{xfig_lib_pm_SUBROUTINES}\index{xfig lib pm!SUBROUTINES}}
\paragraph*{getFileHeader\label{xfig_lib_pm_getFileHeader}\index{xfig lib pm!getFileHeader}}


getFileHeader creates the FIG-file header.



usage: \$header = getFileHeader( [comment] )

\begin{verbatim}
   comment: comment to print into the header
\end{verbatim}
\begin{verbatim}
   returnvalue: string with header
\end{verbatim}
\paragraph*{texttype\label{xfig_lib_pm_texttype}\index{xfig lib pm!texttype}}


creates a \emph{text} object.



usage: \$textobj = texttype( \$text, \$type,
                            \$xpos, \$ypos, \$depthT )

\begin{verbatim}
    text: string
    type: t = title;  c = columnname ; h = head line
    xpos: x-position
    ypos: y-position
  depthT: depth (layer)
\end{verbatim}
\begin{verbatim}
 returnvalue: string with text object
\end{verbatim}
\paragraph*{boxtype\label{xfig_lib_pm_boxtype}\index{xfig lib pm!boxtype}}


creates a \emph{box} object.



usage: \$boxobj = boxtype( \$x1, \$y1 ,\$x2 ,\$y2 )

\begin{verbatim}
   x1,y1: first corner point
   x2,y2: final corner point
\end{verbatim}
\begin{verbatim}
 returnvalue: string with box object
\end{verbatim}
\paragraph*{linetype\label{xfig_lib_pm_linetype}\index{xfig lib pm!linetype}}


creates a \emph{line} object.



usage: \$lineobj = linetype( \$x1, \$y1 ,\$x2 ,\$y2 )

\begin{verbatim}
   x1,y1: first point
   x2,y2: final point
\end{verbatim}
\begin{verbatim}
 returnvalue: string with line object
\end{verbatim}
\paragraph*{arrowlinetype\label{xfig_lib_pm_arrowlinetype}\index{xfig lib pm!arrowlinetype}}


creates a \emph{arrowline} object with a \textbf{backward} arrow.



usage: \$arrowlineobj = arrowlinetype( \$x1, \$y1 ,\$x2 ,\$y2 )

\begin{verbatim}
   x1,y1: first point
   x2,y2: final point
\end{verbatim}
\begin{verbatim}
 returnvalue: string with arrowline object
\end{verbatim}
\paragraph*{compoundtype\label{xfig_lib_pm_compoundtype}\index{xfig lib pm!compoundtype}}


creates a \emph{compound} object.



usage: \$compoundobj = compoundtype( \$objects,
                                   \$upperight\_corner\_x,
                                   \$upperight\_corner\_y,
                                   \$lowerleft\_corner\_x,
                                   \$lowerleft\_corner\_y )

\begin{verbatim}
         objects:  string with xfig objects. Created with
                   texttype,boxtype,linetype and
                   arrowlinetype
\end{verbatim}
\begin{verbatim}
         upperight_corner_x, upperight_corner_y,
         lowerleft_corner_x, lowerleft_corner_y
         defines a rectangle so that all objects are
         inside of this region.
\end{verbatim}
\begin{verbatim}
 returnvalue: string with compound object
\end{verbatim}
\subsubsection*{SEE ALSO\label{xfig_lib_pm_SEE_ALSO}\index{xfig lib pm!SEE ALSO}}


model2xfig, xfig (www.xfig.org)

\subsubsection*{AUTHOR\label{xfig_lib_pm_AUTHOR}\index{xfig lib pm!AUTHOR}}


Hartmut B�rner (haboe@tzv.fal.de)

