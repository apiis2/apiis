\subsubsection{DateDiff\label{DateDiff}\index{DateDiff}}




\paragraph*{SYNOPSIS\label{DateDiff_SYNOPSIS}\index{DateDiff!SYNOPSIS}}


Syntax in model file:

\begin{verbatim}
   DateDiff min_diff max_diff compare_date [reference_column]
\end{verbatim}


Examples:
   CHECK =$>$ ['DateDiff 0 365 2001-03-22'],
   CHECK =$>$ ['DateDiff 1 100 buy\_dt'],
   CHECK =$>$ ['DateDiff 1 50  animal=$>$birth\_dt db\_animal'],

\paragraph*{DESCRIPTION\label{DateDiff_DESCRIPTION}\index{DateDiff!DESCRIPTION}}


DateDiff takes the current value (\$data) of the passed column and computes the difference to the
date, given in the third parameter (compare\_date). If the difference (in
days) between \$data and compare\_date is in the range given by min\_diff and
max\_diff, DateDiff will return 0 for success, otherwise 1. In other words:

\begin{verbatim}
   min_diff <= ($data - compare_date) <= max_diff    # success
\end{verbatim}


compare\_date can either be a fixed format date like '2001-03-22' (must be
in ISO 8601 format) or a date in a column of this record or a date in the
column of another table.  In the latter case, the format is
'tablename=$>$columname' and you additionally have to give the referencing
column of both tables. This referencing column connects both tables
(usually a foreign key).



Examples:

\begin{verbatim}
   # fixed format:
   DateDiff 1 365 2000-01-15
\end{verbatim}
\begin{verbatim}
   # compare to another column in the current record:
   DateDiff 30 50 buy_dt
\end{verbatim}
\begin{verbatim}
   # compare to another table.column:
   DateDiff 80 120 animal=>birthdate db_animal
\end{verbatim}
\paragraph*{AUTHORS\label{DateDiff_AUTHORS}\index{DateDiff!AUTHORS}}


Helmut Lichtenberg $<$heli@tzv.fal.de$>$

\subsubsection{ForeignKey\label{ForeignKey}\index{ForeignKey}}




\paragraph*{SYNOPSIS\label{ForeignKey_SYNOPSIS}\index{ForeignKey!SYNOPSIS}}


Syntax in model file:

\begin{verbatim}
   ForeignKey fk_table fk_column [column=value]
\end{verbatim}
\paragraph*{DESCRIPTION\label{ForeignKey_DESCRIPTION}\index{ForeignKey!DESCRIPTION}}


The internal data of the current table and column must have an according
entry in fk\_table.fk\_column.



Undefined data (NULL) does not violate the rule. It must be checked with
NotNull.



ForeignKey() returns 0 in case of success, otherwise it creates a
descriptive record error object and returns 1;



For internal en-/decoding, the ForeignKey rule is somehow violated with
additional parameters, which are not needed for the pure
FK-checking. If we have a FK-definition:

\begin{verbatim}
   ForeignKey codes db_code class=BREED
\end{verbatim}


the FK-checking only looks in table codes, column db\_code. The additional
'class=BREED' entry is used for the coding stuff.

\subparagraph*{check\_ForeignKey()\label{ForeignKey_check_ForeignKey_}\index{ForeignKey!check\ ForeignKey()}}


\textbf{check\_ForeignKey()} checks the correctness of the input parameters.



In case of errors it sets \$self-$>$status and additionally returns a non-true
returnvalue.

\paragraph*{AUTHORS\label{ForeignKey_AUTHORS}\index{ForeignKey!AUTHORS}}


Helmut Lichtenberg $<$heli@tzv.fal.de$>$

\subsubsection{IsAFloat\label{IsAFloat}\index{IsAFloat}}




\paragraph*{SYNOPSIS\label{IsAFloat_SYNOPSIS}\index{IsAFloat!SYNOPSIS}}


The passed data must be a floating point number

\paragraph*{DESCRIPTION\label{IsAFloat_DESCRIPTION}\index{IsAFloat!DESCRIPTION}}
\subparagraph*{IsAFloat()\label{IsAFloat_IsAFloat_}\index{IsAFloat!IsAFloat()}}


The value of the current column has to be a floating point number.  Empty
values are allowed.



Returnvalues:
   nothing in case of success
   local status with true value, errors are stored in \$record-$>$errors

\subparagraph*{check\_IsAFloat()\label{IsAFloat_check_IsAFloat_}\index{IsAFloat!check\ IsAFloat()}}


\textbf{check\_IsAFloat()} checks the correctness of the input parameters.



In case of errors it sets \$self-$>$status and additionally returns a non-true
returnvalue.



Checks are:
   Existence of additional parameters

\paragraph*{AUTHORS\label{IsAFloat_AUTHORS}\index{IsAFloat!AUTHORS}}


Helmut Lichtenberg $<$heli@tzv.fal.de$>$

\subparagraph*{IsAFloat\label{IsAFloat_IsAFloat}\index{IsAFloat!IsAFloat}}


The passed value \$data is checked to be a float. The , is not substituted to
decimal point as this subroutine only return 0 or 1, not the (substituted)
value. This is a job for the Modify rule CommaToDot.



Returnvalues: 0 if \$data is a legal float, 1 if \$data contains illegal chars

\subsubsection{IsANumber\label{IsANumber}\index{IsANumber}}




\paragraph*{SYNOPSIS\label{IsANumber_SYNOPSIS}\index{IsANumber!SYNOPSIS}}


Checks, if the provided data is a number.

\paragraph*{DESCRIPTION\label{IsANumber_DESCRIPTION}\index{IsANumber!DESCRIPTION}}
\subparagraph*{IsANumber()\label{IsANumber_IsANumber_}\index{IsANumber!IsANumber()}}


The value of the current column has to be a number.  Empty values are
allowed. The test is done by comparing firstly \$value with \$value+0 and, if
this fails, with a more complex regex.



Returnvalues:
   nothing in case of success
   local status with true value, errors are stored in \$record-$>$errors

\subparagraph*{check\_IsANumber()\label{IsANumber_check_IsANumber_}\index{IsANumber!check\ IsANumber()}}


\textbf{check\_IsANumber()} checks the correctness of the input parameters.



In case of errors it sets \$self-$>$status and additionally returns a non-true
returnvalue.



Checks are:
   Existence of additional parameters

\subparagraph*{\_is\_a\_number() (internal)\label{IsANumber__is_a_number_internal_}\index{IsANumber!\ is\ a\ number() (internal)}}


\textbf{\_is\_a\_number()} is an internal routine which is not bind to the record
object. Some rules want to check the passed parameter (e.g. Range), if they
are numbers. \textbf{\_is\_a\_number()} gets as input the value, which it has to
check. \textbf{IsANumber} uses \textbf{\_is\_a\_number()}, too.



In case of errors it returns a non-true returnvalue.

\paragraph*{AUTHORS\label{IsANumber_AUTHORS}\index{IsANumber!AUTHORS}}


Helmut Lichtenberg $<$heli@tzv.fal.de$>$

\subsubsection{IsEqual\label{IsEqual}\index{IsEqual}}




\paragraph*{SYNOPSIS\label{IsEqual_SYNOPSIS}\index{IsEqual!SYNOPSIS}}


Syntax: IsEqual \$table \$id\_column \$column compare\_constant [nullok]

\paragraph*{DESCRIPTION\label{IsEqual_DESCRIPTION}\index{IsEqual!DESCRIPTION}}


\textbf{IsEqual()} is usually used as a CHECK-rule in the model file.



It checks, if a record, identified by \$table.\$id\_column has the value
'compare\_constant' in column \$column.



Example: IsEqual animal db\_animal db\_sex male



This CHECK-rule can be attached to a column like db\_sire in
service and tests if the animal ID, given in the passed column, points indeed
to a male animal. The record from table animal, where db\_animal is equal to
the data of the current column, must have an entry 'male' in column db\_sex.
The the 'compare\_constant' part is a fixed value and specified as external code
(codes.ext\_code).



Returnvalues:

\begin{enumerate}

\item 

0 if the retrieved record from \$table.\$id\_column has an entry of
'compare\_constant' in column \$column.



If the optional parameter 'nullok' is given, an undefined value for this
column (or no retrieved record) will also be accepted.


\item 

All other cases indicate error and an error message exists.

\end{enumerate}
\subparagraph*{check\_IsEqual()\label{IsEqual_check_IsEqual_}\index{IsEqual!check\ IsEqual()}}


\textbf{check\_IsEqual()} checks the correctness of the input parameters.



In case of errors it puts an error into \$self-$>$errors and additionally
returns a non-true returnvalue.



Checks are:

\begin{verbatim}
   Missing parameters
   Last parameter must be 'nullok' if it exists
\end{verbatim}
\paragraph*{AUTHORS\label{IsEqual_AUTHORS}\index{IsEqual!AUTHORS}}


Helmut Lichtenberg $<$heli@tzv.fal.de$>$

\paragraph*{VERSION\label{IsEqual_VERSION}\index{IsEqual!VERSION}}
\begin{verbatim}
   $Revision: 1.15 $
\end{verbatim}
\paragraph*{LastAction\label{LastAction}\index{LastAction}}


Syntax:



LastAction table=$>$chain\_col LA\_chain min\_diff max\_diff



LastAction is a conditional DateDiff depending on the value of LA\_chain,
which is stored in table=$>$chain\_col. The dates to compare are
table=$>$chain\_col\_dt (\_dt extension is hardcoded!) and the passed current
value \$data.



If the date difference between chain\_col\_dt and \$data for this LastAction
in table=$>$chain\_col is \textbf{not} within the defined range, the (error)status
\$self-$>$status is set to 1 and an appropriate error object is created. If
this rule is not violated, the (error)status is 0.



Example for an entry in the model file:

\begin{verbatim}
   CHECK => ['LastAction animal=>la_rep
                  SERVICE 18 62
                  FARROW  40 80'],
\end{verbatim}


\textbf{Note!} As LastAction is a very specific check rule there are some details
hardcoded. The connecting column between tables (foreign key) is db\_animal
in both tables. The column, that contains the date for the last action is
assumed to have the last-action-column name with '\_dt' appended.



If LastAction turns out to be a useful check rule for other purposes where
the hardcoding is an obstacle, it can be rewritten in a generic manner,
likely with the drawback of some changes in parameter passing.

\paragraph*{skip\_LastAction()\label{skip_LastAction_}\index{skip\ LastAction()}}


\textbf{skip\_LastAction()} returns the actions, when checking of this rule should be
skipped. For LastAction, checks during an update operation are useless.



Input: none
Output: arrayref

\paragraph*{check\_LastAction()\label{check_LastAction_}\index{check\ LastAction()}}


\textbf{check\_LastAction()} checks the correctness of the input parameters.



In case of errors it sets \$self-$>$status and additionally returns a non-true
returnvalue.

\subsubsection{AUTHORS\label{AUTHORS}\index{AUTHORS}}


Helmut Lichtenberg $<$heli@tzv.fal.de$>$

\subsubsection{List\label{List}\index{List}}




\paragraph*{SYNOPSIS\label{List_SYNOPSIS}\index{List!SYNOPSIS}}


\textbf{List()} is the poor man's foreign key. The data is checked against a
small list which is provided in the model file.

\paragraph*{DESCRIPTION\label{List_DESCRIPTION}\index{List!DESCRIPTION}}


The model file can provide a small list where the data is checked against.
Example, column db\_sex:

\begin{verbatim}
   CHECK => ['List Male Female'],
\end{verbatim}


The external data is allowed to have the values 'Male' or 'Female'. The
number of List entries is not limited:

\begin{verbatim}
   CHECK => ['List val1 val2 ... valN'],
\end{verbatim}


To circumvent upper/lower case problems you can combine MODIFY and CHECK
rules:

\begin{verbatim}
   MODIFY => ['UpperCase'],
   CHECK  => ['List MALE FEMALE'],
\end{verbatim}


The data is first modified and then checked.



Undefined or NULL data is accepted as it can get controlled with NotNull.



Returnvalues:
   0 if \$data is one of the list values (success),
   1 otherwise (error)

\subparagraph*{check\_List()\label{List_check_List_}\index{List!check\ List()}}


\textbf{check\_List()} checks the correctness of the input parameters.



In case of errors it sets \$self-$>$status and additionally returns a non-true
returnvalue.

\paragraph*{AUTHORS\label{List_AUTHORS}\index{List!AUTHORS}}


Helmut Lichtenberg $<$heli@tzv.fal.de$>$

\subsubsection{NoCheck\label{NoCheck}\index{NoCheck}}




\paragraph*{SYNOPSIS\label{NoCheck_SYNOPSIS}\index{NoCheck!SYNOPSIS}}


The Rule \textbf{NoCheck} is checks nothing. It is intended to be a noop rule to
overwrite existing ones on a lower CHECK-level.

\paragraph*{DESCRIPTION\label{NoCheck_DESCRIPTION}\index{NoCheck!DESCRIPTION}}
\subparagraph*{NoCheck()\label{NoCheck_NoCheck_}\index{NoCheck!NoCheck()}}


Syntax: NoCheck



Returnvalues:
   0 if data is within this range, 1 otherwise
   errors are stored in \$record-$>$errors



A non-true return value can only happen if the rule \textbf{NoCheck} is defined
incorrectly in the model file, i.e. an additional parameter is provided.

\subparagraph*{check\_NoCheck()\label{NoCheck_check_NoCheck_}\index{NoCheck!check\ NoCheck()}}


\textbf{check\_NoCheck()} checks the correctness of the input parameters.



In case of errors it returns a non-true returnvalue.



Checks are:
   existence of additional parameters

\paragraph*{AUTHORS\label{NoCheck_AUTHORS}\index{NoCheck!AUTHORS}}


Helmut Lichtenberg $<$heli@tzv.fal.de$>$

\subsubsection{NoNumber\label{NoNumber}\index{NoNumber}}




\paragraph*{SYNOPSIS\label{NoNumber_SYNOPSIS}\index{NoNumber!SYNOPSIS}}
\paragraph*{DESCRIPTION\label{NoNumber_DESCRIPTION}\index{NoNumber!DESCRIPTION}}


Checks, if the provided data is not a number.

\subparagraph*{NoNumber()\label{NoNumber_NoNumber_}\index{NoNumber!NoNumber()}}


The value of the current column must not be a number.  Empty values are
allowed.



Returnvalues:
   * nothing in case of success
   * local status with true value in case of failure, errors are stored in
     \$record-$>$errors

\subparagraph*{check\_NoNumber()\label{NoNumber_check_NoNumber_}\index{NoNumber!check\ NoNumber()}}


\textbf{check\_NoNumber()} checks the correctness of the input parameters.



In case of errors it sets \$self-$>$status and additionally returns a non-true
returnvalue.



Checks are:
   Existence of additional parameters

\paragraph*{AUTHORS\label{NoNumber_AUTHORS}\index{NoNumber!AUTHORS}}


Helmut Lichtenberg $<$heli@tzv.fal.de$>$

\subsubsection{NotNull\label{NotNull}\index{NotNull}}




\paragraph*{SYNOPSIS\label{NotNull_SYNOPSIS}\index{NotNull!SYNOPSIS}}


\textbf{NotNull()} checks, if the data has a defined value

\paragraph*{DESCRIPTION\label{NotNull_DESCRIPTION}\index{NotNull!DESCRIPTION}}


The passed value \$data is not allowed to be undefined or empty. It may have
the numeric value 0.



\textbf{NotNull()} is usually used as a CHECK-rule in the model file.

\paragraph*{AUTHORS\label{NotNull_AUTHORS}\index{NotNull!AUTHORS}}


Helmut Lichtenberg $<$heli@tzv.fal.de$>$

\subsubsection{Range\label{Range}\index{Range}}




\paragraph*{SYNOPSIS\label{Range_SYNOPSIS}\index{Range!SYNOPSIS}}
\paragraph*{DESCRIPTION\label{Range_DESCRIPTION}\index{Range!DESCRIPTION}}


The Rule \textbf{Range} is given a range of values in the model file. It then
checks, if the provided data is within this range.

\subparagraph*{Range()\label{Range_Range_}\index{Range!Range()}}


Syntax: Range min\_value max\_value



Is the data within a range? min\_value and max\_value are predefined in the
model file.



Returnvalues:
   0 if data is within this range, 1 otherwise
   errors are stored in \$record-$>$errors

\subparagraph*{check\_Range()\label{Range_check_Range_}\index{Range!check\ Range()}}


\textbf{check\_Range()} checks the correctness of the input parameters.



In case of errors it sets \$self-$>$status and additionally returns a non-true
returnvalue.



Checks are:
   if min\_value and max\_value are defined
   if min\_value and max\_value are numbers

\paragraph*{BUGS\label{Range_BUGS}\index{Range!BUGS}}


\textbf{Range} is intended to work only for numerical values.

\paragraph*{AUTHORS\label{Range_AUTHORS}\index{Range!AUTHORS}}


Helmut Lichtenberg $<$heli@tzv.fal.de$>$

\subsubsection{ReservedStrings\label{ReservedStrings}\index{ReservedStrings}}




\paragraph*{SYNOPSIS\label{ReservedStrings_SYNOPSIS}\index{ReservedStrings!SYNOPSIS}}


\textbf{ReservedStrings} checks, if the data contains strings, that are not
allowed as they are used e.g. as concatenation symbol.

\paragraph*{DESCRIPTION\label{ReservedStrings_DESCRIPTION}\index{ReservedStrings!DESCRIPTION}}


\textbf{ReservedStrings} checks if the passed data contains one of the reserved
strings which are defined in apiisrc. NULL data (undefined or empty) will
pass successfully.



\textbf{ReservedStrings} is usually used as a CHECK-rule in the model file.

\subparagraph*{check\_ReservedStrings\label{ReservedStrings_check_ReservedStrings}\index{ReservedStrings!check\ ReservedStrings}}


\textbf{check\_ReservedStrings} checks the correctness of the input parameters.
In case of errors it puts an error into \$record-$>$errors and returns a
non-true returnvalue.



Checks are:
   Existence of parameters

\paragraph*{AUTHORS\label{ReservedStrings_AUTHORS}\index{ReservedStrings!AUTHORS}}


Helmut Lichtenberg $<$heli@tzv.fal.de$>$

\subsubsection{Unique\label{Unique}\index{Unique}}




\paragraph*{SYNOPSIS\label{Unique_SYNOPSIS}\index{Unique!SYNOPSIS}}
\paragraph*{DESCRIPTION\label{Unique_DESCRIPTION}\index{Unique!DESCRIPTION}}
\subparagraph*{Unique()\label{Unique_Unique_}\index{Unique!Unique()}}
\subparagraph*{check\_Unique()\label{Unique_check_Unique_}\index{Unique!check\ Unique()}}


\textbf{check\_Unique()} checks the correctness of the input parameters.



In case of errors it sets \$self-$>$status and additionally returns a non-true
returnvalue.

\paragraph*{AUTHORS\label{Unique_AUTHORS}\index{Unique!AUTHORS}}


Helmut Lichtenberg $<$heli@tzv.fal.de$>$

\subparagraph*{Unique\label{Unique_Unique}\index{Unique!Unique}}


Syntax:



Unique \$table \$data\_column [\$column=\$value]



Unique looks in database if the passed data is unique within this combination
of column(s).
\$table is the concerning database table.
\$data\_column is the column of the passed data. It therefore must *not* have
the =value.
To create composite keys you can specify additional columns
with the expected value. Note: char values have to be surrounded by "'".



Example:



Unique( employee name department='sale' salary=3000 \$data);



Returnvalues:

\begin{enumerate}

\item 

if \$data does exist more than once in this Unique combination in the table,


\item 

otherwise, also accepting NULL values.

\end{enumerate}
