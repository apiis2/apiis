\subsection{DESCRIPTION\label{DESCRIPTION}\index{DESCRIPTION}}


Auth.pm is method of record object and contains subroutines needed for the authentication process.

\subsection{SUBROUTINES\label{SUBROUTINES}\index{SUBROUTINES}}
\subsubsection*{\_auth\label{_auth}\index{\ auth}}
\begin{verbatim}
  This subroutine is responsible for whole authentication process.
  1.Get access rights from the database
  2.Dependant from sql action type
    (a) if DELETE then only recreates list of all clases on which user can 
        executed delete operations.This list is returned by "get_ar" and is 
        recreated to the hash structure.
    (b) if INSERT or UPDATE then checks access rights for the columns 
        defined in this statement
\end{verbatim}
\subsubsection*{check\_ar\label{check_ar}\index{check\ ar}}


Subroutine checks access rights for the columns.
Algorithm copmares column names from the sql statement to column names defined 
in the access hash and returns hash with clases in which user can executed this 
sql statement. Class is added to this hash only if user have defined access rights 
for all columns (from sql statement) in this class.



Returned hash for expand the where clause:

\begin{verbatim}
     @ext =(
                  (
                   COLUMN   => "class",
                   OPERATOR => "=",
                   VALUE    => PL,
                  ),
\end{verbatim}
\begin{verbatim}
                  (
                   COLUMN   => "class",
                   OPERATOR => "=",
                   VALUE    => DE,
                  )
                 )
\end{verbatim}
\subsubsection*{redo\_clases\label{redo_clases}\index{redo\ clases}}
\subsection{AUTHORS\label{AUTHORS}\index{AUTHORS}}


Marek Imialek $<$marek@tzv.fal.de$>$

\subsection{Apiis::Auth::Role.pm\label{Apiis::Auth::Role_pm}\index{Apiis::Auth::Role.pm}}




\subsubsection*{SYNOPSIS\label{Apiis::Auth::Role_pm_SYNOPSIS}\index{Apiis::Auth::Role pm!SYNOPSIS}}


\$role = Apiis::Auth::Role-$>$new(
                                role\_shortcut =$>$ \$role\_shortcut,
                               );

\subsubsection*{DESCRIPTION\label{Apiis::Auth::Role_pm_DESCRIPTION}\index{Apiis::Auth::Role pm!DESCRIPTION}}


This is a module for creating an object for handling user roles. These object are used in the authentication process.
All information about roles are taken from the Roles.conf file.

\subsubsection*{METHODS\label{Apiis::Auth::Role_pm_METHODS}\index{Apiis::Auth::Role pm!METHODS}}
\paragraph*{role\_shortcut
 returns the main role name  which is defined in Roles.conf  file\label{Apiis::Auth::Role_pm_role_shortcut_returns_the_main_role_name_which_is_defined_in_Roles_conf_file}\index{Apiis::Auth::Role pm!role\ shortcut
 returns the main role name  which is defined in Roles.conf  file}}
\paragraph*{short\_name
 returns short role name\label{Apiis::Auth::Role_pm_short_name_returns_short_role_name}\index{Apiis::Auth::Role pm!short\ name
 returns short role name}}
\paragraph*{long\_name
 returns long role name\label{Apiis::Auth::Role_pm_long_name_returns_long_role_name}\index{Apiis::Auth::Role pm!long\ name
 returns long role name}}
\paragraph*{description
 returns description of role\label{Apiis::Auth::Role_pm_description_returns_description_of_role}\index{Apiis::Auth::Role pm!description
 returns description of role}}
\paragraph*{policies
 returns policy numbers; policies number are taken from the role subsectione\label{Apiis::Auth::Role_pm_policies_returns_policy_numbers_policies_number_are_taken_from_the_role_subsectione}\index{Apiis::Auth::Role pm!policies
 returns policy numbers; policies number are taken from the role subsectione}}
\paragraph*{role\_type
  returns role type\label{Apiis::Auth::Role_pm_role_type_returns_role_type}\index{Apiis::Auth::Role pm!role\ type
  returns role type}}
\paragraph*{role\_id
  get next sequence value\label{Apiis::Auth::Role_pm_role_id_get_next_sequence_value}\index{Apiis::Auth::Role pm!role\ id
  get next sequence value}}
\paragraph*{db\_policy
 get policy for the database from db\_policy section\label{Apiis::Auth::Role_pm_db_policy_get_policy_for_the_database_from_db_policy_section}\index{Apiis::Auth::Role pm!db\ policy
 get policy for the database from db\ policy section}}
\paragraph*{os\_policy
 get policy for the operating system from os\_policy section\label{Apiis::Auth::Role_pm_os_policy_get_policy_for_the_operating_system_from_os_policy_section}\index{Apiis::Auth::Role pm!os\ policy
 get policy for the operating system from os\ policy section}}
\subsubsection*{AUTHORS\label{Apiis::Auth::Role_pm_AUTHORS}\index{Apiis::Auth::Role pm!AUTHORS}}


Marek Imialek $<$marek@tzv.fal.de$>$

\subsection{Apiis::Auth::AppAuth -- object for provading data about user access rights for the applications\label{Apiis::Auth::AppAuth_--_object_for_provading_data_about_user_access_rights_for_the_applications}\index{Apiis::Auth::AppAuth -- object for provading data about user access rights for the applications}}




\subsubsection*{SYNOPSIS\label{Apiis::Auth::AppAuth_--_object_for_provading_data_about_user_access_rights_for_the_applications_SYNOPSIS}\index{Apiis::Auth::AppAuth -- object for provading data about user access rights for the applications!SYNOPSIS}}


This object is used to check user access right for the applications which ara curentlly defined in the database.

\subsubsection*{DESCRIPTION\label{Apiis::Auth::AppAuth_--_object_for_provading_data_about_user_access_rights_for_the_applications_DESCRIPTION}\index{Apiis::Auth::AppAuth -- object for provading data about user access rights for the applications!DESCRIPTION}}


Object is created by the one of Apiis object method (\$apiis-$>$join\_auth('user\_login')). This creates Auth object
for the user which is curently log-in and join it to the \$apiis structure.

\subsubsection*{METHODS\label{Apiis::Auth::AppAuth_--_object_for_provading_data_about_user_access_rights_for_the_applications_METHODS}\index{Apiis::Auth::AppAuth -- object for provading data about user access rights for the applications!METHODS}}
\paragraph*{new (public)\label{Apiis::Auth::AppAuth_--_object_for_provading_data_about_user_access_rights_for_the_applications_new_public_}\index{Apiis::Auth::AppAuth -- object for provading data about user access rights for the applications!new (public)}}
\begin{verbatim}
  returns an object reference for a new Auth object.
\end{verbatim}
\paragraph*{\_get\_user\_roles (internal)\label{Apiis::Auth::AppAuth_--_object_for_provading_data_about_user_access_rights_for_the_applications__get_user_roles_internal_}\index{Apiis::Auth::AppAuth -- object for provading data about user access rights for the applications!\ get\ user\ roles (internal)}}
\begin{verbatim}
  retrieves all role_id from table 'roles' for current user
\end{verbatim}
\paragraph*{\_get\_user\_id (internal)\label{Apiis::Auth::AppAuth_--_object_for_provading_data_about_user_access_rights_for_the_applications__get_user_id_internal_}\index{Apiis::Auth::AppAuth -- object for provading data about user access rights for the applications!\ get\ user\ id (internal)}}
\begin{verbatim}
  retrieves current user id
\end{verbatim}
\paragraph*{\_get\_policy\_ids (internal)\label{Apiis::Auth::AppAuth_--_object_for_provading_data_about_user_access_rights_for_the_applications__get_policy_ids_internal_}\index{Apiis::Auth::AppAuth -- object for provading data about user access rights for the applications!\ get\ policy\ ids (internal)}}
\begin{verbatim}
  retrieves all policies for current role
\end{verbatim}
\paragraph*{print\_os\_actions (public)\label{Apiis::Auth::AppAuth_--_object_for_provading_data_about_user_access_rights_for_the_applications_print_os_actions_public_}\index{Apiis::Auth::AppAuth -- object for provading data about user access rights for the applications!print\ os\ actions (public)}}
\begin{verbatim}
  prints all applications or actions with their classes which user can execut
\end{verbatim}
\begin{verbatim}
      example: $apiis->Auth->print_os_actions;
\end{verbatim}
\paragraph*{os\_actions\label{Apiis::Auth::AppAuth_--_object_for_provading_data_about_user_access_rights_for_the_applications_os_actions}\index{Apiis::Auth::AppAuth -- object for provading data about user access rights for the applications!os\ actions}}
\begin{verbatim}
  method return list of all actions which are allowed for the user (if you run it without any parameter).
  If you run it with the parameter "action type" then you can get the list of allowed actions for this specified action type. 
  You can use following action type: program, form, rapor,t subroutine, www, action. Curently the action types are hard-coded 
  in the AccessControl.pm
\end{verbatim}
\begin{verbatim}
      example: $apiis->Auth->os_actions
               $apiis->Auth->os_actions('program')
\end{verbatim}
\paragraph*{types\_of\_actions (public)\label{Apiis::Auth::AppAuth_--_object_for_provading_data_about_user_access_rights_for_the_applications_types_of_actions_public_}\index{Apiis::Auth::AppAuth -- object for provading data about user access rights for the applications!types\ of\ actions (public)}}
\begin{verbatim}
  returns all type of actions which are curently allowed for the user.
\end{verbatim}
\begin{verbatim}
      example: $apiis->Auth->types_of_actions
\end{verbatim}
\paragraph*{check\_os\_action (public)\label{Apiis::Auth::AppAuth_--_object_for_provading_data_about_user_access_rights_for_the_applications_check_os_action_public_}\index{Apiis::Auth::AppAuth -- object for provading data about user access rights for the applications!check\ os\ action (public)}}
\begin{verbatim}
  check that user can executs action (action name is defined as a parameter).
\end{verbatim}
\begin{verbatim}
      example: $apiis->Auth->check_os_action('runall_ar.pl','program');
\end{verbatim}
\subsubsection*{AUTHORS\label{Apiis::Auth::AppAuth_--_object_for_provading_data_about_user_access_rights_for_the_applications_AUTHORS}\index{Apiis::Auth::AppAuth -- object for provading data about user access rights for the applications!AUTHORS}}


Marek Imialek $<$marek@tzv.fal.de$>$

\subsection{Apiis::Auth::AccessControl -- used by the runall.pl and access\_control.pl scripts to define user access rights\label{Apiis::Auth::AccessControl_--_used_by_the_runall_pl_and_access_control_pl_scripts_to_define_user_access_rights}\index{Apiis::Auth::AccessControl -- used by the runall.pl and access\ control.pl scripts to define user access rights}}




\subsubsection*{SYNOPSIS\label{Apiis::Auth::AccessControl_--_used_by_the_runall_pl_and_access_control_pl_scripts_to_define_user_access_rights_SYNOPSIS}\index{Apiis::Auth::AccessControl -- used by the runall pl and access control pl scripts to define user access rights!SYNOPSIS}}


Adding, deleting roles and users in the Apiis system.

\subsubsection*{DESCRIPTION\label{Apiis::Auth::AccessControl_--_used_by_the_runall_pl_and_access_control_pl_scripts_to_define_user_access_rights_DESCRIPTION}\index{Apiis::Auth::AccessControl -- used by the runall pl and access control pl scripts to define user access rights!DESCRIPTION}}


These subroutines are used to define access rights in the system. New roles and users are created on the basis of information
which are defined in the \$APIIS\_LOCAL/etc/Roles.conf. Roles and users name are set as a parameters.

\subsubsection*{SUBROUTINES\label{Apiis::Auth::AccessControl_--_used_by_the_runall_pl_and_access_control_pl_scripts_to_define_user_access_rights_SUBROUTINES}\index{Apiis::Auth::AccessControl -- used by the runall pl and access control pl scripts to define user access rights!SUBROUTINES}}
\paragraph*{access\_rights\label{Apiis::Auth::AccessControl_--_used_by_the_runall_pl_and_access_control_pl_scripts_to_define_user_access_rights_access_rights}\index{Apiis::Auth::AccessControl -- used by the runall pl and access control pl scripts to define user access rights!access\ rights}}
\begin{verbatim}
 this subroutine can be used to define access rights directly in the code without access_control.pl script
\end{verbatim}
\paragraph*{creates\_schema\label{Apiis::Auth::AccessControl_--_used_by_the_runall_pl_and_access_control_pl_scripts_to_define_user_access_rights_creates_schema}\index{Apiis::Auth::AccessControl -- used by the runall pl and access control pl scripts to define user access rights!creates\ schema}}
\begin{verbatim}
  this subroutine creates individual user schema
\end{verbatim}
\paragraph*{creates\_user\label{Apiis::Auth::AccessControl_--_used_by_the_runall_pl_and_access_control_pl_scripts_to_define_user_access_rights_creates_user}\index{Apiis::Auth::AccessControl -- used by the runall pl and access control pl scripts to define user access rights!creates\ user}}
\begin{verbatim}
  this subroutine adds new user.
\end{verbatim}
\paragraph*{creates\_role\label{Apiis::Auth::AccessControl_--_used_by_the_runall_pl_and_access_control_pl_scripts_to_define_user_access_rights_creates_role}\index{Apiis::Auth::AccessControl -- used by the runall pl and access control pl scripts to define user access rights!creates\ role}}
\begin{verbatim}
  this subroutine adds new role.
\end{verbatim}
\paragraph*{assigns\_role\label{Apiis::Auth::AccessControl_--_used_by_the_runall_pl_and_access_control_pl_scripts_to_define_user_access_rights_assigns_role}\index{Apiis::Auth::AccessControl -- used by the runall pl and access control pl scripts to define user access rights!assigns\ role}}
\begin{verbatim}
  this subroutine assigns role to the user.
\end{verbatim}
\paragraph*{creates\_db\_policies\label{Apiis::Auth::AccessControl_--_used_by_the_runall_pl_and_access_control_pl_scripts_to_define_user_access_rights_creates_db_policies}\index{Apiis::Auth::AccessControl -- used by the runall pl and access control pl scripts to define user access rights!creates\ db\ policies}}
\begin{verbatim}
 This subroutine reads database policies from the Roles.conf file and adds it to the database.
 Only these policies are added which are defined for the current role.
\end{verbatim}
\paragraph*{asigns\_db\_policies\label{Apiis::Auth::AccessControl_--_used_by_the_runall_pl_and_access_control_pl_scripts_to_define_user_access_rights_asigns_db_policies}\index{Apiis::Auth::AccessControl -- used by the runall pl and access control pl scripts to define user access rights!asigns\ db\ policies}}
\begin{verbatim}
  this subroutine asigns databse policies to the current role.
\end{verbatim}
\paragraph*{creates\_os\_policies\label{Apiis::Auth::AccessControl_--_used_by_the_runall_pl_and_access_control_pl_scripts_to_define_user_access_rights_creates_os_policies}\index{Apiis::Auth::AccessControl -- used by the runall pl and access control pl scripts to define user access rights!creates\ os\ policies}}
\begin{verbatim}
 This subroutine reads os policies from the Roles.conf file and adds it to the database.
 Only these policies are added which are defined for the current role
\end{verbatim}
\paragraph*{asigns\_os\_policies\label{Apiis::Auth::AccessControl_--_used_by_the_runall_pl_and_access_control_pl_scripts_to_define_user_access_rights_asigns_os_policies}\index{Apiis::Auth::AccessControl -- used by the runall pl and access control pl scripts to define user access rights!asigns\ os\ policies}}
\begin{verbatim}
  this subroutine asigns operating sytstem policies to the current role.
\end{verbatim}
\paragraph*{creates\_access\_view\label{Apiis::Auth::AccessControl_--_used_by_the_runall_pl_and_access_control_pl_scripts_to_define_user_access_rights_creates_access_view}\index{Apiis::Auth::AccessControl -- used by the runall pl and access control pl scripts to define user access rights!creates\ access\ view}}
\begin{verbatim}
  this subroutine creates user access view.
\end{verbatim}
\paragraph*{check\_policies\label{Apiis::Auth::AccessControl_--_used_by_the_runall_pl_and_access_control_pl_scripts_to_define_user_access_rights_check_policies}\index{Apiis::Auth::AccessControl -- used by the runall pl and access control pl scripts to define user access rights!check\ policies}}
\begin{verbatim}
 This subroutine checks that policies which we want to load 
 already are defined in the database.
\end{verbatim}
\paragraph*{del\_user\label{Apiis::Auth::AccessControl_--_used_by_the_runall_pl_and_access_control_pl_scripts_to_define_user_access_rights_del_user}\index{Apiis::Auth::AccessControl -- used by the runall pl and access control pl scripts to define user access rights!del\ user}}
\begin{verbatim}
  this subroutine deletes user from the system.
\end{verbatim}
\paragraph*{del\_role\label{Apiis::Auth::AccessControl_--_used_by_the_runall_pl_and_access_control_pl_scripts_to_define_user_access_rights_del_role}\index{Apiis::Auth::AccessControl -- used by the runall pl and access control pl scripts to define user access rights!del\ role}}
\begin{verbatim}
  this subroutine deletes role from the system.
\end{verbatim}
\paragraph*{del\_role\_from\_user\label{Apiis::Auth::AccessControl_--_used_by_the_runall_pl_and_access_control_pl_scripts_to_define_user_access_rights_del_role_from_user}\index{Apiis::Auth::AccessControl -- used by the runall pl and access control pl scripts to define user access rights!del\ role\ from\ user}}
\begin{verbatim}
  this subroutine revoke role from the user.
\end{verbatim}
\paragraph*{revoke\_priv\label{Apiis::Auth::AccessControl_--_used_by_the_runall_pl_and_access_control_pl_scripts_to_define_user_access_rights_revoke_priv}\index{Apiis::Auth::AccessControl -- used by the runall pl and access control pl scripts to define user access rights!revoke\ priv}}
\begin{verbatim}
  this subroutine revoke user privileges from the PostgreSQL.
\end{verbatim}
\paragraph*{select\_db\label{Apiis::Auth::AccessControl_--_used_by_the_runall_pl_and_access_control_pl_scripts_to_define_user_access_rights_select_db}\index{Apiis::Auth::AccessControl -- used by the runall pl and access control pl scripts to define user access rights!select\ db}}
\begin{verbatim}
  this subroutine print information about users and roles which are curently defined in the system.
\end{verbatim}
\paragraph*{public\_views\label{Apiis::Auth::AccessControl_--_used_by_the_runall_pl_and_access_control_pl_scripts_to_define_user_access_rights_public_views}\index{Apiis::Auth::AccessControl -- used by the runall pl and access control pl scripts to define user access rights!public\ views}}
\begin{verbatim}
  this subroutine creates system of the public views in the user schema.
\end{verbatim}
\paragraph*{col\_in\_classes\label{Apiis::Auth::AccessControl_--_used_by_the_runall_pl_and_access_control_pl_scripts_to_define_user_access_rights_col_in_classes}\index{Apiis::Auth::AccessControl -- used by the runall pl and access control pl scripts to define user access rights!col\ in\ classes}}
\begin{verbatim}
  This subroutine querys DB about allowed columns for current user. Query 
  is executed with select action and current table name as a paramaters. 
  Returned list of columns is added to the hash with class names.
\end{verbatim}
\paragraph*{main\_columns\label{Apiis::Auth::AccessControl_--_used_by_the_runall_pl_and_access_control_pl_scripts_to_define_user_access_rights_main_columns}\index{Apiis::Auth::AccessControl -- used by the runall pl and access control pl scripts to define user access rights!main\ columns}}
\begin{verbatim}
  This subroutine creates main list of column for the current table . Only these column names 
  are taken to the list, to which user have access rights (in several  classes). If table has 
  translation table then this list is created from two tables. This list is needed to 
  creates structure of view via UNION (we have to have list of all columne which will be in the 
  view).
\end{verbatim}
\paragraph*{class\_columns\label{Apiis::Auth::AccessControl_--_used_by_the_runall_pl_and_access_control_pl_scripts_to_define_user_access_rights_class_columns}\index{Apiis::Auth::AccessControl -- used by the runall pl and access control pl scripts to define user access rights!class\ columns}}
\begin{verbatim}
  Compares main column list to the column allowed for the user 
  in several classes. Columns list is created for each class. Columns order for the each class 
  have to be the same like in main list. Value NULL is putted to the list if some column doesn't 
  occure on the main list and user haven't access rights to this column in this class
  This all is needed to create finall sql where "union all" expresion is used.
\end{verbatim}
\paragraph*{creates\_view\label{Apiis::Auth::AccessControl_--_used_by_the_runall_pl_and_access_control_pl_scripts_to_define_user_access_rights_creates_view}\index{Apiis::Auth::AccessControl -- used by the runall pl and access control pl scripts to define user access rights!creates\ view}}
\begin{verbatim}
  This subroutine creates user view for current table.
\end{verbatim}
\paragraph*{drop\_views\label{Apiis::Auth::AccessControl_--_used_by_the_runall_pl_and_access_control_pl_scripts_to_define_user_access_rights_drop_views}\index{Apiis::Auth::AccessControl -- used by the runall pl and access control pl scripts to define user access rights!drop\ views}}
\begin{verbatim}
  this subroutine drop current view from user schema.
\end{verbatim}
\paragraph*{query\_db\label{Apiis::Auth::AccessControl_--_used_by_the_runall_pl_and_access_control_pl_scripts_to_define_user_access_rights_query_db}\index{Apiis::Auth::AccessControl -- used by the runall pl and access control pl scripts to define user access rights!query\ db}}
\begin{verbatim}
  this subroutine executs system SQL statements.
\end{verbatim}
\paragraph*{change\_password\label{Apiis::Auth::AccessControl_--_used_by_the_runall_pl_and_access_control_pl_scripts_to_define_user_access_rights_change_password}\index{Apiis::Auth::AccessControl -- used by the runall pl and access control pl scripts to define user access rights!change\ password}}
\begin{verbatim}
  this subroutine is used to change user password.
\end{verbatim}
\paragraph*{creates\_views\_v\_\label{Apiis::Auth::AccessControl_--_used_by_the_runall_pl_and_access_control_pl_scripts_to_define_user_access_rights_creates_views_v_}\index{Apiis::Auth::AccessControl -- used by the runall pl and access control pl scripts to define user access rights!creates\ views\ v\ }}


this code was copied from MakeSQL.pm module and partialy changed.  This
subroutine is used to create "v\_" view under user schema.



Note (2008-04-08 heli):
As MakeSQL changed to allow self-referencing foreign keys I also had to change
this part of it. :\^{}(

\subsubsection*{AUTHOR\label{Apiis::Auth::AccessControl_--_used_by_the_runall_pl_and_access_control_pl_scripts_to_define_user_access_rights_AUTHOR}\index{Apiis::Auth::AccessControl -- used by the runall pl and access control pl scripts to define user access rights!AUTHOR}}


Marek Imialek $<$marek@tzv.fal.de$>$

