\subsection{Apiis::Form::Init -- base package for Form objects of all types\label{Apiis::Form::Init_--_base_package_for_Form_objects_of_all_types}\index{Apiis::Form::Init -- base package for Form objects of all types}}




\subsubsection*{SYNOPSIS\label{Apiis::Form::Init_--_base_package_for_Form_objects_of_all_types_SYNOPSIS}\index{Apiis::Form::Init -- base package for Form objects of all types!SYNOPSIS}}


This base package provides the main functionality and methods needed for
all Form object types (Tk, Html, etc.).

\subsubsection*{DESCRIPTION\label{Apiis::Form::Init_--_base_package_for_Form_objects_of_all_types_DESCRIPTION}\index{Apiis::Form::Init -- base package for Form objects of all types!DESCRIPTION}}


The generic public methods of this base class are described below. Most of
them can be inherited by subsequent Form objects.

\subsubsection*{METHODS\label{Apiis::Form::Init_--_base_package_for_Form_objects_of_all_types_METHODS}\index{Apiis::Form::Init -- base package for Form objects of all types!METHODS}}
\paragraph*{exists\_fieldtype (public)\label{Apiis::Form::Init_--_base_package_for_Form_objects_of_all_types_exists_fieldtype_public_}\index{Apiis::Form::Init -- base package for Form objects of all types!exists\ fieldtype (public)}}


Subroutines to check, if the passed type exists in the
list of hardcoded fieldtypes.

\paragraph*{exists\_ds\_type (internal)\label{Apiis::Form::Init_--_base_package_for_Form_objects_of_all_types_exists_ds_type_internal_}\index{Apiis::Form::Init -- base package for Form objects of all types!exists\ ds\ type (internal)}}


Returns true if the passed parameter is in the list of hardcoded DataSource
Types (like sql, record, function, none).

\paragraph*{is\_a\_listfield (internal)\label{Apiis::Form::Init_--_base_package_for_Form_objects_of_all_types_is_a_listfield_internal_}\index{Apiis::Form::Init -- base package for Form objects of all types!is\ a\ listfield (internal)}}


Returns true if the passed parameter is in the list of hardcoded fields, that
have a list character (like ScrollingList, BrowseEntry, etc.).

\paragraph*{is\_misc\_blockelement (internal)\label{Apiis::Form::Init_--_base_package_for_Form_objects_of_all_types_is_misc_blockelement_internal_}\index{Apiis::Form::Init -- base package for Form objects of all types!is\ misc\ blockelement (internal)}}


Subroutines to check, if the passed element exists in the
list of hardcoded Block elements, which are not Field, DataSource, etc..

\paragraph*{new (public)\label{Apiis::Form::Init_--_base_package_for_Form_objects_of_all_types_new_public_}\index{Apiis::Form::Init -- base package for Form objects of all types!new (public)}}


\textbf{new} creates a new form object.
Input parameter is an anonymous hash with the key 'xmlfile' and the path to this
file as its value.



The \textbf{new} method of Init.pm is not invoked directly but via inheritance through
the widget specific modules.



Example:

\begin{verbatim}
   my $form_obj = Apiis::Form::Tk->new(
      xmlfile => '/path/to/xmlfile.frm'
   );
\end{verbatim}
\paragraph*{\_init (internal)\label{Apiis::Form::Init_--_base_package_for_Form_objects_of_all_types__init_internal_}\index{Apiis::Form::Init -- base package for Form objects of all types!\ init (internal)}}


\textbf{\_init()} is only invoked if you want to create an object of type
Apiis::Form directly. As this is only a base class an error is yielded.

\paragraph*{xmlfile (public)\label{Apiis::Form::Init_--_base_package_for_Form_objects_of_all_types_xmlfile_public_}\index{Apiis::Form::Init -- base package for Form objects of all types!xmlfile (public)}}


\textbf{xmlfile()} returns the full path of the xml file for this form.

\paragraph*{add\_formlib\_path (public)\label{Apiis::Form::Init_--_base_package_for_Form_objects_of_all_types_add_formlib_path_public_}\index{Apiis::Form::Init -- base package for Form objects of all types!add\ formlib\ path (public)}}


\textbf{add\_formlib\_path} will prepended some Form/Project specific directories to
the library search for Form modules in up to down order. So the most specific
will be searched first and wins over the more generic ones. Form modules for
example for Form\_0 and GUI-type Tk are searched in:

\begin{verbatim}
   $APIIS_HOME/lib/Apiis/Form/Tk/Form_0/
   $APIIS_HOME/lib/Apiis/Form/Tk/
   $APIIS_HOME/lib/Apiis/Form/Form_0/
   $APIIS_HOME/lib/Apiis/Form/Event/
\end{verbatim}


As Form-name and GUI-type are determined at runtime, the addition of this
search paths also have to happen at runtime.



\textbf{add\_formlib\_path} expects as input parameters:

\begin{verbatim}
   1. the GUI-type (e.g. Tk, HTML, Qt )
   2. the form name as in $self->formname
\end{verbatim}


Usage:

\begin{verbatim}
   $self->add_formlib_path( 'Tk', $self->formname );
\end{verbatim}
\paragraph*{form\_status\_msg (public)\label{Apiis::Form::Init_--_base_package_for_Form_objects_of_all_types_form_status_msg_public_}\index{Apiis::Form::Init -- base package for Form objects of all types!form\ status\ msg (public)}}


\textbf{form\_status\_msg} is a write-only method to put form-global status messages
somewhere. They can be displayed e.g. in a status line (textfield) in the form.



Example:

\begin{verbatim}
   $self->form_status_msg('Retrieved 20 records');
\end{verbatim}
\paragraph*{GetValue (public)\label{Apiis::Form::Init_--_base_package_for_Form_objects_of_all_types_GetValue_public_}\index{Apiis::Form::Init -- base package for Form objects of all types!GetValue (public)}}


\textbf{GetValue} is the main method to access the configuration data and the data
references of this form object. All the real elements (not the pseudo auxiliary
elements) of the xml file have a unique name and are accessible trough this
name. All elements without the identifying name are flattened into their parent
element. Their attributes become attributes of the parent.



Syntax:

\begin{verbatim}
   my $attr_value = $form_obj->GetValue($elementname, $attribut);
\end{verbatim}


Example:

\begin{verbatim}
   my $label = $form_obj->GetValue('Field_1', 'Label');
\end{verbatim}


\textbf{GetValue} returns also the values for dynamically created items like
'\_field\_list' for each block or '\_column\_list' for each datasource.



Example:

\begin{verbatim}
   foreach my $field ( @{ $form_obj->GetValue( 'Block_0', '_field_list')} ){
      # do something
   }
\end{verbatim}


The list values return an array reference so they must be dereferenced for usage
in loops.

\paragraph*{SetValue (public)\label{Apiis::Form::Init_--_base_package_for_Form_objects_of_all_types_SetValue_public_}\index{Apiis::Form::Init -- base package for Form objects of all types!SetValue (public)}}


\textbf{SetValue} is the counterpart to GetValue.



Syntax:

\begin{verbatim}
   my $old_value = $form_obj->SetValue($elementname, $attribut, $new_value);
\end{verbatim}


\textbf{SetValue} returns the old value.

\paragraph*{IncValue $|$ DecValue (public)\label{Apiis::Form::Init_--_base_package_for_Form_objects_of_all_types_IncValue_DecValue_public_}\index{Apiis::Form::Init -- base package for Form objects of all types!IncValue $|$ DecValue (public)}}


Increment and decrement values in the flattened structure.



Syntax:

\begin{verbatim}
   $form_obj->IncValue($elementname, $attribut);
   $form_obj->DecValue($elementname, $attribut);
\end{verbatim}
\paragraph*{GetEvent (public)\label{Apiis::Form::Init_--_base_package_for_Form_objects_of_all_types_GetEvent_public_}\index{Apiis::Form::Init -- base package for Form objects of all types!GetEvent (public)}}


\textbf{GetEvent} should make access to event definitions at certain levels easy.



Syntax:

\begin{verbatim}
   my $events_ref = $form_obj->GetEvent($elementname, 'OnClick');
\end{verbatim}


\textbf{GetEvent} returns an arrayreference to all 'OnClick'-Eventnames
of \$elementname or undef if none exists.

\paragraph*{RunEvent (public)\label{Apiis::Form::Init_--_base_package_for_Form_objects_of_all_types_RunEvent_public_}\index{Apiis::Form::Init -- base package for Form objects of all types!RunEvent (public)}}


input: hash reference with required keys: elementname, eventtype



output: array reference with the names of the processed events

\paragraph*{gui\_type (public)\label{Apiis::Form::Init_--_base_package_for_Form_objects_of_all_types_gui_type_public_}\index{Apiis::Form::Init -- base package for Form objects of all types!gui\ type (public)}}


\textbf{gui\_type} returns (and sets) the type of the running Graphical User
Interface, e.g. Tk, HTML.

\paragraph*{top (public)\label{Apiis::Form::Init_--_base_package_for_Form_objects_of_all_types_top_public_}\index{Apiis::Form::Init -- base package for Form objects of all types!top (public)}}


\textbf{top} returns (and sets) a reference to toplevel window (Tk, query =$>$ HTML).

\paragraph*{formname (public)\label{Apiis::Form::Init_--_base_package_for_Form_objects_of_all_types_formname_public_}\index{Apiis::Form::Init -- base package for Form objects of all types!formname (public)}}


\textbf{formname} returns the name of the Form.



For the consistency of method names, there is also a method \textbf{formnames},
which returns an arrayref of the list of formnames.



Both methods might be seldom used.

\paragraph*{generalname (public)\label{Apiis::Form::Init_--_base_package_for_Form_objects_of_all_types_generalname_public_}\index{Apiis::Form::Init -- base package for Form objects of all types!generalname (public)}}


\textbf{generalname} returns the name of the General section. There could only be
one General section.



For the consistency of method names, there is also a method \textbf{generalnames},
which returns an arrayref of the list of generalnames.

\paragraph*{blocknames $|$ datasourcenames $|$ fieldnames $|$columnnames (public)\label{Apiis::Form::Init_--_base_package_for_Form_objects_of_all_types_blocknames_datasourcenames_fieldnames_columnnames_public_}\index{Apiis::Form::Init -- base package for Form objects of all types!blocknames $|$ datasourcenames $|$ fieldnames $|$columnnames (public)}}


These methods all return an arrayref of the list of names.

\paragraph*{misc\_blockelements (public)\label{Apiis::Form::Init_--_base_package_for_Form_objects_of_all_types_misc_blockelements_public_}\index{Apiis::Form::Init -- base package for Form objects of all types!misc\ blockelements (public)}}


\textbf{misc\_blockelements} return an array (or reference) to a list of non-Field
elements of a block (Line, Image, Frage, etc.).

\paragraph*{master\_detail (public)\label{Apiis::Form::Init_--_base_package_for_Form_objects_of_all_types_master_detail_public_}\index{Apiis::Form::Init -- base package for Form objects of all types!master\ detail (public)}}


Form level flag, if this form is a more complex one with master/detail
relationships.



Usage:

\begin{verbatim}
   $self->master_detail(1); # set flag
   &easy_going if not $self->master_detail;
\end{verbatim}
\paragraph*{query\_block\_order (public)\label{Apiis::Form::Init_--_base_package_for_Form_objects_of_all_types_query_block_order_public_}\index{Apiis::Form::Init -- base package for Form objects of all types!query\ block\ order (public)}}


\textbf{query\_block\_order} return the blocks in the right order for Master/Detail
handling. The query order depends on the master/detail relationships between
blocks. This order is determined in Event/Query.pm, when the first query is
started. After this it is usually used readonly.

\paragraph*{encode\_list\_ref $|$ decode\_list\_ref\label{Apiis::Form::Init_--_base_package_for_Form_objects_of_all_types_encode_list_ref_decode_list_ref}\index{Apiis::Form::Init -- base package for Form objects of all types!encode\ list\ ref $|$ decode\ list\ ref}}


\textbf{encode\_list\_ref} and \textbf{decode\_list\_ref} exchange the primary/foreignkey
value with the more readable one and vice versa. These methods are used, if a
field-specific DataSource exists.



Both take as input parameters:

\begin{verbatim}
   1. Fieldname
   2. The data to en/de-code
\end{verbatim}


They return the en/de-coded value.

\paragraph*{fieldtype (internal)\label{Apiis::Form::Init_--_base_package_for_Form_objects_of_all_types_fieldtype_internal_}\index{Apiis::Form::Init -- base package for Form objects of all types!fieldtype (internal)}}


\textbf{fieldtype} returns the widget-set specific fieldtype for a passed metatype.



Example:

\begin{verbatim}
   my $ft = $self->fieldtype('textfield');   # returns 'TextField' for Tk
\end{verbatim}


All metatypes are in lower case.



To make error messages more helpfull, you should add a second parameter,
the fieldname:

\begin{verbatim}
   my $ft = $self->fieldtype( $type, $fieldname );
\end{verbatim}


If no error occurs, the fieldname is simply ignored, otherwise it will
give a valueable hint, where to search in the XML file.

\paragraph*{insert\_block $|$ insert\_form $|$ update\_block $|$ query\_block $|$ clear\_block $|$ clear\_form (public)\label{Apiis::Form::Init_--_base_package_for_Form_objects_of_all_types_insert_block_insert_form_update_block_query_block_clear_block_clear_form_public_}\index{Apiis::Form::Init -- base package for Form objects of all types!insert\ block $|$ insert\ form $|$ update\ block $|$ query\ block $|$ clear\ block $|$ clear\ form (public)}}


These methods handle form events on block level, initiated by the user. These
are the public interfaces, the real methods are rolled out into the
Apiis::Form::Event namespace. See details there.



The blockname is provided during the invocation of the do\_$<$commands$>$ in the
widget-specific button handling, e.g. in Tk/Button.pm.

\paragraph*{insert\_blocks $|$ insert\_form $|$ clear\_form (public)\label{Apiis::Form::Init_--_base_package_for_Form_objects_of_all_types_insert_blocks_insert_form_clear_form_public_}\index{Apiis::Form::Init -- base package for Form objects of all types!insert\ blocks $|$ insert\ form $|$ clear\ form (public)}}


These are the same methods except that they act on multiple/all blocks.
The blocknames for \textbf{insert\_blocks} are defined in the xml file.

\paragraph*{next $|$ prev $|$ first $|$ last (public)\label{Apiis::Form::Init_--_base_package_for_Form_objects_of_all_types_next_prev_first_last_public_}\index{Apiis::Form::Init -- base package for Form objects of all types!next $|$ prev $|$ first $|$ last (public)}}


These methods are for navigation through the records while querying data.

\paragraph*{return\_value (public)\label{Apiis::Form::Init_--_base_package_for_Form_objects_of_all_types_return_value_public_}\index{Apiis::Form::Init -- base package for Form objects of all types!return\ value (public)}}


\textbf{return\_value} stores and gives back a SCALAR value, which could be either a
simple scalar, a reference to a hash, to an array or even to another object.
It's up to the configuration, what type of return value is expected.

\subsubsection*{CreateCSSProperties\label{CreateCSSProperties}\index{CreateCSSProperties}}


This function has an object name as argument. It fetchs the object about getobject(\$name) and make a loop over the css-children-objects. After test whether the children-objects (color, format, position, miscellaneous, text) are valid each methodes of the objects were ask if it has an entry. About hash \%css the correct css-name will fetched. After loop all properties were collect and return as STYLE-element

