\subsection{Apiis.pm\label{Apiis_pm}\index{Apiis.pm}}




\subsubsection*{SYNOPSIS\label{Apiis_pm_SYNOPSIS}\index{Apiis pm!SYNOPSIS}}
\begin{verbatim}
   use Apiis;
   Apiis->initialize( VERSION => '$Revision: 1.4 $' );
\end{verbatim}
\subsubsection*{DESCRIPTION\label{Apiis_pm_DESCRIPTION}\index{Apiis pm!DESCRIPTION}}


\textbf{initialize} is the primary method for executables to load the Apiis
system. It does basic checking, creates, and exports the \$apiis object into
the main namespace.



To avoid numerous nasty error messages you are strongly advised to start
your program with this BEGIN block:

\begin{verbatim}
   BEGIN {
      use Env qw( APIIS_HOME );
      die "\n\tAPIIS_HOME is not set!\n\n" unless $APIIS_HOME;
      push @INC, "$APIIS_HOME/lib";
   }
\end{verbatim}


This catches errors due to an unset APIIS\_HOME environment variable and
adds \$APIIS\_HOME/lib to your library path to find the Apiis modules.

\subsubsection*{SUBROUTINES\label{Apiis_pm_SUBROUTINES}\index{Apiis pm!SUBROUTINES}}
\paragraph*{initialize\label{Apiis_pm_initialize}\index{Apiis pm!initialize}}


\textbf{initialize} loads Apiis::Init, creates a new Apiis::Init object and
assigns it to the global variable \$apiis, which is exported by default.
Also exported is the global subroutine \_\_() for nationalisation of the
code.



\textbf{initialize} currently takes one (hash) argument:

\begin{verbatim}
   Apiis->initialize( VERSION => '$Revision: 1.4 $' );
\end{verbatim}


It propagetes the cvs version as the program version and can be retrieved
with \$apiis-$>$version.

\subsubsection*{Author\label{Apiis_pm_Author}\index{Apiis pm!Author}}


Helmut Lichtenberg $<$heli@tzv.fal.de$>$

\subsection{Apiis::Init -- Basic initialisation object for the complete APIIS structure\label{Apiis::Init_--_Basic_initialisation_object_for_the_complete_APIIS_structure}\index{Apiis::Init -- Basic initialisation object for the complete APIIS structure}}




\subsubsection*{SYNOPSIS\label{Apiis::Init_--_Basic_initialisation_object_for_the_complete_APIIS_structure_SYNOPSIS}\index{Apiis::Init -- Basic initialisation object for the complete APIIS structure!SYNOPSIS}}
\begin{verbatim}
   our $apiis = Apiis::Init->new(
     version     => $version,
         programname => $programname
   );
\end{verbatim}


This is the basic module for spreading the APIIS configuration during
runtime.  It is invoked automatically if you run the common
initialisation block which includes \$APIIS\_HOME/lib/apiis\_init.pm. You
can access this basic object via the global variable \$apiis.

\subsubsection*{DESCRIPTION\label{Apiis::Init_--_Basic_initialisation_object_for_the_complete_APIIS_structure_DESCRIPTION}\index{Apiis::Init -- Basic initialisation object for the complete APIIS structure!DESCRIPTION}}


Apiis::Init creates an internal structure and public methods to access
this structure.



Public and internal methods are:

\subsubsection*{INTERNAL METHODS\label{Apiis::Init_--_Basic_initialisation_object_for_the_complete_APIIS_structure_INTERNAL_METHODS}\index{Apiis::Init -- Basic initialisation object for the complete APIIS structure!INTERNAL METHODS}}
\paragraph*{new (mostly internal)\label{Apiis::Init_--_Basic_initialisation_object_for_the_complete_APIIS_structure_new_mostly_internal_}\index{Apiis::Init -- Basic initialisation object for the complete APIIS structure!new (mostly internal)}}


new creates the object where we usually refer to as \$apiis.

\paragraph*{\_init (internal)\label{Apiis::Init_--_Basic_initialisation_object_for_the_complete_APIIS_structure__init_internal_}\index{Apiis::Init -- Basic initialisation object for the complete APIIS structure!\ init (internal)}}


\_init does the main initialization and creates the internal structure for:



APIIS\_HOME os\_user version programname date\_format
entry\_views reserved\_strings language codes\_table browser fileselector



This is done by querying parameter from the operating system (username) and
the user environment (APIIS\_HOME). APIIS\_LOCAL is set after a certain
project is selected and the model file is joined into \$apiis.



The main resources for this basic structure are the configuration files
\$APIIS\_HOME/etc/apiisrc and later \$APIIS\_LOCAL/etc/apiisrc.

\paragraph*{\_get\_user\_from\_os (internal)\label{Apiis::Init_--_Basic_initialisation_object_for_the_complete_APIIS_structure__get_user_from_os_internal_}\index{Apiis::Init -- Basic initialisation object for the complete APIIS structure!\ get\ user\ from\ os (internal)}}


The username \$apiis-$>$os\_user is determined by the operating system. This is
mainly needed for initial log messages, who started the program.

\paragraph*{projects (public)\label{Apiis::Init_--_Basic_initialisation_object_for_the_complete_APIIS_structure_projects_public_}\index{Apiis::Init -- Basic initialisation object for the complete APIIS structure!projects (public)}}


Returns the names of the projects defined in \$APIIS\_HOME/etc/apiisrc.

\paragraph*{project (public)\label{Apiis::Init_--_Basic_initialisation_object_for_the_complete_APIIS_structure_project_public_}\index{Apiis::Init -- Basic initialisation object for the complete APIIS structure!project (public)}}


Returns the \$APIIS\_LOCAL path for a specific project and is therefore mostly
redundant with \$apiis-$>$APIIS\_LOCAL().



Example:

\begin{verbatim}
   $local_path = $apiis->project('ref_breedprg');
\end{verbatim}
\paragraph*{formpath (public)\label{Apiis::Init_--_Basic_initialisation_object_for_the_complete_APIIS_structure_formpath_public_}\index{Apiis::Init -- Basic initialisation object for the complete APIIS structure!formpath (public)}}


Returns the default path for a specific project where the form definitions are
stored, usually at \$APIIS\_LOCAL/etc/forms. This default location is set during
join\_model.



It can be set to a different value with:



Example:

\begin{verbatim}
   $apiis->formpath( './forms' );
\end{verbatim}
\paragraph*{l10n\_init (public)\label{Apiis::Init_--_Basic_initialisation_object_for_the_complete_APIIS_structure_l10n_init_public_}\index{Apiis::Init -- Basic initialisation object for the complete APIIS structure!l10n\ init (public)}}


\textbf{l10n\_init} does the localisation from Apiis::I18N::L10N. The language is
passed as input parameter.



The failure handler for Locale::Maketext is set to return the untranslated
english string (default language).



Also the defined projects translations table is imported into the l10n
schema.



Input: language



Output: none

\paragraph*{l10n\_import (public)\label{Apiis::Init_--_Basic_initialisation_object_for_the_complete_APIIS_structure_l10n_import_public_}\index{Apiis::Init -- Basic initialisation object for the complete APIIS structure!l10n\ import (public)}}


\textbf{l10n\_import} imports an additional lexicon. This is usually done by
\textbf{l10n\_init}. In case you want to load another lexicon, use \textbf{l10n\_import}.



Input:

\begin{verbatim}
   1. language
   2. file
\end{verbatim}


Output: none



Example:

\begin{verbatim}
   $self->l10n_import( $lang, $lexicon ) if -f $lexicon;
\end{verbatim}
\paragraph*{\_\_()\label{Apiis::Init_--_Basic_initialisation_object_for_the_complete_APIIS_structure__}\index{Apiis::Init -- Basic initialisation object for the complete APIIS structure!\ \ ()}}


After initialization of the language handle \$lh:

\begin{verbatim}
   $lh = Apiis::L10N->get_handle( $apiis->language );
\end{verbatim}


you could write for localising text:

\begin{verbatim}
   print $lh->maketext('Just another Perl hacker'), "\n";
\end{verbatim}


To make it more convenient I created a shortcut wrapper around this,
the subroutine \_\_(). So you can write:

\begin{verbatim}
   print __('Just another Perl hacker'), "\n";
\end{verbatim}


Note: I must use an anonymous subroutine to have access to \$lh. See
'Programming Perl', p. 976 for error message: 'Variable "\$lh" will not stay
shared'.



Note2: The bare underscore \_ is treated specially, as it is always forced into
the package main ( like \$\_, @\_ ). See "Programming Perl', p. 591.
So we don't have to export it.



Note3: The single underscore \_ produced errors several times when it clashed
with the "Perl special filehandle used to cache the information from the
last successfull stat, lstat, or file test operator". ('Programming Perl',
p. 657).
This global underline subroutine is used e.g. in the CPAN or CGI modules. So
it's better to *not* use \_() for localisation. Preferred shortcut now is
\_\_().  (5. Aug. 2004 - heli)

\paragraph*{\_add\_obj (internal)\label{Apiis::Init_--_Basic_initialisation_object_for_the_complete_APIIS_structure__add_obj_internal_}\index{Apiis::Init -- Basic initialisation object for the complete APIIS structure!\ add\ obj (internal)}}


\$self-$>$\_add\_obj is used to mount an additional object structure into the
apiis core structure. An example is the addition of the model file
information under \$apiis-$>$Model.



usage:

\begin{verbatim}
   $self->_add_obj(
      Model  => [ $mod_obj ],
      caller => [ $package, $file, $line ]
   );
\end{verbatim}
\subsubsection*{PUBLIC METHODS\label{Apiis::Init_--_Basic_initialisation_object_for_the_complete_APIIS_structure_PUBLIC_METHODS}\index{Apiis::Init -- Basic initialisation object for the complete APIIS structure!PUBLIC METHODS}}
\paragraph*{\$apiis-$>$[ os\_user $|$ APIIS\_HOME $|$ version $|$ programname $|$ date\_format $|$ entry\_views $|$ reserved\_strings $|$ codes\_table $|$ browser $|$ fileselector $|$ use\_filelog $|$ filelog\_filename $|$use\_syslog $|$ syslog\_facility $|$ use\_sql\_logging $|$ sql\_logfile $|$ sql\_log\_dml\_only $|$ node\_name $|$ node\_ip $|$ sequence\_interval $|$ multilanguage ] (all public)\label{Apiis::Init_--_Basic_initialisation_object_for_the_complete_APIIS_structure__apiis-_os_user_APIIS_HOME_version_programname_date_format_entry_views_reserved_strings_codes_table_browser_fileselector_use_filelog_filelog_filename_use_syslog_syslog_facility_use_sql_logging_sql_logfile_sql_log_dml_only_node_name_node_ip_sequence_interval_multilanguage_all_public_}\index{Apiis::Init -- Basic initialisation object for the complete APIIS structure!\$apiis-$>$[ os\ user $|$ APIIS\ HOME $|$ version $|$ programname $|$ date\ format $|$ entry\ views $|$ reserved\ strings $|$ codes\ table $|$ browser $|$ fileselector $|$ use\ filelog $|$ filelog\ filename $|$use\ syslog $|$ syslog\ facility $|$ use\ sql\ logging $|$ sql\ logfile $|$ sql\ log\ dml\ only $|$ node\ name $|$ node\ ip $|$ sequence\ interval $|$ multilanguage ] (all public)}}


These public methods provide an interface for the user to access the
internal structure.



They are readonly and usually return a scalar value except entry\_views and
reserved\_strings.



\$apiis-$>$entry\_views returns a hash reference with the table names as keys
and the according entry views (which only contain active records of this
table) as values:

\begin{verbatim}
   codes => entry_codes
   unit => entry_unit
   transfer => entry_transfer
\end{verbatim}


\$apiis-$>$reserved\_strings returns a hash reference to the names and values
of the reserved strings for data entry:

\begin{verbatim}
   v_concat => ' >=< '
\end{verbatim}


(One problem here could be the intended blanks as part of the delimiter.
 Maybe they get lost by reading the config file with Config::IniFiles.)

\paragraph*{\$apiis-$>$[ language ] (public)\label{Apiis::Init_--_Basic_initialisation_object_for_the_complete_APIIS_structure__apiis-_language_public_}\index{Apiis::Init -- Basic initialisation object for the complete APIIS structure!\$apiis-$>$[ language ] (public)}}


\textbf{language} is a public read/write method. Initially it's populated by the
apiisrc configuration files, but it can be changed during program
execution. When you set a new language, the old one is returned:

\begin{verbatim}
   my $oldlang = $apiis->language( <newlang> );
\end{verbatim}
\paragraph*{\$apiis-$>$[ date\_order $|$ time\_order $|$ extdate2iso $|$ iso2extdate $|$ exttime2iso $|$ iso2exttime $|$ date\_parts $|$ time\_parts $|$ isodate $|$ isotime $|$ date\_sep $|$ time\_sep $|$ date\_conf\_err $|$ time\_conf\_err ] (public)\label{Apiis::Init_--_Basic_initialisation_object_for_the_complete_APIIS_structure__apiis-_date_order_time_order_extdate2iso_iso2extdate_exttime2iso_iso2exttime_date_parts_time_parts_isodate_isotime_date_sep_time_sep_date_conf_err_time_conf_err_public_}\index{Apiis::Init -- Basic initialisation object for the complete APIIS structure!\$apiis-$>$[ date\ order $|$ time\ order $|$ extdate2iso $|$ iso2extdate $|$ exttime2iso $|$ iso2exttime $|$ date\ parts $|$ time\ parts $|$ isodate $|$ isotime $|$ date\ sep $|$ time\ sep $|$ date\ conf\ err $|$ time\ conf\ err ] (public)}}


The Apiis default format for date and time accords to the widely accepted
ISO 8601 standard. Have a look at

\begin{verbatim}
   http://www.cl.cam.ac.uk/~mgk25/iso-time.html
\end{verbatim}


for a good summary or other resources for detailed descriptions. You are
strongly encouraged, to also use ISO 8601 date formats in your software.



\textbf{date\_order} returns the initially in apiisrc defined order of the date as
a scalar string. You can set the date format during program execution (e.g.
when you batch process several data streams) in the following syntax:

\begin{verbatim}
   my $oldformat = $apiis->date_order(
      order => 'DD.MM.YYYY',
      sep   => '.',
   );
\end{verbatim}


The two required parameters are the order of the parts and the separator.
The string to define the order has the following limitations:

\begin{itemize}

\item only the separator and the capital letters Y, M, and D are allowed.
\item the year has to be specified in the 4 digit form YYYY to avoid
     ambiguity.
\item the day (DD) and month (MM) formats have 2 digits each.
\item a valid order string with separators therefore must have the length of
     10 characters.
\item a valid order string without separator must have the length of 8
     characters.
\item for year, month, and day values only digits are allowed.\end{itemize}


Example without separator:

\begin{verbatim}
   my $oldformat = $apiis->date_order(
      order => 'YYYYMMDD',
      sep   => '',
   );
\end{verbatim}


If you want to set \textbf{date\_order} to new values, it returns a reference to
the hash of the previously configured parameters order and sep. You thus
can reset the old date format with:

\begin{verbatim}
   $apiis->date_order( %$oldformat );
\end{verbatim}


If the chosen date order accords to ISO 8601 (YYYY-MM-DD) the status flag
\$apiis-$>$isodate() is set to 1, otherwise its 0. The same applies to the
time order (hh:mm:ss) and isotime().



Another flag \textbf{date\_conf\_err()} is internally used to mark a bad date
format configuration and as a result of it skip all date tests.



If you really have to parse dates on your own you can get the separators
(besides the format string with \textbf{date\_order}) by invoking:

\begin{verbatim}
   my $d_sep = $apiis->date_sep();
   my $t_sep = $apiis->time_sep();
\end{verbatim}


\textbf{date\_parts()} is a readonly public methods that returns an array (or an
arrayref, depending on the invoking context) of the configured parts of the
date format in the correct order (e.g. ["YYYY", "MM", "DD"]).



This method is mainly usefull in internal date calculations.



\textbf{extdate2iso} converts your external date format into the internal
ISO 8601 format.  It additionally checks, if the passed date is valid.



In scalar context, a formatted date string is returned. In list context,
you get the date parts in the shown order:

\begin{verbatim}
   Example:
   $apiis->date_order( order => 'DD.MM.YYYY', sep => '.' );
\end{verbatim}
\begin{verbatim}
   # scalar context;
   my $ext_date = '11.2.2005 13:37:00';
   print $apiis->extdate2iso($ext_date), "\n";
   # prints: 2005-02-11 13:37:00
\end{verbatim}
\begin{verbatim}
   # list context:
   my ( $year, $month, $day, $hour, $minute, $second )
      = $apiis->extdate2iso($ext_date);
\end{verbatim}


The same return schema for scalar and list context applies to
\textbf{exttime2iso}, \textbf{iso2extdate}, and \textbf{iso2exttime}.



Note, that also the \textbf{iso2extdate} and \textbf{iso2exttime} methods keep this order in
list context. It does not make sense to make them return in the configured
external order as the list context is useful for programming purposes and a
changing order would force you to parse the configuration. And this is not,
what you want.

\paragraph*{substitute\_env (internal)\label{Apiis::Init_--_Basic_initialisation_object_for_the_complete_APIIS_structure_substitute_env_internal_}\index{Apiis::Init -- Basic initialisation object for the complete APIIS structure!substitute\ env (internal)}}


Does some postprocessing for special cases (substitution of APIIS\_HOME and
APIIS\_LOCAL with their values).



The value to check for substituting is passed as a reference so that
substituting is done in place:

\begin{verbatim}
   $self->substitute_env( \$val_to_substitute );
\end{verbatim}


It doesn't matter if there is a dollar sign \$ in front of APIIS\_HOME and
APIIS\_LOCAL or not.

\paragraph*{\_join\_user (internal)\label{Apiis::Init_--_Basic_initialisation_object_for_the_complete_APIIS_structure__join_user_internal_}\index{Apiis::Init -- Basic initialisation object for the complete APIIS structure!\ join\ user (internal)}}


\textbf{\$apiis-}\_join\_user$>$ takes a hashref with a User object (required) and
verifies this user against the database. If it's a valid user, his data gets
mounted into the \$apiis structure as the User object.



Example:
   \$apiis-$>$\_join\_user( \{ userobj =$>$ \$user\_obj \} );

\paragraph*{exists\_user (public)\label{Apiis::Init_--_Basic_initialisation_object_for_the_complete_APIIS_structure_exists_user_public_}\index{Apiis::Init -- Basic initialisation object for the complete APIIS structure!exists\ user (public)}}


\$apiis-$>$exists\_user returns 1 if the User object is already mounted into the
\$apiis structure, 0 otherwise.

\paragraph*{use\_filelog/use\_syslog/use\_sql\_logging (public)\label{Apiis::Init_--_Basic_initialisation_object_for_the_complete_APIIS_structure_use_filelog_use_syslog_use_sql_logging_public_}\index{Apiis::Init -- Basic initialisation object for the complete APIIS structure!use\ filelog/use\ syslog/use\ sql\ logging (public)}}


These methods mainly reflect the settings in apiisrc. They are read/write to
enable changing these values in rare cases, e.g. when running check\_integrity,
where logging make only little sense.

\paragraph*{syslog\_priority/filelog\_priority (public)\label{Apiis::Init_--_Basic_initialisation_object_for_the_complete_APIIS_structure_syslog_priority_filelog_priority_public_}\index{Apiis::Init -- Basic initialisation object for the complete APIIS structure!syslog\ priority/filelog\ priority (public)}}


syslog\_priority is read/write although it mostly won't be overwritten. But
in some cases you may want to switch the logging level for a certain part
of the code to e.g. 'debug', while other parts stay at e.g. 'warn'.
The allowed priorities are debug, info, notice, warn, warning, error, err, crit,
alert, emerg, panic in this order (err = error, warn = warning, emerg =
panic).



If syslog\_priority is set with

\begin{verbatim}
   my $oldvalue = $apiis->syslog_priority('debug');
\end{verbatim}


it returns the old value of syslog\_priority. You then can reset it with

\begin{verbatim}
   $apiis->syslog_priority( $oldvalue );
\end{verbatim}


Otherwise it returns the current value of syslog\_priority.



The same applies to \textbf{filelog\_priority}.

\paragraph*{log\_priority (public)\label{Apiis::Init_--_Basic_initialisation_object_for_the_complete_APIIS_structure_log_priority_public_}\index{Apiis::Init -- Basic initialisation object for the complete APIIS structure!log\ priority (public)}}


\textbf{log\_priority} is write-only and sets the values of syslog\_priority and
filelog\_priority to the same value which is passed as the argument. This is
mainly a development help as you don't know if the configuration is just set
to syslog or filelog.

\paragraph*{debug (public)\label{Apiis::Init_--_Basic_initialisation_object_for_the_complete_APIIS_structure_debug_public_}\index{Apiis::Init -- Basic initialisation object for the complete APIIS structure!debug (public)}}


\textbf{debug} returns 1 if the debug level is set, 0 otherwise.
Any true input value sets \$self-$>$debug to 1, any false value to 0.



\textbf{debug} can be used to query or set a debug flag, which can be used to
prevent the expensive invokation of \$apiis-$>$log on debug level. This flag
depends on the settings of filelog\_priority and syslog\_priority. If either of
them is set to 'debug', \$self-$>$debug always returns 1, even if you pass 0
to it. If you set \$self-$>$debug(1), filelog\_priority and syslog\_priority keep
their values;

\paragraph*{log (public)\label{Apiis::Init_--_Basic_initialisation_object_for_the_complete_APIIS_structure_log_public_}\index{Apiis::Init -- Basic initialisation object for the complete APIIS structure!log (public)}}
\begin{verbatim}
   $apiis->log('warn', "Cannot open file: $!");
\end{verbatim}


or

\begin{verbatim}
   $apiis->log('warn', 'Cannot open file: %s', $!);
\end{verbatim}


log() is the interface to the syslog utility. It takes as first input
parameter the syslog priority, at which it shall be printed into the system
log files (debug info notice warn warning error err crit alert emerg
panic). All levels below \$apiis-$>$syslog\_priority are suppressed, all of
\$apiis-$>$syslog\_priority and above are sent to syslog.



As an addition it can also log the sql statements into a file for basic
database recovery. It the passed priority is of type 'sql' like in

\begin{verbatim}
   $apiis->log('sql', $sqltext);
\end{verbatim}


and use\_sql\_logging is set to a true value in apiisrc, the sqltext will get
logged into the configured sql\_logfile together with a timestamp,
dabasename, and username (in a separate line with a sql comment). After a
defined backup state you simply have to run this file through your favorite
frontend to the database to recover the current state. If sql\_log\_dml\_only
is true in apiisrc, select statements are not logged.
Messages of priority 'sql' are not passed to syslog.

\paragraph*{status (public)\label{Apiis::Init_--_Basic_initialisation_object_for_the_complete_APIIS_structure_status_public_}\index{Apiis::Init -- Basic initialisation object for the complete APIIS structure!status (public)}}


\$apiis-$>$status returns a general status which is accessible everywhere and
at any time during execution. A status of 0 means success, all true values
indicate an error.



If you pass a parameter this will set the status to this value.

\paragraph*{running\_check\_integrity (public)\label{Apiis::Init_--_Basic_initialisation_object_for_the_complete_APIIS_structure_running_check_integrity_public_}\index{Apiis::Init -- Basic initialisation object for the complete APIIS structure!running\ check\ integrity (public)}}


\$apiis-$>$running\_check\_integrity is a simple switch that has to be set in
the program check\_integrity. Some checks on record level have different
behaviour (less checks) if they are invoked by check\_integrity.

\paragraph*{check\_status (public)\label{Apiis::Init_--_Basic_initialisation_object_for_the_complete_APIIS_structure_check_status_public_}\index{Apiis::Init -- Basic initialisation object for the complete APIIS structure!check\ status (public)}}


Checks \$apiis-$>$status and prints errors (if any). Optionally dies above a
certain severity level and ignores errors below a certain security level.



Input parameter can be a hash with the keys:

\begin{itemize}

\item \textbf{die} -- you can pass a level of severity to let the program die
        at this point and all levels above (in severity).
\item \textbf{ignore} -- below this level of severity the error messages are ignored\end{itemize}


\textbf{check\_status} returns the boolean value of the status stored in
\$obj-$>$status().



Example:

\begin{verbatim}
   $apiis->check_status(
       die => 'CRIT',
       ignore => 'INFO',
   );
\end{verbatim}
\paragraph*{errors (public)\label{Apiis::Init_--_Basic_initialisation_object_for_the_complete_APIIS_structure_errors_public_}\index{Apiis::Init -- Basic initialisation object for the complete APIIS structure!errors (public)}}


\$apiis-$>$errors returns the stored errors as an array of objects or an array
reference, just as requested by the caller.
If new errors are stored, errors() returns the error id(s). If you store
one error object, the error id of this error is returned as a scalar. If
you store an array of error objects, an array or arrayref of the error ids
of these error objects is returned in the order of the error objects.



Examples:
   my \$err\_id      = \$apiis-$>$errors(\$error\_object);
   my @err\_ids     = \$apiis-$>$errors(@error\_objects);
   my \$err\_ids\_ref = \$apiis-$>$errors(@error\_objects);

\paragraph*{error (public)\label{Apiis::Init_--_Basic_initialisation_object_for_the_complete_APIIS_structure_error_public_}\index{Apiis::Init -- Basic initialisation object for the complete APIIS structure!error (public)}}


\$apiis-$>$error takes as parameter an error id and returns the error object
for this id. This enables you to write code like this:

\begin{verbatim}
   $apiis->error(3)->print;
   $apiis->error(4)->severity('CRIT');
\end{verbatim}


If you pass an invalid error id, an error object is created and passed back
to the caller.

\paragraph*{del\_errors (public)\label{Apiis::Init_--_Basic_initialisation_object_for_the_complete_APIIS_structure_del_errors_public_}\index{Apiis::Init -- Basic initialisation object for the complete APIIS structure!del\ errors (public)}}


\$apiis-$>$del\_errors deletes all error objects.

\paragraph*{del\_error (public)\label{Apiis::Init_--_Basic_initialisation_object_for_the_complete_APIIS_structure_del_error_public_}\index{Apiis::Init -- Basic initialisation object for the complete APIIS structure!del\ error (public)}}


\$apiis-$>$del\_error takes as parameter an error id and deletes this error object
from the \$apiis-$>$errors array. Example:

\begin{verbatim}
   $apiis->del_error(3);
\end{verbatim}


If you pass an invalid error id, an error object is created, added to
\$apiis-$>$errors and additionally passed back to the caller.

\paragraph*{localtime (public)\label{Apiis::Init_--_Basic_initialisation_object_for_the_complete_APIIS_structure_localtime_public_}\index{Apiis::Init -- Basic initialisation object for the complete APIIS structure!localtime (public)}}


\$apiis-$>$localtime provides you with an unformatted timestamp. Usually this
is not used. The preferred methods are \$apiis-$>$today and \$apiis-$>$now as
they convert the date/time to the localized format.



\$apiis-$>$localtime returns a list of parameters. Example:

\begin{verbatim}
   my ($year, $mon, $mday, $hour, $min, $sec)
      = $apiis->localtime;
\end{verbatim}
\paragraph*{today (public)\label{Apiis::Init_--_Basic_initialisation_object_for_the_complete_APIIS_structure_today_public_}\index{Apiis::Init -- Basic initialisation object for the complete APIIS structure!today (public)}}


\$apiis-$>$today returns a formatted string of the current day.

\paragraph*{now (public)\label{Apiis::Init_--_Basic_initialisation_object_for_the_complete_APIIS_structure_now_public_}\index{Apiis::Init -- Basic initialisation object for the complete APIIS structure!now (public)}}


\$apiis-$>$now returns a formatted string of the current day and time.
For internal use it accepts an input parameter

\begin{verbatim}
   $apiis->now( format => 'today' );
\end{verbatim}


to return only the day without time. This is the whole magic behind
\$apiis-$>$today. :\^{})

\paragraph*{join\_model (public)\label{Apiis::Init_--_Basic_initialisation_object_for_the_complete_APIIS_structure_join_model_public_}\index{Apiis::Init -- Basic initialisation object for the complete APIIS structure!join\ model (public)}}


\textbf{\$apiis-}join\_model("modelfile")$>$ mounts all informations of the model file
into the core apiis structure and provides methods to access them.



As required input you have to provide the key 'userobj'. The value must be a
valid User-object.



Example:

\begin{verbatim}
   $apiis->join_model('breedprg',
      userobj => $user_obj,
   );
\end{verbatim}


\textbf{join\_model} creates an Apiis::Model object and passes it to \_add\_obj.
With the key 'Model', the model object is passed as the first and
only element of an anon array reference.



Besides the model file name there is another (hash) parameter 'database' to
\textbf{join\_model}.



With 'database =$>$ 0', the model file will be joined into \$apiis without
connection to the database.  For later joining the database into \$apiis, use
the public method \textbf{\$apiis-}join\_database$>$.



Using \textbf{join\_model} without connecting to the database will be
used in quite rare cases. One usefull operation will be when you want to
drop the complete database during basic initialisation.
In this case you have to provide some dummy User object like:

\begin{verbatim}
   require Apiis::DataBase::User;
   my $dummy = Apiis::DataBase::User->new(
       id       => ($apiis->os_user || 'nobody'),
       password => 'nopassword',
   );
\end{verbatim}
\begin{verbatim}
   $apiis->join_model('breedprg',
      userobj => $dummy,
      database => 0,
   );
\end{verbatim}
\paragraph*{exists\_model (public)\label{Apiis::Init_--_Basic_initialisation_object_for_the_complete_APIIS_structure_exists_model_public_}\index{Apiis::Init -- Basic initialisation object for the complete APIIS structure!exists\ model (public)}}


\$apiis-$>$exists\_model returns 1 if the model file is already mounted into the
\$apiis structure, 0 otherwise.

\paragraph*{\_join\_database (internal)\label{Apiis::Init_--_Basic_initialisation_object_for_the_complete_APIIS_structure__join_database_internal_}\index{Apiis::Init -- Basic initialisation object for the complete APIIS structure!\ join\ database (internal)}}


\$apiis-$>$\_join\_database initializes the database access.



It adds the newly created Apiis::DataBase::Init object into the existing
\$apiis-tree with the key 'DataBase':

\paragraph*{join\_database (public)\label{Apiis::Init_--_Basic_initialisation_object_for_the_complete_APIIS_structure_join_database_public_}\index{Apiis::Init -- Basic initialisation object for the complete APIIS structure!join\ database (public)}}


\$apiis-$>$join\_database is simply a public wrapper for \_join\_database.



The public method join\_database is usually not needed as join\_model()
automatically joins the database into \$apiis. For some rare cases (e.g.
initial creation of database), you can join\_model() without connection to
the database by passing the parameter 'database =$>$ 0'.



So
   \$apiis-$>$join\_model('breedprg', database =$>$ 0);
   \$apiis-$>$join\_database;



is equivalent to
   \$apiis-$>$join\_model('breedprg');

\paragraph*{exists\_database (public)\label{Apiis::Init_--_Basic_initialisation_object_for_the_complete_APIIS_structure_exists_database_public_}\index{Apiis::Init -- Basic initialisation object for the complete APIIS structure!exists\ database (public)}}


\$apiis-$>$exists\_database returns 1 if the database initialisation is already
done, 0 otherwise. The existance of the database object does not
necessarily include the database connection. If you invoke join\_model with
the parameter 'database =$>$ 0' the database object is created without
connecting to the database. This is needed for special cases like mksql,
where you need the configuration data like the db-specific datatype for the
metatypes like TIMESTAMP to create the database.

\paragraph*{exists\_auth (public)\label{Apiis::Init_--_Basic_initialisation_object_for_the_complete_APIIS_structure_exists_auth_public_}\index{Apiis::Init -- Basic initialisation object for the complete APIIS structure!exists\ auth (public)}}


\textbf{exists\_auth} is a boolean switch to show, if the Auth object for
authentication/authorisation is joined into the global \$apiis structure. It is
0/undef, if no Auth object/method exists, 1 otherwise.

\paragraph*{get\_db\_conf (mainly internal)\label{Apiis::Init_--_Basic_initialisation_object_for_the_complete_APIIS_structure_get_db_conf_mainly_internal_}\index{Apiis::Init -- Basic initialisation object for the complete APIIS structure!get\ db\ conf (mainly internal)}}


Read the config file for the passed Database from
\$APIIS\_HOME/etc/apiis/$<$Database$>$.conf and return a hash reference of this
structure.

\paragraph*{AUTOLOAD (internal)\label{Apiis::Init_--_Basic_initialisation_object_for_the_complete_APIIS_structure_AUTOLOAD_internal_}\index{Apiis::Init -- Basic initialisation object for the complete APIIS structure!AUTOLOAD (internal)}}


\textbf{AUTOLOAD()} catches all invocations of methods, that don't exist. On this
level it makes mainly sense for the structural elements Cache, Model,
DataBase, User, etc. It's difficult to catch them otherwise in
expressions like \$apiis-$>$Model-$>$tables, when join\_model has failed before
and therefore no method Model() exists. This case usually produces Error
objects, but every developer is free to ignore them.



Currently some more or less useful error messages are generated, printed
to STDOUT and the process dies. This is not optimal for processes that run
in a grapical environment (Tk, Html) and don't have access to a terminal.
But does it make sense to create an Error object if the developer tends to
ignore them?



Additionally, the produced error message is stored in the logfile/syslog,
if configured.

\subsection{Apiis::Init::Config mainly ready apiisrc config files\label{Apiis::Init::Config_mainly_ready_apiisrc_config_files}\index{Apiis::Init::Config mainly ready apiisrc config files}}




\subsubsection*{DESCRIPTION\label{Apiis::Init::Config_mainly_ready_apiisrc_config_files_DESCRIPTION}\index{Apiis::Init::Config mainly ready apiisrc config files!DESCRIPTION}}


Apiis::Init::Config contains internal methods to read the different apiisrc
files.

\subsubsection*{METHODS\label{Apiis::Init::Config_mainly_ready_apiisrc_config_files_METHODS}\index{Apiis::Init::Config mainly ready apiisrc config files!METHODS}}
\paragraph*{\_import\_apiisrc (internal)\label{Apiis::Init::Config_mainly_ready_apiisrc_config_files__import_apiisrc_internal_}\index{Apiis::Init::Config mainly ready apiisrc config files!\ import\ apiisrc (internal)}}


Imports the default apiis config file \$APIIS\_HOME/etc/apiisrc.

\paragraph*{\_import\_apiisrc\_local (internal)\label{Apiis::Init::Config_mainly_ready_apiisrc_config_files__import_apiisrc_local_internal_}\index{Apiis::Init::Config mainly ready apiisrc config files!\ import\ apiisrc\ local (internal)}}


Overwrites the defaults from apiis apiisrc config file with the
project specific one.

\paragraph*{\_import\_user\_apiisrc (internal)\label{Apiis::Init::Config_mainly_ready_apiisrc_config_files__import_user_apiisrc_internal_}\index{Apiis::Init::Config mainly ready apiisrc config files!\ import\ user\ apiisrc (internal)}}


Overwrites the project definition from global apiisrc config file.
This can be used for developers on a multiuser server to point to their
private copy of the project tree.



It reads only the [PROJECTS] section of apiisrc.

\paragraph*{\_xml2model (internal)\label{Apiis::Init::Config_mainly_ready_apiisrc_config_files__xml2model_internal_}\index{Apiis::Init::Config mainly ready apiisrc config files!\ xml2model (internal)}}


\textbf{\_xml2model} parses the passed xmlfile and returns a reference to a
datastructure, representing the model file.



usage:

\begin{verbatim}
    eval { $href = $self->Apiis::Init::Config::_xml2model(
                       xmlfile => $filename
                   );
    };
\end{verbatim}


This results in a structure like this:

\begin{verbatim}
    $href->{
       general => {...},
       table   => {
          <tablename> => {
             struct_type => '...',
             trigger  => {...},
             sequence => [...],
             index    => [...],
             pk       => {...},
             column   => {
                <columnname> => {...},
                <columnname> => {...},
             },
             _column_order => [...],
          },
       },
       _table_order => [...],
    };
\end{verbatim}


Only for internal use.



Note: This method will disappear in the near future when the model file
structure and parsing is rewritten.

\paragraph*{\_get\_db\_conf (internal)\label{Apiis::Init::Config_mainly_ready_apiisrc_config_files__get_db_conf_internal_}\index{Apiis::Init::Config mainly ready apiisrc config files!\ get\ db\ conf (internal)}}


Read the config file for the passed Database from
\$APIIS\_HOME/etc/apiis/$<$Database$>$.conf and return a hash reference of this
structure.

\subsubsection*{\$apiis-$>$\_check\_date\_conf (internal)\label{_apiis-_check_date_conf_internal_}\index{\$apiis-$>$\ check\ date\ conf (internal)}}


Internal routine to run some checks on the configured date format.
Sets the \_date\_parts array in case of success.

\subsubsection*{\$apiis-$>$\_check\_time\_conf (internal)\label{_apiis-_check_time_conf_internal_}\index{\$apiis-$>$\ check\ time\ conf (internal)}}


Internal routine to run some checks on the configured time format.
Sets the \_time\_parts array in case of success.

\subsection{SYNOPSIS\label{SYNOPSIS}\index{SYNOPSIS}}
\begin{verbatim}
   $xml_obj = Apiis::Init::XML->new(%args);
   $xml_obj = Apiis::Init::XML->new(
         dtd=>$dtd_file,
         xml=>$xml_file,
         gui=>$what_type_of_gui
   );
\end{verbatim}
\subsection{DESCRIPTION\label{DESCRIPTION}\index{DESCRIPTION}}


XML.pm init a file of configuration written in xml. XML.pm merge definitions
from the configuration file and the default values from the dtd-scheme.



Suppositions:



Each xml-element need a unique name over all configuration and subconfiguration
files, which will be defined in "Name". Access to each attribute take place with
a method in combination with the name of the element:

\begin{verbatim}
   xml:
    <PageHeader Name="PageHeader_10">
      <Lines Name="Line_1" Column="1-4" Row="1" LineType="solid"/>
    </PageHeader>             
    ---------
\end{verbatim}
\begin{verbatim}
   code: 
    $c=$xml_obj->Line_1->LineType
    $c is "solid"
\end{verbatim}
\begin{verbatim}
    $c=$xml_obj->Line_1->Name
    $c is "Line_1"
\end{verbatim}


Independent of the xml-definition a complete set of methods will be initiate
depend on the definition in the dtd-scheme. The default settings come from the
dtd-scheme and will overwritten if a the same attribute is defined in the
xml-scheme. E.g.

\begin{verbatim}
   dtd: 
   <!ATTLIST Text  
           Name       ID                          #REQUIRED
           Content    CDATA                       #REQUIRED
           Position   (static|absolute|relative)  "relative"
   >
\end{verbatim}
\begin{verbatim}
  xml:
  <PageHeader Name="PageHeader_10">
     <Text Position="relative" Name="Text_1" Content="test"/>
  </PageHeader>
\end{verbatim}
\begin{verbatim}
  code: 
   $c=$xml_obj->Text_1->Position
   $c is "relative"
\end{verbatim}


Each xml-file has a hierachical structure. XML makes a flat structure.

\subsection{METHODS\label{METHODS}\index{METHODS}}
\subsubsection*{\$apiis-$>$GUI-$>$[fullname $|$ basename $|$ ext $|$ path $|$ gui\_file ] (all public, readonly)\label{_apiis-_GUI-_fullname_basename_ext_path_gui_file_all_public_readonly_}\index{\$apiis-$>$GUI-$>$[fullname $|$ basename $|$ ext $|$ path $|$ gui\ file ] (all public, readonly)}}


fullname, basename, ext, path provide the fullname (basename.extension),
basename (without extension), extension, and path of the gui file.

\subsection{Apiis::Model -- methods to access the model file data via the \$apiis
structure\label{Apiis::Model_--_methods_to_access_the_model_file_data_via_the_apiis_structure}\index{Apiis::Model -- methods to access the model file data via the \$apiis
structure}}




\subsubsection*{SYNOPSIS\label{Apiis::Model_--_methods_to_access_the_model_file_data_via_the_apiis_structure_SYNOPSIS}\index{Apiis::Model -- methods to access the model file data via the apiis structure!SYNOPSIS}}
\begin{verbatim}
   $apiis->join_model('breedprg');
\end{verbatim}


The configuration data of the model file is mounted into the \$apiis
structure simply by running the join\_model method with the model file name
as the only parameter.

\subsubsection*{DESCRIPTION\label{Apiis::Model_--_methods_to_access_the_model_file_data_via_the_apiis_structure_DESCRIPTION}\index{Apiis::Model -- methods to access the model file data via the apiis structure!DESCRIPTION}}


This Model.pm module provides an object and the appropriate access methods.
With join\_model they are passed to Apiis::Init.pm and there with \_add\_obj
added to the global structure

\subsubsection*{METHODS\label{Apiis::Model_--_methods_to_access_the_model_file_data_via_the_apiis_structure_METHODS}\index{Apiis::Model -- methods to access the model file data via the apiis structure!METHODS}}
\paragraph*{new (mostly internal)\label{Apiis::Model_--_methods_to_access_the_model_file_data_via_the_apiis_structure_new_mostly_internal_}\index{Apiis::Model -- methods to access the model file data via the apiis structure!new (mostly internal)}}


Apiis::Model-$>$new is mainly invoked by Apiis::Init. The user interface is
join\_model.

\paragraph*{\_init (internal)\label{Apiis::Model_--_methods_to_access_the_model_file_data_via_the_apiis_structure__init_internal_}\index{Apiis::Model -- methods to access the model file data via the apiis structure!\ init (internal)}}


\_init does the main initialization and creates the internal structure to
keep the model file values.

\paragraph*{\$apiis-$>$Model-$>$[fullname $|$ basename $|$ ext $|$ path $|$ db\_driver $|$ db\_name
$|$ db\_host $|$ db\_port $|$ db\_user $|$ db\_password $|$ max\_check\_level] (all public, readonly)\label{Apiis::Model_--_methods_to_access_the_model_file_data_via_the_apiis_structure__apiis-_Model-_fullname_basename_ext_path_db_driver_db_name_db_host_db_port_db_user_db_password_max_check_level_all_public_readonly_}\index{Apiis::Model -- methods to access the model file data via the apiis structure!\$apiis-$>$Model-$>$[fullname $|$ basename $|$ ext $|$ path $|$ db\ driver $|$ db\ name
$|$ db\ host $|$ db\ port $|$ db\ user $|$ db\ password $|$ max\ check\ level] (all public, readonly)}}


fullname, basename, ext, path provide the fullname (basename.extension),
basename (without extension), extension, and path of the model file.



The db\_... methods reflect the database configurations at the top of the
model file.



max\_check\_level gives you the maximal configured checklevel of this model
file, if anybody really needs it.

\paragraph*{tables (public, readonly)\label{Apiis::Model_--_methods_to_access_the_model_file_data_via_the_apiis_structure_tables_public_readonly_}\index{Apiis::Model -- methods to access the model file data via the apiis structure!tables (public, readonly)}}


\$apiis-Model-$>$tables returns the names of the defined tables. If you want
an array, it gives you an array of these tables. If you want a scalar, you
also get what you want, a reference to the same array.

\paragraph*{table (public, readonly)\label{Apiis::Model_--_methods_to_access_the_model_file_data_via_the_apiis_structure_table_public_readonly_}\index{Apiis::Model -- methods to access the model file data via the apiis structure!table (public, readonly)}}
\begin{verbatim}
   $apiis->Model->table( $tablename );
\end{verbatim}


returns an object of Apiis::Model::TableObj for this tablename.

\paragraph*{check\_level (public, read/write)\label{Apiis::Model_--_methods_to_access_the_model_file_data_via_the_apiis_structure_check_level_public_read_write_}\index{Apiis::Model -- methods to access the model file data via the apiis structure!check\ level (public, read/write)}}
\begin{verbatim}
   my $current_level = $apiis->Model->check_level;
   my $old_level = $apiis->Model->check_level(2);
      ... do some work
   $apiis->Model->check_level( $old_level );
\end{verbatim}


Without an parameter check\_level returns the current check level. You can
change the current check level by passing the new level to check\_level,
which then returns the old check level.



check\_level also tests, if a passed new level is numeric and does not
exceed the maximum defined level in the model file.

\subsubsection*{Apiis::Model::TableObj -- internal package to provide a table object with
methods to access a single table and its columns\label{Apiis::Model_--_methods_to_access_the_model_file_data_via_the_apiis_structure_Apiis::Model::TableObj_--_internal_package_to_provide_a_table_object_with_methods_to_access_a_single_table_and_its_columns}\index{Apiis::Model -- methods to access the model file data via the apiis structure!Apiis::Model::TableObj -- internal package to provide a table object with
methods to access a single table and its columns}}




\paragraph*{SYNOPSIS\label{Apiis::Model::TableObj_--_internal_package_to_provide_a_table_object_with_methods_to_access_a_single_table_and_its_columns_SYNOPSIS}\index{Apiis::Model::TableObj -- internal package to provide a table object with methods to access a single table and its columns!SYNOPSIS}}


Programming interface:

\begin{verbatim}
   $table_obj = Apiis::Model::TableObj->new( $tablename, $struct_ref);
\end{verbatim}


Usage:

\begin{verbatim}
   $table_obj = $apiis->Model->table('animal');
\end{verbatim}
\paragraph*{METHODS\label{Apiis::Model::TableObj_--_internal_package_to_provide_a_table_object_with_methods_to_access_a_single_table_and_its_columns_METHODS}\index{Apiis::Model::TableObj -- internal package to provide a table object with methods to access a single table and its columns!METHODS}}
\subparagraph*{new (mostly internal)\label{Apiis::Model::TableObj_--_internal_package_to_provide_a_table_object_with_methods_to_access_a_single_table_and_its_columns_new_mostly_internal_}\index{Apiis::Model::TableObj -- internal package to provide a table object with methods to access a single table and its columns!new (mostly internal)}}


To create the table object, new() needs as input the table name and a
reference to the datastructure of this table from the model file:

\begin{verbatim}
   $table_obj = Apiis::Model::TableObj->new( $tablename, $struct_ref);
\end{verbatim}


The order of the columns in the model file is preserved.

\subparagraph*{column (public, readonly)\label{Apiis::Model::TableObj_--_internal_package_to_provide_a_table_object_with_methods_to_access_a_single_table_and_its_columns_column_public_readonly_}\index{Apiis::Model::TableObj -- internal package to provide a table object with methods to access a single table and its columns!column (public, readonly)}}


\$table\_obj-$>$column( \$col\_name ) returns the column object for this column

\subparagraph*{name (public, readonly)\label{Apiis::Model::TableObj_--_internal_package_to_provide_a_table_object_with_methods_to_access_a_single_table_and_its_columns_name_public_readonly_}\index{Apiis::Model::TableObj -- internal package to provide a table object with methods to access a single table and its columns!name (public, readonly)}}


\$table\_obj-$>$name returns the name of this table.

\subparagraph*{struct\_type (public, readonly)\label{Apiis::Model::TableObj_--_internal_package_to_provide_a_table_object_with_methods_to_access_a_single_table_and_its_columns_struct_type_public_readonly_}\index{Apiis::Model::TableObj -- internal package to provide a table object with methods to access a single table and its columns!struct\ type (public, readonly)}}


\$table\_obj-$>$struct\_type returns the structural type of this table.
Current values of struct\_type can be mandatory, recommended, and optional.

\subparagraph*{columns/cols (public, readonly)\label{Apiis::Model::TableObj_--_internal_package_to_provide_a_table_object_with_methods_to_access_a_single_table_and_its_columns_columns_cols_public_readonly_}\index{Apiis::Model::TableObj -- internal package to provide a table object with methods to access a single table and its columns!columns/cols (public, readonly)}}


\$table\_obj-$>$cols returns the columns of this table.
\$table\_obj-$>$columns is just an alias.

\subparagraph*{primarykey (public, readonly)\label{Apiis::Model::TableObj_--_internal_package_to_provide_a_table_object_with_methods_to_access_a_single_table_and_its_columns_primarykey_public_readonly_}\index{Apiis::Model::TableObj -- internal package to provide a table object with methods to access a single table and its columns!primarykey (public, readonly)}}


primarykey() needs one argument, which is either 'ref\_col', 'view',
'where', or 'concat'.

\begin{verbatim}
   $table_obj->primarykey('ref_col')
\end{verbatim}


returns the reference column to which this primary key in the table refers
to.

\begin{verbatim}
   $table_obj->primarykey('concat')
\end{verbatim}


returns the external columns, that build the concatenated primary key. The
old syntax of \$table\_obj-$>$primarykey('ext\_cols') is still supported but
deprecated.

\begin{verbatim}
   $table_obj->primarykey('view')
\end{verbatim}


returns the viewname of the view, that finally provides the foreignkey
through the where clause:

\begin{verbatim}
   $table_obj->primarykey('where')
\end{verbatim}


Often the where clause is 'closing\_dt is NULL'. The resulting view then
shows only records, which are not closed.

\subparagraph*{\$table\_obj-$>$[sequence $|$ sequences $|$ index $|$ indices $|$ indexes] (public,
readonly)\label{Apiis::Model::TableObj_--_internal_package_to_provide_a_table_object_with_methods_to_access_a_single_table_and_its_columns__table_obj-_sequence_sequences_index_indices_indexes_public_readonly_}\index{Apiis::Model::TableObj -- internal package to provide a table object with methods to access a single table and its columns!\$table\ obj-$>$[sequence $|$ sequences $|$ index $|$ indices $|$ indexes] (public,
readonly)}}


They return the index and the sequence entries for the table,
either as an array or as an array reference. There are only two
methods, the others act like aliases.



usage:

\begin{verbatim}
   my @indices = $table_obj->indices;
   my $sequences_ref = $table_obj->sequences;
\end{verbatim}
\subparagraph*{\$table\_obj-$>$triggers( \$triggertype ) (public, readonly)\label{Apiis::Model::TableObj_--_internal_package_to_provide_a_table_object_with_methods_to_access_a_single_table_and_its_columns__table_obj-_triggers_triggertype_public_readonly_}\index{Apiis::Model::TableObj -- internal package to provide a table object with methods to access a single table and its columns!\$table\ obj-$>$triggers( \$triggertype ) (public, readonly)}}


The method \textbf{triggers} takes the following triggertypes as argument:

\begin{verbatim}
   $table_obj->triggers( 'preinsert' );
   $table_obj->triggers( 'postinsert' );
   $table_obj->triggers( 'preupdate' );
   $table_obj->triggers( 'postupdate' );
   $table_obj->triggers( 'predelete' );
   $table_obj->triggers( 'postdelete' );
\end{verbatim}


and returns the triggers for this type. Depending on the calling context
they will be returned as a list or as an array reference.

\subparagraph*{\$table\_obj-$>$[datatype $|$ length $|$ default $|$ description $|$ check $|$ modify $|$ foreignkey $|$ label] (public, readonly)\label{Apiis::Model::TableObj_--_internal_package_to_provide_a_table_object_with_methods_to_access_a_single_table_and_its_columns__table_obj-_datatype_length_default_description_check_modify_foreignkey_label_public_readonly_}\index{Apiis::Model::TableObj -- internal package to provide a table object with methods to access a single table and its columns!\$table\ obj-$>$[datatype $|$ length $|$ default $|$ description $|$ check $|$ modify $|$ foreignkey $|$ label] (public, readonly)}}


Although these methods are column methods, they are kept here for
compatibility reasons.



The old, still valid (but deprecated) syntax

\begin{verbatim}
   my $descr = $table_obj->description( $column_name );
\end{verbatim}


should now be better written as:

\begin{verbatim}
   my $descr = $column_obj->description;
\end{verbatim}


or

\begin{verbatim}
   my $descr = $table_obj->column( $column_name )->description;
\end{verbatim}
\paragraph*{Apiis::Model::ColumnObj -- internal package to provide a column object with
methods to access a single column of a table\label{Apiis::Model::TableObj_--_internal_package_to_provide_a_table_object_with_methods_to_access_a_single_table_and_its_columns_Apiis::Model::ColumnObj_--_internal_package_to_provide_a_column_object_with_methods_to_access_a_single_column_of_a_table}\index{Apiis::Model::TableObj -- internal package to provide a table object with methods to access a single table and its columns!Apiis::Model::ColumnObj -- internal package to provide a column object with
methods to access a single column of a table}}




\subparagraph*{SYNOPSIS\label{Apiis::Model::ColumnObj_--_internal_package_to_provide_a_column_object_with_methods_to_access_a_single_column_of_a_table_SYNOPSIS}\index{Apiis::Model::ColumnObj -- internal package to provide a column object with methods to access a single column of a table!SYNOPSIS}}
\begin{verbatim}
   $col_obj = $table_obj->column( $column_name );
\end{verbatim}
\subparagraph*{DESCRIPTION\label{Apiis::Model::ColumnObj_--_internal_package_to_provide_a_column_object_with_methods_to_access_a_single_column_of_a_table_DESCRIPTION}\index{Apiis::Model::ColumnObj -- internal package to provide a column object with methods to access a single column of a table!DESCRIPTION}}
\subparagraph*{METHODS\label{Apiis::Model::ColumnObj_--_internal_package_to_provide_a_column_object_with_methods_to_access_a_single_column_of_a_table_METHODS}\index{Apiis::Model::ColumnObj -- internal package to provide a column object with methods to access a single column of a table!METHODS}}
\*{\$column\_obj-$>$[datatype $|$ length $|$ default $|$ description $|$ check $|$ modify $|$ struct\_type $|$ label] (public, readonly)\label{Apiis::Model::ColumnObj_--_internal_package_to_provide_a_column_object_with_methods_to_access_a_single_column_of_a_table__column_obj-_datatype_length_default_description_check_modify_struct_type_label_public_readonly_}\index{Apiis::Model::ColumnObj -- internal package to provide a column object with methods to access a single column of a table!\$column\ obj-$>$[datatype $|$ length $|$ default $|$ description $|$ check $|$ modify $|$ struct\ type $|$ label] (public, readonly)}}


Example:
   my \$datatype = \$column\_obj-$>$datatype;



The according values from the model file are returned. All these
methods are readonly.



check() returns the rules for the current check level. If a check level
for a column is defined/exists, this one is taken.



If there is no CHECKn defined for check level n the default CHECK is taken.
This also applies if e.g. CHECK2 is defined in the model file but no
CHECK1. In this case the default CHECK is taken for CHECK1 as this is
undef.

\*{foreignkey (public, readonly)\label{Apiis::Model::ColumnObj_--_internal_package_to_provide_a_column_object_with_methods_to_access_a_single_column_of_a_table_foreignkey_public_readonly_}\index{Apiis::Model::ColumnObj -- internal package to provide a column object with methods to access a single column of a table!foreignkey (public, readonly)}}
\begin{verbatim}
   my ($fk_table, $fk_column) = $column_obj->foreignkey;
\end{verbatim}


foreignkey() returns the defined foreign key table and the foreign key
column for this column, either as an array or as a
reference to an array, depending on the callers context.



It returns undef if no foreign key is defined.

\subsection{Apiis::Errors -- Provide error objects for generic error handling in APIIS\label{Apiis::Errors_--_Provide_error_objects_for_generic_error_handling_in_APIIS}\index{Apiis::Errors -- Provide error objects for generic error handling in APIIS}}




\subsubsection*{SYNOPSIS\label{Apiis::Errors_--_Provide_error_objects_for_generic_error_handling_in_APIIS_SYNOPSIS}\index{Apiis::Errors -- Provide error objects for generic error handling in APIIS!SYNOPSIS}}
\begin{verbatim}
   my $err_obj = Apiis::Errors->new(
      type      => 'CONFIG',
      severity  => 'INFO',
      from      => 'test.Errors',
      msg_short => "No date format defined",
   );
\end{verbatim}


Apiis::Errors-$>$new() creates an error object, that describes an error
comprehensively to enable further adequate processing.

\subsubsection*{DESCRIPTION\label{Apiis::Errors_--_Provide_error_objects_for_generic_error_handling_in_APIIS_DESCRIPTION}\index{Apiis::Errors -- Provide error objects for generic error handling in APIIS!DESCRIPTION}}


Apiis::Errors provides an error object with the following traits:

\begin{itemize}

\item Error \textbf{type}, currently:\begin{itemize}

\item \textbf{DATA}    the passed data is not ok (usually in CheckRules)
\item \textbf{DB}      errors from the database (e.g. unique index violation)
\item \textbf{OS}      errors from the operation system (e.g. full hard disk)
\item \textbf{AUTH}    errors concerning access rights
\item \textbf{PARSE}   errors in ParsePseudoSQL with parsing pseudo SQL code
\item \textbf{CODE}    programming errors, e.g. from applications like load objects
\item \textbf{PARAM}   passed parameter is wrong or missing
\item \textbf{CONFIG}  one of the configuration files is wrong or has missing entries
\item \textbf{INSTALL} there is an error in the Apiis/Perl installation
\item \textbf{UNKNOWN} is unknown.\end{itemize}

\item Error \textbf{severity}, currently \textbf{DEBUG INFO NOTICE WARNING ERR CRIT
ALERT EMERG}. These severity values are the same as the unix syslog
priorities. See also 'man syslog.conf' under Unix/Linux.\begin{itemize}

\item \textbf{DEBUG}   debugging messages for bug hunting
\item \textbf{INFO}    informational notice
\item \textbf{NOTICE}  more than information, somebody should notice it
\item \textbf{WARNING} influences further processing but is not so severe
\item \textbf{WARN}    deprecated, use WARNING
\item \textbf{ERR}     error, handled in the normal flow control
\item \textbf{ERROR}   deprecated, use ERR
\item \textbf{CRIT}    critical error, but can be handled under certain circumstances
\item \textbf{ALERT}   alarm, immediate intervention necessary
\item \textbf{EMERG}   no further processing possible (e.g. disk full)
\item \textbf{PANIC}   deprecated, use EMERG\end{itemize}

\item Error \textbf{action}, currently:\begin{itemize}

\item \textbf{INSERT}  the error occurred during a database insert
\item \textbf{UPDATE}  the error occurred during a database update
\item \textbf{DELETE}  the error occurred during a database delete
\item \textbf{SELECT}  the error occurred during a database select
\item \textbf{FETCH}   like SELECT
\item \textbf{DECODE}  the error occurred during an attempt to decode the data
\item \textbf{ENCODE}  the error occurred during an attempt to encode the data
\item \textbf{UNKNOWN} the action is unknown\end{itemize}
\end{itemize}


The internal structure provides the following fields to describe a
certain error:

\begin{verbatim}
 %struct = (
    type           => undef,    # predefined values above
    id             => undef,    # error id
    severity       => undef,    # predefined values above
    action         => undef,    # predefined values above
    from           => undef,    # location where this error comes from
                                # (e.g. sub, rule)
    record_id      => undef,    # id of this record, e.g. record_seq
                                # from inspool
    unit           => undef,    # unit that provides this data
    db_table       => undef,    # database table concerned
    db_column      => undef,    # database column concerned
    data           => undef,    # just handled incorrect data
    ext_fields     => undef,    # involved external fields (array)
    ext_fields_idx => undef,    # index of these external fields (for tabulars)
    ds             => undef,    # data stream name
    err_code       => undef,    # coded error message
    msg_short      => undef,    # main error message for end users
    msg_long       => undef,    # detailed error message
    misc1          => undef,    # user defined scalar
    misc2          => undef,    # user defined scalar
    misc_arr1      => undef,    # user defined array
    misc_arr2      => undef,    # user defined array
    backtrace      => undef,    # backtrace in Carp::longmess style
 );
\end{verbatim}


Public and internal methods are:

\subsubsection*{INTERNAL METHODS\label{Apiis::Errors_--_Provide_error_objects_for_generic_error_handling_in_APIIS_INTERNAL_METHODS}\index{Apiis::Errors -- Provide error objects for generic error handling in APIIS!INTERNAL METHODS}}
\paragraph*{new (mostly internal)\label{Apiis::Errors_--_Provide_error_objects_for_generic_error_handling_in_APIIS_new_mostly_internal_}\index{Apiis::Errors -- Provide error objects for generic error handling in APIIS!new (mostly internal)}}


new creates the object and checks access rights to the object structure.

\paragraph*{\$error\_obj-$>$[ type\_values $|$ severity\_values $|$ action\_values ] (all external)\label{Apiis::Errors_--_Provide_error_objects_for_generic_error_handling_in_APIIS__error_obj-_type_values_severity_values_action_values_all_external_}\index{Apiis::Errors -- Provide error objects for generic error handling in APIIS!\$error\ obj-$>$[ type\ values $|$ severity\ values $|$ action\ values ] (all external)}}


These public methods provide read only access to the preconfigured values.

\paragraph*{\$error\_obj-$>$[ from $|$ line $|$ backtrace $|$ record\_id $|$ unit
            $|$ db\_table $|$ db\_column $|$ data $|$ ext\_fields $|$ ext\_fields\_idx
            $|$ ds $|$ err\_code $|$ msg\_short $|$ msg\_long $|$ misc1 $|$ misc2
            $|$ misc\_arr1 $|$ misc\_arr2 ] (all external)\label{Apiis::Errors_--_Provide_error_objects_for_generic_error_handling_in_APIIS__error_obj-_from_line_backtrace_record_id_unit_db_table_db_column_data_ext_fields_ext_fields_idx_ds_err_code_msg_short_msg_long_misc1_misc2_misc_arr1_misc_arr2_all_external_}\index{Apiis::Errors -- Provide error objects for generic error handling in APIIS!\$error\ obj-$>$[ from $|$ line $|$ backtrace $|$ record\ id $|$ unit
            $|$ db\ table $|$ db\ column $|$ data $|$ ext\ fields $|$ ext\ fields\ idx
            $|$ ds $|$ err\ code $|$ msg\ short $|$ msg\ long $|$ misc1 $|$ misc2
            $|$ misc\ arr1 $|$ misc\ arr2 ] (all external)}}


These public methods provide read/write access to the structur elements.

\paragraph*{print (external)\label{Apiis::Errors_--_Provide_error_objects_for_generic_error_handling_in_APIIS_print_external_}\index{Apiis::Errors -- Provide error objects for generic error handling in APIIS!print (external)}}


Print the defined elements of this error object in the order of the
hash \%struct (actually the @struct array). This is mainly used for debugging.



Second input parameter can be a hash with the key:

\begin{itemize}

\item \textbf{filehandle} -- the output then goes to this filehandle instead of
        STDOUT (default)
        note: the filehandle has to be passed as a typeglob\end{itemize}


Example:

\begin{verbatim}
   $err_obj->print(
      filehandle => *ERR_FILE,
   );
\end{verbatim}
\paragraph*{sprint (external)\label{Apiis::Errors_--_Provide_error_objects_for_generic_error_handling_in_APIIS_sprint_external_}\index{Apiis::Errors -- Provide error objects for generic error handling in APIIS!sprint (external)}}


Return the formatted error message as a string (used by \textbf{print}).

\paragraph*{sprint\_html (external)\label{Apiis::Errors_--_Provide_error_objects_for_generic_error_handling_in_APIIS_sprint_html_external_}\index{Apiis::Errors -- Provide error objects for generic error handling in APIIS!sprint\ html (external)}}


Return the formatted error message as a string (used by \textbf{print}).

\paragraph*{syslog\_print (external)\label{Apiis::Errors_--_Provide_error_objects_for_generic_error_handling_in_APIIS_syslog_print_external_}\index{Apiis::Errors -- Provide error objects for generic error handling in APIIS!syslog\ print (external)}}


The error message is formatted for unix syslog (used by \$apiis-$>$log).

\subsection{Apiis::Misc -- Provides some usefull subroutines, mainly for compatibility reasons\label{Apiis::Misc_--_Provides_some_usefull_subroutines_mainly_for_compatibility_reasons}\index{Apiis::Misc -- Provides some usefull subroutines, mainly for compatibility reasons}}




\subsubsection*{SYNOPSIS\label{Apiis::Misc_--_Provides_some_usefull_subroutines_mainly_for_compatibility_reasons_SYNOPSIS}\index{Apiis::Misc -- Provides some usefull subroutines mainly for compatibility reasons!SYNOPSIS}}
\begin{verbatim}
   use Apiis::Misc qw( <subroutine_name> );
\end{verbatim}
\subsubsection*{DESCRIPTION\label{Apiis::Misc_--_Provides_some_usefull_subroutines_mainly_for_compatibility_reasons_DESCRIPTION}\index{Apiis::Misc -- Provides some usefull subroutines mainly for compatibility reasons!DESCRIPTION}}


Apiis::Misc gives you access to the subroutines (not object methods!):

\begin{verbatim}
   show_progress mychomp elapsed is_true
   Info Error
   LocalToRawDate RawToLocalDate Decode_Date_NativeRDBMS
   find_pod_path file2variable mimetype_of
\end{verbatim}


You can load some of them by writing:

\begin{verbatim}
   use Apiis::Misc qw( show_progress mychomp );
\end{verbatim}


They are also grouped:

\begin{verbatim}
   use Apiis::Misc qw( :Tk );    # exports the Tk routines Info and Error
   use Apiis::Misc qw( :date );  # exports the date routines
   use Apiis::Misc qw( :all );   # exports all routines
\end{verbatim}
\subsubsection*{Subroutines\label{Apiis::Misc_--_Provides_some_usefull_subroutines_mainly_for_compatibility_reasons_Subroutines}\index{Apiis::Misc -- Provides some usefull subroutines mainly for compatibility reasons!Subroutines}}
\paragraph*{show\_progress\label{Apiis::Misc_--_Provides_some_usefull_subroutines_mainly_for_compatibility_reasons_show_progress}\index{Apiis::Misc -- Provides some usefull subroutines mainly for compatibility reasons!show\ progress}}


\textbf{show\_progress} gives you some kind of progress view. It prints a dot every \$mod
times (default 100), every \$mod*10 times it prints the number.
   input:  1.) a reference to the counter (usually starting with 1).
           2.) optional: modulus operator \$mod
   return: none. The counter has to be incremented outside this routine!

\paragraph*{MaskForLatex\label{Apiis::Misc_--_Provides_some_usefull_subroutines_mainly_for_compatibility_reasons_MaskForLatex}\index{Apiis::Misc -- Provides some usefull subroutines mainly for compatibility reasons!MaskForLatex}}


Creates Escape-sequences for print in latex

\paragraph*{mychomp\label{Apiis::Misc_--_Provides_some_usefull_subroutines_mainly_for_compatibility_reasons_mychomp}\index{Apiis::Misc -- Provides some usefull subroutines mainly for compatibility reasons!mychomp}}


By: Chris Nandor (from the Perl Function Repository)
removes end-of-line regardless of originating platform
of file

\paragraph*{Info\label{Apiis::Misc_--_Provides_some_usefull_subroutines_mainly_for_compatibility_reasons_Info}\index{Apiis::Misc -- Provides some usefull subroutines mainly for compatibility reasons!Info}}


show an Tk-Info window

\begin{verbatim}
  usage: Info("Infomessage");
\end{verbatim}
\paragraph*{Error\label{Apiis::Misc_--_Provides_some_usefull_subroutines_mainly_for_compatibility_reasons_Error}\index{Apiis::Misc -- Provides some usefull subroutines mainly for compatibility reasons!Error}}


show an error window

\begin{verbatim}
  usage: Error("Errormessage");
\end{verbatim}
\paragraph*{LocalToRawDate\label{Apiis::Misc_--_Provides_some_usefull_subroutines_mainly_for_compatibility_reasons_LocalToRawDate}\index{Apiis::Misc -- Provides some usefull subroutines mainly for compatibility reasons!LocalToRawDate}}


Date conversion.



Standard SQL format seems to be 'DD-MON-YYYY', e.g. '24-MAY-2000'. At
least PostgreSQL and Oracle6 accept this.
As long as DBI/DBD does not convert different date formats to the standard
formats of the databases we have to provide this conversion in apiis.
Date::Calc has date formats EU (european format day-month-year) and US
(US american format month-day-year)
   input:  type of local dateformat [EU$|$US]
           local date
   return: old version (before Aug. 2001):
           date in native database format or -1 in case of errors
           LocalToRawDate('EU', '24.5.2000') will return '24-MAY-2000'.
           new version:
           list of ( "new\_date\_string", \$status, \$err\_msg )
           LocalToRawDate('EU', '24.5.2000') will return ('24-MAY-2000',0,undef).



Note: This is old stuff and remains here only for compatibility reasons. It
will be removed in the near future. (2005-02-28 heli)

\paragraph*{RawToLocalDate\label{Apiis::Misc_--_Provides_some_usefull_subroutines_mainly_for_compatibility_reasons_RawToLocalDate}\index{Apiis::Misc -- Provides some usefull subroutines mainly for compatibility reasons!RawToLocalDate}}


RawToLocalDate - change native database date format to EU/US-format (Date::Calc)
   input:  type of local dateformat [EU$|$US]
           native database date
   return: date in local format or -1 in case of errors



RawToLocalDate('EU', '24-MAY-2000') will return '24.5.2000'.



Note: This is old stuff and remains here only for compatibility reasons. It
will be removed in the near future. (2005-02-28 heli)

\paragraph*{Decode\_Date\_NativeRDBMS\label{Apiis::Misc_--_Provides_some_usefull_subroutines_mainly_for_compatibility_reasons_Decode_Date_NativeRDBMS}\index{Apiis::Misc -- Provides some usefull subroutines mainly for compatibility reasons!Decode\ Date\ NativeRDBMS}}


This sub is used like Decode\_Date\_EU and Decode\_Date\_US, where 'EU' and 'US'
is set in apiisrc. For check\_integrity only the native date values from the
database is used. It takes DATESEP and DATEORDER from the definitions in
Database.pm and parsed the passed date (which is in DATEORDER). It returns
(\$year, \$month, \$day) like the Decode\_Date\_ routines from Date::Calc.



Example: (\$year, \$month, \$day) = Decode\_Date\_NativeRDBMS('1999-5-13')



Note: This is old stuff and remains here only for compatibility reasons. It
will be removed in the near future. (2005-02-28 heli)

\paragraph*{elapsed\label{Apiis::Misc_--_Provides_some_usefull_subroutines_mainly_for_compatibility_reasons_elapsed}\index{Apiis::Misc -- Provides some usefull subroutines mainly for compatibility reasons!elapsed}}


do some profiling:
   input:  array reference with entries in order of Now()
   output: String "Hours:Minutes:Seconds" elapsed since passed start time

\paragraph*{is\_true\label{Apiis::Misc_--_Provides_some_usefull_subroutines_mainly_for_compatibility_reasons_is_true}\index{Apiis::Misc -- Provides some usefull subroutines mainly for compatibility reasons!is\ true}}


\textbf{is\_true} tests the passed scalar value, if it is true or false in the
boolean sense.



False are:

\begin{verbatim}
   * undef
   * 0       (zero, either as number or as string of length 1)
   * ''      (the empty string)
   * all other strings and numbers that are not true
\end{verbatim}


True are:

\begin{verbatim}
   * all numbers different from 0
   * the number 0E0 is zero, but true (Perl internal)
   * all strings, which are defined as representing true like
     'true', 'yes'. For different languages you can add the according string
     for 'yes' like 'ja' in german.
     Only the first character of the string in lowercase is checked.
\end{verbatim}
\begin{verbatim}
   input:  any scalar value that has to be checked for being true or not
\end{verbatim}
\begin{verbatim}
   return: 1 if the passed value is true, 0 or undef or the empty string
           otherwise
\end{verbatim}
\paragraph*{find\_pod\_path\label{Apiis::Misc_--_Provides_some_usefull_subroutines_mainly_for_compatibility_reasons_find_pod_path}\index{Apiis::Misc -- Provides some usefull subroutines mainly for compatibility reasons!find\ pod\ path}}


\textbf{find\_pod\_path} tries to find the appropriate Perl POD documentation for the
invoked programm. It first looks for language specific pod-files (taking
\$apiis-$>$language) and continues the search to the less specific versions. Last
ressort is the program itself:

\begin{verbatim}
    <programname>_<lang>.pod
    <programname>.<lang>.pod
    <programname>.pod
    <programname>
\end{verbatim}


The $<$programname$>$ contains the complete path, found by the Perl core module
FindBin.

\paragraph*{file2variable\label{Apiis::Misc_--_Provides_some_usefull_subroutines_mainly_for_compatibility_reasons_file2variable}\index{Apiis::Misc -- Provides some usefull subroutines mainly for compatibility reasons!file2variable}}


\textbf{file2variable} loads file from the local file system and tries to guess its
mime-type

\begin{verbatim}
   input:  file name
   return: scalar with the file content and scalar with the mime-type
\end{verbatim}
\paragraph*{mimetype\_of\label{Apiis::Misc_--_Provides_some_usefull_subroutines_mainly_for_compatibility_reasons_mimetype_of}\index{Apiis::Misc -- Provides some usefull subroutines mainly for compatibility reasons!mimetype\ of}}


\textbf{mimetype\_of} tries to find out the mimetype of a given filename. It first
uses MIME::Types, which seems to give the best results. Only if this module is
not installed or doesn't find any entry, a second test with File::Type is
done. \textbf{mimetype\_of} returns the found mimetype (e.g.
application/vnd.ms-powerpoint) or undef otherwise.

\begin{verbatim}
   Input:  filename with full path.
   Return: mimetype or undef
\end{verbatim}


Example:

\begin{verbatim}
   use Apiis::Misc qw( mimetype_of );
   my $mt = mimetype_of($file_name);
\end{verbatim}
\subsection{Apiis::CheckFile -- Find configuration files\label{Apiis::CheckFile_--_Find_configuration_files}\index{Apiis::CheckFile -- Find configuration files}}




\subsubsection*{SYNOPSIS\label{Apiis::CheckFile_--_Find_configuration_files_SYNOPSIS}\index{Apiis::CheckFile -- Find configuration files!SYNOPSIS}}
\begin{verbatim}
   my $a = CheckFile->new( file=>'apiis.model' );
\end{verbatim}


CheckFile gets a file name as argument and looks where it finds this file
under \$apiis\_local.

\subsubsection*{DESCRIPTION\label{Apiis::CheckFile_--_Find_configuration_files_DESCRIPTION}\index{Apiis::CheckFile -- Find configuration files!DESCRIPTION}}


CheckFile gets a file name as argument and looks where to find this file
under \$apiis\_local.
The file name argument can either be



a complete path (e.g.  \$apiis\_local/model/apiis.model)
or only the file (e.g. apiis.model)
or only the basename of the file (e.g. apiis)



Examples:

\begin{verbatim}
   my $a = CheckFile->new(file=>'apiis.model'); 
   my $a = CheckFile->new(file=>'apiis'); 
   my $a = CheckFile->new(file=>"$apiis_local/model/forms/abc");
   my $a = CheckFile->new(file=>"../../apiis/reports/jjj");
\end{verbatim}


The recognized filename extensions are defined in \$self-$>$\{\_suffixes\},
the locations for doing the search in \$self-$>$\{\_locations\}.

