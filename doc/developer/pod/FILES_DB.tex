\subsection{Apiis::DataBase::Init -- Basic database initialisation\label{Apiis::DataBase::Init_--_Basic_database_initialisation}\index{Apiis::DataBase::Init -- Basic database initialisation}}




\subsubsection*{SYNOPSIS\label{Apiis::DataBase::Init_--_Basic_database_initialisation_SYNOPSIS}\index{Apiis::DataBase::Init -- Basic database initialisation!SYNOPSIS}}


Database Initialisation based on the configuration in apiisrc and model file

\subsubsection*{DESCRIPTION\label{Apiis::DataBase::Init_--_Basic_database_initialisation_DESCRIPTION}\index{Apiis::DataBase::Init -- Basic database initialisation!DESCRIPTION}}


When loading the model file, database initialisation usually is done
automatically. For certain special cases it can be delayed.

\paragraph*{\$apiis-$>$DataBase-$>$disconnect()\label{Apiis::DataBase::Init_--_Basic_database_initialisation__apiis-_DataBase-_disconnect_}\index{Apiis::DataBase::Init -- Basic database initialisation!\$apiis-$>$DataBase-$>$disconnect()}}


\textbf{\$apiis-}DataBase-$>$disconnect()$>$ disconnects the global database handle
\$dbh (\$apiis-$>$DataBase-$>$dbh) from the database. If you pass a handle to
\textbf{disconnect()} like:

\begin{verbatim}
   $apiis->DataBase->disconnect( $my_db_handle );
\end{verbatim}


then this \$my\_db\_handle will be disconnected.

\paragraph*{commit(), user\_commit(), sys\_commit(), rollback(), user\_rollback(), sys\_rollback()\label{Apiis::DataBase::Init_--_Basic_database_initialisation_commit_user_commit_sys_commit_rollback_user_rollback_sys_rollback_}\index{Apiis::DataBase::Init -- Basic database initialisation!commit(), user\ commit(), sys\ commit(), rollback(), user\ rollback(), sys\ rollback()}}


These methods commit/rollback all transactions for either the system
or the user database handle.



While

\begin{verbatim}
   $apiis->DataBase->sys_commit;
   $apiis->DataBase->sys_rollback;
\end{verbatim}


and

\begin{verbatim}
   $apiis->DataBase->commit;
   $apiis->DataBase->rollback;
\end{verbatim}


commit/rollback changes to the system database handle,

\begin{verbatim}
   $apiis->DataBase->user_commit;
   $apiis->DataBase->user_rollback;
\end{verbatim}


perform this for the user database handle.



If the commit/rollback fails, an error object is created and an error status
of 1 is returned and the \$object-$>$status is set to 1.

\paragraph*{seq\_next\_val\label{Apiis::DataBase::Init_--_Basic_database_initialisation_seq_next_val}\index{Apiis::DataBase::Init -- Basic database initialisation!seq\ next\ val}}
\begin{verbatim}
   $apiis->DataBase->seq_next_val( <sequence_name> );
\end{verbatim}


returns the next value of the sequence $<$sequence\_name$>$.

\paragraph*{\_get\_users (internal)\label{Apiis::DataBase::Init_--_Basic_database_initialisation__get_users_internal_}\index{Apiis::DataBase::Init -- Basic database initialisation!\ get\ users (internal)}}


\textbf{\_get\_users} retrieves all configured users from table 'users' and fills
an internal datastructure to access the needed values.



\textbf{\_get\_users} is invoked by Apiis::Init::\_join\_database.



Note: As this served as some kind of caching and we have performance
problems during initialization (especially for the web interface), this
approach will be deactivated and replaced by queries for every single user
(based on the login data). (20.12.04 - heli)

\paragraph*{\_get\_user (internal)\label{Apiis::DataBase::Init_--_Basic_database_initialisation__get_user_internal_}\index{Apiis::DataBase::Init -- Basic database initialisation!\ get\ user (internal)}}


\textbf{\_get\_user} retrieves the data for the passed user from table 'users' and fills
an internal datastructure to access the needed values.



\textbf{\_get\_user} is invoked by Apiis::DataBase::Init::verify\_user.

\paragraph*{\_get\_user\_roles (internal)\label{Apiis::DataBase::Init_--_Basic_database_initialisation__get_user_roles_internal_}\index{Apiis::DataBase::Init -- Basic database initialisation!\ get\ user\ roles (internal)}}


All roles of the \$apiis-$>$DataBase-$>$users are fetched from database and stored
in the user objects.



\textbf{\_get\_user\_roles} is invoked by Apiis::Init::\_join\_database.

\paragraph*{users (public)\label{Apiis::DataBase::Init_--_Basic_database_initialisation_users_public_}\index{Apiis::DataBase::Init -- Basic database initialisation!users (public)}}


\textbf{users} returns all database users of this project.

\paragraph*{user (public)\label{Apiis::DataBase::Init_--_Basic_database_initialisation_user_public_}\index{Apiis::DataBase::Init -- Basic database initialisation!user (public)}}


\textbf{user} returns a User object for the passed user.

\paragraph*{verify\_user (internal)\label{Apiis::DataBase::Init_--_Basic_database_initialisation_verify_user_internal_}\index{Apiis::DataBase::Init -- Basic database initialisation!verify\ user (internal)}}


\textbf{verify\_user} checks the login data of the passed user object (name and
password) against the internal values from the database.

\paragraph*{crosstab\label{Apiis::DataBase::Init_--_Basic_database_initialisation_crosstab}\index{Apiis::DataBase::Init -- Basic database initialisation!crosstab}}


\textbf{crosstab} is a wrapper around the CPAN-Modules DBIx::SQLCrosstab and
DBIx::SQLCrosstab::Format. They allow a convenient way of creating
cross tabulations from the database and outputting it into different formats.
See 'man DBIx::SQLCrosstab' and 'man DBIx::SQLCrosstab::Format' for details.



\$apiis-$>$DataBase-$>$crosstab integrates DBIx::SQLCrosstab into the apiis
framework. It assumes all necessary parameters getting provided with via a
hash reference.



Input parameters:

\begin{verbatim}
   * $hash_ref->{params} with all parameters according to DBIx::SQLCrosstab.
     This includes a database handle (either user_dbh or sys_dbh).
   * $hash_ref->{format} defines the output format
   * $hash_ref->{aux} passes an additional parameter, which some formats
     expect, e.g. as_xls('filename'):
        $hash_ref->{format} = 'as_xls';
        $hash_ref->{aux}    = 'filename';
\end{verbatim}


Output parameters:

\begin{verbatim}
   * According to the documentation of DBIx::SQLCrosstab::Format
\end{verbatim}


Example:

\begin{verbatim}
   my $return_val = $apiis->DataBase->crosstab($hash_ref);
\end{verbatim}
\subsection{DBCreation.pm\label{DBCreation_pm}\index{DBCreation.pm}}




\subsubsection*{DESCRIPTION\label{DBCreation_pm_DESCRIPTION}\index{DBCreation pm!DESCRIPTION}}


This module is used in runall process to create initial database structure. The structure of the database 
is taken from the model file. New database is created in the system user schema (system user name is 
taken from the modelfile) and the public schema is removed. Then the languages are loaded from 
defined file and information about node is inserted (from apissrc settings). At the end sequences are
corectly set.

\subsubsection*{SUBROUTINES\label{DBCreation_pm_SUBROUTINES}\index{DBCreation pm!SUBROUTINES}}
\paragraph*{CreateDatabase\label{DBCreation_pm_CreateDatabase}\index{DBCreation pm!CreateDatabase}}


This subroutine call all subroutines which are written in this file. The parameters are defined as:
- \$db\_encoding  -  character encoding which will be used in the database 
- \$db\_name - databse name
- \$user\_creator - user name which creates database
- \$lang\_file -  file name where the initila languages are written (language.dat in the reference database) 
- \$lang\_dir - directory for the language file

\paragraph*{InitDB\label{DBCreation_pm_InitDB}\index{DBCreation pm!InitDB}}


This subroutine creates initial database structure. PUBLIC schema is removed and the database 
is placed in the system user schema. System user schema name is taken from the user name
defined in the modelfile.

\paragraph*{LoadLang\label{DBCreation_pm_LoadLang}\index{DBCreation pm!LoadLang}}


This subroutine load initial languages to the languages table.

\paragraph*{LoadNodeData\label{DBCreation_pm_LoadNodeData}\index{DBCreation pm!LoadNodeData}}


This subroutine load initial information about node. These information are inserted in the nodes table.

\paragraph*{SetSequences\label{DBCreation_pm_SetSequences}\index{DBCreation pm!SetSequences}}


This subroutine sets initial sequences.

\subsubsection*{AUTHORS\label{DBCreation_pm_AUTHORS}\index{DBCreation pm!AUTHORS}}
\begin{verbatim}
 Marek Imialek <marek@tzv.fal.de>
\end{verbatim}
\subsection{Apiis::DataBase::User -- collecting and providing user data\label{Apiis::DataBase::User_--_collecting_and_providing_user_data}\index{Apiis::DataBase::User -- collecting and providing user data}}




\subsubsection*{SYNOPSIS\label{Apiis::DataBase::User_--_collecting_and_providing_user_data_SYNOPSIS}\index{Apiis::DataBase::User -- collecting and providing user data!SYNOPSIS}}
\begin{verbatim}
   my $usr_obj = Apiis::DataBase::User->new( id => <userid>, %args );
\end{verbatim}
\subsubsection*{DESCRIPTION\label{Apiis::DataBase::User_--_collecting_and_providing_user_data_DESCRIPTION}\index{Apiis::DataBase::User -- collecting and providing user data!DESCRIPTION}}


To create a new User object, we need at least a user id.
A User object is created and returned.

\begin{verbatim}
   my $usr_obj = Apiis::DataBase::User->new( id => <userid> );
\end{verbatim}


Other parameters can be passed to fill the object at creation time:

\begin{verbatim}
   my $usr_obj = Apiis::DataBase::User->new(
      id       => <userid>,
      password => <top_secret>,
   );
\end{verbatim}


When you run

\begin{verbatim}
   $apiis->join_user( user_obj => $this_obj, %args );
\end{verbatim}


the User object \$this\_obj is joined into the \$apiis structure. This can happen
only once as only one user can run a program at a time. If no user\_obj is passwd,
join\_user falls back to ask for login data by itself.



Nevertheless is it possible, to create other User object, e.g. to insert a new 
user into the database.



With

\begin{verbatim}
   my $user_obj = $apiis->DataBase->user( <userid> );
\end{verbatim}


you can retrieve all user information, stored in the database, into a User object.

\subsubsection*{TODO\label{Apiis::DataBase::User_--_collecting_and_providing_user_data_TODO}\index{Apiis::DataBase::User -- collecting and providing user data!TODO}}


Maybe we get some class Apiis::Person later, whereof Apiis::DataBase::User is a
subclass. So User should be restricted to the data which is needed for database
connection (authentication and authorisation), whilst other personal data like name,
address, etc. should be retrieved from the unit/naming/address setup via some foreign
key. In table users, column login will then become the primary key (enforce uniqueness)
and user\_id could point to unit (or somewhere else).

\subsubsection*{METHODS\label{Apiis::DataBase::User_--_collecting_and_providing_user_data_METHODS}\index{Apiis::DataBase::User -- collecting and providing user data!METHODS}}
\paragraph*{new (public)\label{Apiis::DataBase::User_--_collecting_and_providing_user_data_new_public_}\index{Apiis::DataBase::User -- collecting and providing user data!new (public)}}


\textbf{new()} returns an object reference for a new User object.

\paragraph*{roles (public)\label{Apiis::DataBase::User_--_collecting_and_providing_user_data_roles_public_}\index{Apiis::DataBase::User -- collecting and providing user data!roles (public)}}


\$usr\_obj-$>$roles returns the roles of this user either as an array or an
arrayreference.
Roles are stored here by:

\begin{verbatim}
   $usr_obj->roles( \@these_roles );
   or
   $usr_obj->roles( 'role1', 'role2', 'roleN' );
\end{verbatim}
\paragraph*{password\label{Apiis::DataBase::User_--_collecting_and_providing_user_data_password}\index{Apiis::DataBase::User -- collecting and providing user data!password}}
\begin{verbatim}
   $usr_obj->password
\end{verbatim}


returns the encrypted password. If a new password is passed with

\begin{verbatim}
   $usr_obj->password( <new_password> );
\end{verbatim}


this password is first encrypted and then stored. If you want to store an already
encrypted password (like one from the database) you have to invoke it

\begin{verbatim}
   $usr_obj->password( <new_password>, encrypted => 1 );
\end{verbatim}
\paragraph*{authenticated (public)\label{Apiis::DataBase::User_--_collecting_and_providing_user_data_authenticated_public_}\index{Apiis::DataBase::User -- collecting and providing user data!authenticated (public)}}


This is a flag to show successful authentication of this user against his 
database password.



It is read-only to allow setting of this flag only from inside of \textbf{verify\_user}

\paragraph*{print (external)\label{Apiis::DataBase::User_--_collecting_and_providing_user_data_print_external_}\index{Apiis::DataBase::User -- collecting and providing user data!print (external)}}


Return the contents of the User object as a string, nicely formatted.
The elements of this object are printed according to the order of the @methods
array, if they contain data.

\paragraph*{print (external)\label{Apiis::DataBase::User_--_collecting_and_providing_user_data_print_external_}\index{Apiis::DataBase::User -- collecting and providing user data!print (external)}}


Print this user object to STDOUT (default) or the passed filehandle with the
formatting of sprint.



Examples:
  \$user\_obj-$>$print;
  \$user\_obj-$>$print( filehandle =$>$ *FILE );

\paragraph*{user\_language\_id (external)\label{Apiis::DataBase::User_--_collecting_and_providing_user_data_user_language_id_external_}\index{Apiis::DataBase::User -- collecting and providing user data!user\ language\ id (external)}}


\$usr\_obj-$>$user\_language\_id returns the language id as stored in table ar\_users
(new Auth/AR setup).

\paragraph*{lang\_id (external)\label{Apiis::DataBase::User_--_collecting_and_providing_user_data_lang_id_external_}\index{Apiis::DataBase::User -- collecting and providing user data!lang\ id (external)}}


\$usr\_obj-$>$lang\_id returns the language id as stored in table users (old,
deprecated Auth/AR setup).

\paragraph*{language (external)\label{Apiis::DataBase::User_--_collecting_and_providing_user_data_language_external_}\index{Apiis::DataBase::User -- collecting and providing user data!language (external)}}


\$usr\_obj-$>$language returns the value of iso\_lang from table languages with
the given lang\_id. It is preset with \$apiis-$>$language as default.

