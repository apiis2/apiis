\subsubsection{DataStream\label{DataStream}\index{DataStream}}




\paragraph*{SYNOPSIS\label{DataStream_SYNOPSIS}\index{DataStream!SYNOPSIS}}
\begin{verbatim}
    $ds = Apiis::DataBase::SQL::DataStream->new(
                                                ds     => $ds,
                                                debug  => $debug
   );
\end{verbatim}


This is the module for creating an object for handling datasteams.

\paragraph*{DESCRIPTION\label{DataStream_DESCRIPTION}\index{DataStream!DESCRIPTION}}


Creates the internal structure for the hash previously used in the batch processing and provides methods for accessing them.



Public and internal methods are:

\paragraph*{PUBLIC METHODS\label{DataStream_PUBLIC_METHODS}\index{DataStream!PUBLIC METHODS}}
\subparagraph*{ds\label{DataStream_ds}\index{DataStream!ds}}


- returns the name of the datastream

\subparagraph*{job\_start\label{DataStream_job_start}\index{DataStream!job\ start}}


- sets or returns the time when the job was started

\subparagraph*{job\_end\label{DataStream_job_end}\index{DataStream!job\ end}}
\begin{verbatim}
 - sets or returns the time when the job was finished
\end{verbatim}
\subparagraph*{records\_total\label{DataStream_records_total}\index{DataStream!records\ total}}


- sets or returns total number of processed records

\subparagraph*{records\_error\label{DataStream_records_error}\index{DataStream!records\ error}}


- sets or returns the number of errors that occure in the records - it is not precise and can give only impression of real number of errors

\subparagraph*{records\_ok\label{DataStream_records_ok}\index{DataStream!records\ ok}}


- sets or returns total number of records that was OK

\subparagraph*{data\label{DataStream_data}\index{DataStream!data}}


- sets or returns the data

\subparagraph*{LO\_keys\label{DataStream_LO_keys}\index{DataStream!LO\ keys}}


- sets or returns the list of LO\_keys

\subparagraph*{ext\_unit\label{DataStream_ext_unit}\index{DataStream!ext\ unit}}


- sets or returns the ext\_unit that has supplied this record

\subparagraph*{record\_seq\label{DataStream_record_seq}\index{DataStream!record\ seq}}


- sets or returns the record number from INSPOOL system

\subparagraph*{target\_column\label{DataStream_target_column}\index{DataStream!target\ column}}


- sets or returns an extra information for linking the errors to special column

\subparagraph*{debug\label{DataStream_debug}\index{DataStream!debug}}


- sets or returns debug level

\subparagraph*{verbose\label{DataStream_verbose}\index{DataStream!verbose}}


- sets or returns the verbose mode (more output messages)

\subparagraph*{sth\_update\_inspool\label{DataStream_sth_update_inspool}\index{DataStream!sth\ update\ inspool}}


- sets or returns statement handle for updating inspool

\subparagraph*{sth\_inspool\_err\label{DataStream_sth_inspool_err}\index{DataStream!sth\ inspool\ err}}


- sets or returns statement handle for inserting new record in inspool\_err

\subparagraph*{sth\_ds\label{DataStream_sth_ds}\index{DataStream!sth\ ds}}


- sets or returns statement handle for reading record from  inspool

\subparagraph*{sth\_load\_stat\label{DataStream_sth_load_stat}\index{DataStream!sth\ load\ stat}}


- sets or returns statement handle for inserting new record in load\_stat

\subparagraph*{status\label{DataStream_status}\index{DataStream!status}}


- sets or returns the object status - inherited from apiis

\subparagraph*{errors\label{DataStream_errors}\index{DataStream!errors}}


- sets or returns list of error objects - inherited from apiis

\subparagraph*{\_standard\_keys (internal)\label{DataStream__standard_keys_internal_}\index{DataStream!\ standard\ keys (internal)}}


- encapsulates the names of the automatically created methods

\subparagraph*{\_accessible (internal)\label{DataStream__accessible_internal_}\index{DataStream!\ accessible (internal)}}


- checks if the method is read-only or read-write

\subparagraph*{new\label{DataStream_new}\index{DataStream!new}}


- creates the datastream object

\subparagraph*{\_init (internal)\label{DataStream__init_internal_}\index{DataStream!\ init (internal)}}


- initializes the counters and prepares several database statements

\subparagraph*{CheckDS\label{DataStream_CheckDS}\index{DataStream!CheckDS}}


Verifies if the number of elements in the DS is the same as the number of LO keys

\subparagraph*{PostHandling\label{DataStream_PostHandling}\index{DataStream!PostHandling}}


Writes the summary statistics on the screen and in tableload\_stat and  error\_messages in table inspool\_err

\paragraph*{AUTHORS\label{DataStream_AUTHORS}\index{DataStream!AUTHORS}}


Helmut Lichtenberg $<$heli@tzv.fal.de$>$
Zhivko Duchev $<$duchev@tzv.fal.de$>$

\subsubsection{Apiis::DataBase::SQL::DirectSQL  Direct access to SQL\label{Apiis::DataBase::SQL::DirectSQL_Direct_access_to_SQL}\index{Apiis::DataBase::SQL::DirectSQL  Direct access to SQL}}




\paragraph*{SYNOPSIS\label{Apiis::DataBase::SQL::DirectSQL_Direct_access_to_SQL_SYNOPSIS}\index{Apiis::DataBase::SQL::DirectSQL Direct access to SQL!SYNOPSIS}}
\begin{verbatim}
   $apiis->DataBase->user_sql( $sql_statement );
   $apiis->DataBase->sys_sql( $sql_statement );
\end{verbatim}
\paragraph*{DESCRIPTION\label{Apiis::DataBase::SQL::DirectSQL_Direct_access_to_SQL_DESCRIPTION}\index{Apiis::DataBase::SQL::DirectSQL Direct access to SQL!DESCRIPTION}}


Both of these methods point to the internal work horse \textbf{\_sql} with the
invocation parameters as a hash reference:

\begin{verbatim}
   _sql( { statement => $sql_statement, user => 'system' } )
\end{verbatim}


in case of sys\_sql or without the 'user' parameter for user\_sql.



\textbf{\_sql} creates its own statement object, which is returned.

\paragraph*{METHODS\label{Apiis::DataBase::SQL::DirectSQL_Direct_access_to_SQL_METHODS}\index{Apiis::DataBase::SQL::DirectSQL Direct access to SQL!METHODS}}
\subparagraph*{user\_sql $|$ sys\_sql $|$ \_sql\label{Apiis::DataBase::SQL::DirectSQL_Direct_access_to_SQL_user_sql_sys_sql_sql}\index{Apiis::DataBase::SQL::DirectSQL Direct access to SQL!user\ sql $|$ sys\ sql $|$ \ sql}}


The only input parameter for \textbf{user\_sql} and \textbf{sys\_sql} is a valid SQL
statement. Returned will be a statement object with the following methods:



Examples:

\begin{verbatim}
    my $statement_obj    = $apiis->DataBase->user_sql('select * from codes');
    my $statement_handle = $statement_obj->handle;
    my $processed_rows   = $statement_obj->rows;
    if ( $statement_obj->status ) {
        for my $error ( $statement_obj->errors ) {
            # handle error:
            $error->print;
        }
    }
\end{verbatim}


See 'man DBI' for detailed information about the statement handle and rows.

\subsubsection{PseudoStatement\label{PseudoStatement}\index{PseudoStatement}}




\paragraph*{SYNOPSIS\label{PseudoStatement_SYNOPSIS}\index{PseudoStatement!SYNOPSIS}}
\begin{verbatim}
    $statement = Apiis::DataBase::SQL::PseudoStatement->new(
         pseudosql     => $sqltext,
         data_hash     => \%data_hash
   );
\end{verbatim}


This is the module for creating an object for parsed PseudoSQL.

\paragraph*{DESCRIPTION\label{PseudoStatement_DESCRIPTION}\index{PseudoStatement!DESCRIPTION}}


Creates the internal structure for the important elements of an PseudoSQL statement and provides methods for accessing them.
For parsing the PseudoSQL, the original parser written by Helmut Lichtenberg $<$heli@tzv.fal.de$>$ was used.



Public and internal methods are:

\paragraph*{PUBLIC METHODS\label{PseudoStatement_PUBLIC_METHODS}\index{PseudoStatement!PUBLIC METHODS}}
\subparagraph*{actionname\label{PseudoStatement_actionname}\index{PseudoStatement!actionname}}


- returns the sql action

\subparagraph*{tablename\label{PseudoStatement_tablename}\index{PseudoStatement!tablename}}


- returns the table name used in the statement - only one table is allowed per statement!

\subparagraph*{columns\label{PseudoStatement_columns}\index{PseudoStatement!columns}}
\begin{verbatim}
 - returns list of column names
\end{verbatim}
\subparagraph*{values\label{PseudoStatement_values}\index{PseudoStatement!values}}


- returns list of column values

\subparagraph*{extfields\label{PseudoStatement_extfields}\index{PseudoStatement!extfields}}


- returns list of external fields that are targeted for errors

\subparagraph*{value\label{PseudoStatement_value}\index{PseudoStatement!value}}


- returns the value of the supplied column

\subparagraph*{column\_extfields\label{PseudoStatement_column_extfields}\index{PseudoStatement!column\ extfields}}


- returns list of external fields for a certain column

\subparagraph*{whereclause\label{PseudoStatement_whereclause}\index{PseudoStatement!whereclause}}


- returns the where part of the statement

\subparagraph*{status\label{PseudoStatement_status}\index{PseudoStatement!status}}


- returns the object status - inherited from apiis

\subparagraph*{errors\label{PseudoStatement_errors}\index{PseudoStatement!errors}}


- returns list of error object - inherited from apiis

\subparagraph*{\_standard\_keys (internal)\label{PseudoStatement__standard_keys_internal_}\index{PseudoStatement!\ standard\ keys (internal)}}


- encapsulates the names of the automatically created methods

\subparagraph*{\_accessible (internal)\label{PseudoStatement__accessible_internal_}\index{PseudoStatement!\ accessible (internal)}}


- checks if the method is read-only or read-write

\subparagraph*{new\label{PseudoStatement_new}\index{PseudoStatement!new}}


- creates the parsed sql object

\subparagraph*{\_init (internal)\label{PseudoStatement__init_internal_}\index{PseudoStatement!\ init (internal)}}


- calls \_ParsePseudoSQL for parsing of the SQL and fills the structure

\subparagraph*{pull\_quotes\label{PseudoStatement_pull_quotes}\index{PseudoStatement!pull\ quotes}}
\begin{verbatim}
 pull_quotes -- tom christiansen, tchrist@convex.com
\end{verbatim}
\paragraph*{AUTHORS\label{PseudoStatement_AUTHORS}\index{PseudoStatement!AUTHORS}}


Zhivko Duchev $<$duchev@tzv.fal.de$>$
Helmut Lichtenberg $<$heli@tzv.fal.de$>$

\subsubsection{Statement\label{Statement}\index{Statement}}




\paragraph*{SYNOPSIS\label{Statement_SYNOPSIS}\index{Statement!SYNOPSIS}}
\begin{verbatim}
    $statement = Apiis::DataBase::SQL::Statement->new(
         sql     => $sqltext
   );
\end{verbatim}


This is the module for creating an object for parsed normal SQL. This module is intended for parsing simple SQL statements: INSERT,UPDATE,DELETE,SELECT (without aggregate functions).

\paragraph*{DESCRIPTION\label{Statement_DESCRIPTION}\index{Statement!DESCRIPTION}}


Creates the internal structure for important elements of SQL statement and provides methods for accessing them



Public and internal methods are:

\paragraph*{PUBLIC METHODS\label{Statement_PUBLIC_METHODS}\index{Statement!PUBLIC METHODS}}
\subparagraph*{actionname\label{Statement_actionname}\index{Statement!actionname}}


- returns the sql action

\subparagraph*{tablename\label{Statement_tablename}\index{Statement!tablename}}


- returns the table name used in the statement - only one table is allowed per statement!

\subparagraph*{columns\label{Statement_columns}\index{Statement!columns}}
\begin{verbatim}
 - returns list of column names
\end{verbatim}
\subparagraph*{values\label{Statement_values}\index{Statement!values}}


- returns list of column values

\subparagraph*{value\label{Statement_value}\index{Statement!value}}


- returns the value of the supplied column

\subparagraph*{whereclause\label{Statement_whereclause}\index{Statement!whereclause}}


- returns the where part of the statement

\subparagraph*{status\label{Statement_status}\index{Statement!status}}


- returns the object status - inherited from apiis

\subparagraph*{errors\label{Statement_errors}\index{Statement!errors}}


- returns list of error object - inherited from apiis

\subparagraph*{\_standard\_keys (internal)\label{Statement__standard_keys_internal_}\index{Statement!\ standard\ keys (internal)}}


encapsulates the names of the automatically created methods

\subparagraph*{\_accessible (internal)\label{Statement__accessible_internal_}\index{Statement!\ accessible (internal)}}


checks if the method is read-only or read-write

\subparagraph*{new\label{Statement_new}\index{Statement!new}}


- creates the parsed sql object

\subparagraph*{\_init (internal)\label{Statement__init_internal_}\index{Statement!\ init (internal)}}


- parses the SQL and fills the structure

\paragraph*{AUTHORS\label{Statement_AUTHORS}\index{Statement!AUTHORS}}


Zhivko Duchev $<$duchev@tzv.fal.de$>$

\subsubsection{Apiis::DataBase::MakeSQL Module to create SQL-statements from the model file\label{Apiis::DataBase::MakeSQL_Module_to_create_SQL-statements_from_the_model_file}\index{Apiis::DataBase::MakeSQL Module to create SQL-statements from the model file}}




\paragraph*{DESCRIPTION\label{Apiis::DataBase::MakeSQL_Module_to_create_SQL-statements_from_the_model_file_DESCRIPTION}\index{Apiis::DataBase::MakeSQL Module to create SQL-statements from the model file!DESCRIPTION}}


\textbf{MakeSQL} as the main method and some auxiliary routines read the Apiis model
file and create a file with SQL data definition commands to create tables,
views, indices, sequences, etc.



The resulting file will either be written to STDOUT or placed in the
var-subdirectory of the given project. Its name is created from the project
name with database driver and .sql extension appended:

\begin{verbatim}
   <project>_<db_driver>.sql
   breedprg_Pg.sql
\end{verbatim}
\paragraph*{METHODS\label{Apiis::DataBase::MakeSQL_Module_to_create_SQL-statements_from_the_model_file_METHODS}\index{Apiis::DataBase::MakeSQL Module to create SQL-statements from the model file!METHODS}}


Besides the main method \textbf{MakeSQL} there are the auxiliary methods
\textbf{Cascaded\_FK}, \textbf{resolve\_concatenations}, and \textbf{HasFKRule}.



Read 'perldoc \$APIIS\_HOME/bin/mksql' for the most prominent implementation and
for detailed usage information.

