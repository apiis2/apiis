\subsubsection{Apiis::DataBase::Record -- package for DataBase Record objects\label{Apiis::DataBase::Record_--_package_for_DataBase_Record_objects}\index{Apiis::DataBase::Record -- package for DataBase Record objects}}




\paragraph*{SYNOPSIS\label{Apiis::DataBase::Record_--_package_for_DataBase_Record_objects_SYNOPSIS}\index{Apiis::DataBase::Record -- package for DataBase Record objects!SYNOPSIS}}


This base package provides the functionality and methods needed for
database record object types.

\paragraph*{DESCRIPTION\label{Apiis::DataBase::Record_--_package_for_DataBase_Record_objects_DESCRIPTION}\index{Apiis::DataBase::Record -- package for DataBase Record objects!DESCRIPTION}}


The public and internal methods of this base class are described below.

\paragraph*{METHODS\label{Apiis::DataBase::Record_--_package_for_DataBase_Record_objects_METHODS}\index{Apiis::DataBase::Record -- package for DataBase Record objects!METHODS}}
\subparagraph*{new (public)\label{Apiis::DataBase::Record_--_package_for_DataBase_Record_objects_new_public_}\index{Apiis::DataBase::Record -- package for DataBase Record objects!new (public)}}


\textbf{new()} returns an object reference for a new record object.

\subparagraph*{name (public, readonly)\label{Apiis::DataBase::Record_--_package_for_DataBase_Record_objects_name_public_readonly_}\index{Apiis::DataBase::Record -- package for DataBase Record objects!name (public, readonly)}}


Returns the name of this record (usually identical with the tablename of
the database record):

\subparagraph*{columns (public, readonly)\label{Apiis::DataBase::Record_--_package_for_DataBase_Record_objects_columns_public_readonly_}\index{Apiis::DataBase::Record -- package for DataBase Record objects!columns (public, readonly)}}


Returns an array of all column names of this record in the order of the
model file structure.

\subparagraph*{column (public, readonly)\label{Apiis::DataBase::Record_--_package_for_DataBase_Record_objects_column_public_readonly_}\index{Apiis::DataBase::Record -- package for DataBase Record objects!column (public, readonly)}}


Returns a column object by column name. Example:

\begin{verbatim}
   my $col_obj = $record_obj->column($thiscolumn);
\end{verbatim}
\subparagraph*{addcolumn (public)\label{Apiis::DataBase::Record_--_package_for_DataBase_Record_objects_addcolumn_public_}\index{Apiis::DataBase::Record -- package for DataBase Record objects!addcolumn (public)}}


Adds a column object to the record. Maybe only for internal use.

\subparagraph*{delcolumn (public)\label{Apiis::DataBase::Record_--_package_for_DataBase_Record_objects_delcolumn_public_}\index{Apiis::DataBase::Record -- package for DataBase Record objects!delcolumn (public)}}


Deletes a column object from the record. Maybe only for internal use.

\subparagraph*{rows $|$ value $|$ fk\_table $|$ values (public, read/write)\label{Apiis::DataBase::Record_--_package_for_DataBase_Record_objects_rows_value_fk_table_values_public_read_write_}\index{Apiis::DataBase::Record -- package for DataBase Record objects!rows $|$ value $|$ fk\ table $|$ values (public, read/write)}}


Methods for return status parameters, e.g. for SQL query results.

\subparagraph*{print\label{Apiis::DataBase::Record_--_package_for_DataBase_Record_objects_print}\index{Apiis::DataBase::Record -- package for DataBase Record objects!print}}


\textbf{print} prints out the defined column values of the Record object, both
internal and external values and the associated ext\_fields.



Example:

\begin{verbatim}
   $record_obj->print;
\end{verbatim}


There are some switches to control the behaviour of \textbf{print}:

\begin{verbatim}
   quiet => 1         # if set to 1, informational output is reduced
   columns => \@cols  # prints out only the defined columns
   m_int   => 1       # displays also mirrored internal data
   m_ext   => 1       # displays also mirrored external data
   id_set  => 1       # displays also the id_set definition of the column
   all     => 1       # includes m_int, m_ext, id_set
   sprintf => 1       # doesn't print to STDOUT but returns an arrayref to the
                      output lines
\end{verbatim}


Example:

\begin{verbatim}
   $record_obj->print(
       {   quiet   => 1,
           columns => [qw/ db_animal guid /],
           m_int   => 1,
           id_set  => 1,
       }
   );
\end{verbatim}
\begin{verbatim}
   $record_obj->print(
       {   quiet   => 1,
           columns => [qw/ db_animal guid /],
           all     => 1,
       }
   );
\end{verbatim}
\begin{verbatim}
   my $output_ref = $record_obj->print( { sprintf => 1 } );
   print join("\n", @$output_ref);
\end{verbatim}


The parameters have to be passed as a hash reference, the columns as an array
reference.

\subsubsection{Apiis::DataBase::Record::Column -- package for DataBase Record columns\label{Apiis::DataBase::Record::Column_--_package_for_DataBase_Record_columns}\index{Apiis::DataBase::Record::Column -- package for DataBase Record columns}}




\paragraph*{SYNOPSIS\label{Apiis::DataBase::Record::Column_--_package_for_DataBase_Record_columns_SYNOPSIS}\index{Apiis::DataBase::Record::Column -- package for DataBase Record columns!SYNOPSIS}}


\textbf{Apiis::DataBase::Column} creates database columns that build an
Apiis::DataBase::Record and provides methods to access them.

\paragraph*{DESCRIPTION\label{Apiis::DataBase::Record::Column_--_package_for_DataBase_Record_columns_DESCRIPTION}\index{Apiis::DataBase::Record::Column -- package for DataBase Record columns!DESCRIPTION}}


The public and internal methods of this class are described below.

\paragraph*{METHODS\label{Apiis::DataBase::Record::Column_--_package_for_DataBase_Record_columns_METHODS}\index{Apiis::DataBase::Record::Column -- package for DataBase Record columns!METHODS}}
\subparagraph*{use\_entry\_view (public)\label{Apiis::DataBase::Record::Column_--_package_for_DataBase_Record_columns_use_entry_view_public_}\index{Apiis::DataBase::Record::Column -- package for DataBase Record columns!use\ entry\ view (public)}}


\textbf{use\_entry\_view} is a column method to allow encoding distinguish between
values from a table $<$Table$>$ or from the view entry\_$<$Table$>$. In case of
transfer this flag decides, if encoding takes the db\_animal from
entry\_transfer (an active animal) or from transfer directly, which may be any
db\_animal with these external data.

\subparagraph*{id\_set (public)\label{Apiis::DataBase::Record::Column_--_package_for_DataBase_Record_columns_id_set_public_}\index{Apiis::DataBase::Record::Column -- package for DataBase Record columns!id\ set (public)}}


The read/write method \textbf{id\_set} stores and offers the ID set information to
retrieve the right record from table transfer when decoding db\_animal.



Input:

\begin{verbatim}
   $rec_obj->column('db_animal')->id_set('HB');             # single value
   $rec_obj->column('db_animal')->id_set( 'HB', 'Piglet' ); # list
   my @id_sets = (qw/ HB Piglet Lifetime /);
   $rec_obj->column('db_animal')->id_set(@id_sets);         # array
   $rec_obj->column('db_animal')->id_set( \@id_sets );      # arrayref
\end{verbatim}


Output:

\begin{verbatim}
   my $id_set_ref = $rec_obj->column('db_animal')->id_set   # arrayref
   my @id_sets    = $rec_obj->column('db_animal')->id_set   # array
\end{verbatim}
\subparagraph*{best\_id\_set (public)\label{Apiis::DataBase::Record::Column_--_package_for_DataBase_Record_columns_best_id_set_public_}\index{Apiis::DataBase::Record::Column -- package for DataBase Record columns!best\ id\ set (public)}}


The read/write method \textbf{best\_id\_set} stores that ID set information, that has
been the best/first while trying to decode through the array of ID sets in
id\_set().

\subsubsection{Fetch\label{Fetch}\index{Fetch}}




\paragraph*{SYNOPSIS\label{Fetch_SYNOPSIS}\index{Fetch!SYNOPSIS}}
\begin{verbatim}
    # fill record with query data, then:
    $record_obj->fetch(
        expect_rows    => 'one',                   # ['one'|'many']
        expect_columns => qw/ db_sex db_breed/,    # all columns if absent
        order_by       => [                        # optional sorting
            { column => 'db_sex',   order => 'desc' },
            { column => 'db_breed', order => 'asc' },
        ],
        user => 'system',                          # optional for internal use
    );
\end{verbatim}


Apiis::DataBase::Record::Fetch fetches the specified record(s), creates a
record object for each of them and returns an array of these objects.

\paragraph*{DESCRIPTION\label{Fetch_DESCRIPTION}\index{Fetch!DESCRIPTION}}


The query is created according to the already filled-in data in this record
object.



All columns of this record object, which contain data, build
'column=value'-pairs. The operator is currently only '=' and all data fields
are ANDed.



Special allowed values are 'null' and 'not null', which are reflected in the
resulting where clause. Both strings are case insensitive.



Example:

\begin{verbatim}
    $rec_obj->column('exit_dt')->extdata('not null');
\end{verbatim}


or

\begin{verbatim}
    $rec_obj->column('db_breeder')->intdata('null');
    $rec_obj->column('db_breeder')->encoded(1);
\end{verbatim}


The qualifiers \textit{expect\_rows()} and \textit{expect\_colums()} can either be
provided as methods to the record object or as hash parameters to fetch.



\textit{expect\_rows()} can have the values 'one' and 'many'(default).



If \textit{expect\_columns()} is omitted, all columns of the record are retrieved.
Every records also retrieves the rowid/oid.



Example:

\begin{verbatim}
    my $rec_obj = Apiis::DataBase::Record->new( tablename => 'codes' );
    $rec_obj->column('class')->extdata('BREED');
    my @query_objects = $rec_obj->fetch(
        expect_rows    => 'many',
        expect_columns => [qw/ db_code ext_code /],
        sort_by        => [ { column => 'ext_code', order => 'asc' } ],
    );
\end{verbatim}


This query returns columns db\_code and ext\_code of all the records from codes,
where the BREED is coded. They are sorted by ext\_code in ascending order.
The resulting rows are packed into separate record objects each and passed
back in an array of record objects.



\$rec\_obj-$>$expect\_rows('many') is default and can be omitted.



Another example:

\begin{verbatim}
    my $rec_obj = Apiis::DataBase::Record->new( tablename => 'animal' );
\end{verbatim}
\begin{verbatim}
    # if you know the internal code:
    $rec_obj->column('db_animal')->intdata(8608);
    # you must set the encoded(1) flag for internal data:
    $rec_obj->column('db_animal')->encoded(1);
\end{verbatim}
\begin{verbatim}
    # or, if you have the external data:
    $rec_obj->column('db_animal')->extdata( 'society|sex', '32|1', '63' );
\end{verbatim}
\begin{verbatim}
    # now specify the query:
    $rec_obj->expect_columns(qw/ db_sex db_breed /);
    $rec_obj->expect_rows('one');
    $rec_obj->fetch;
\end{verbatim}


If you provide the external data, it is encoded before the query is created:

\begin{verbatim}
   SELECT db_sex,db_breed from animal where db_animal = 8608;
\end{verbatim}


If this results in more than one record (which should not happen), an error is
risen.



Note: If you don't want the record to get encoded (perhaps you provide
internal data anyway) you have to set \$record-$>$encoded(1), which will skip
encoding. In case of providing internal data without external one, leaving out
\$record-$>$encoded(1) will yield an error. During encoding, the internal
value will simply get deleted.



Ordering the query result can be done with:

\begin{verbatim}
    $record_obj->fetch(
        order_by => [
            { column => 'birth_dt', order => 'desc' },
            { column => 'parity',   order => 'asc' },
        ],
    );
\end{verbatim}


\textbf{order\_by} expects an array reference of hash references. According to the
SQL standard, this query is sorted by column birth\_dt in descending order and,
for equal values in birth\_dt, by column parity in ascending order.



This makes mainly sense for columns without encoded values as the sorting
happens on the internal database values.

