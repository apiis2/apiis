\chapter{Fixed APIIS Structure}

This chapter describes that part of the database structure which is
the same in all APIIS database.


\section{TRANSFER}


\section{CODES}


\section{UNIT, NAMING, ADDRESS}

These three tables cosntitute one block which administers all units,
persons and addresses in the database. As such it should be quite
generic and should be used, largely unchanged in every database.


\subsection{ADDRESS}

This table holds the individual physical addresses, one entry for
each. The primary key is the sequence DB\_ADDRESS


\subsection{NAMING}

Here we store all individual persons and organization which are part
of the system. Many people can have the same address, that is why
in a normalized way this has been split into ADDRESS and NAMING. The
primary key is the DB\_NAME which is a sequence.


\subsection{UNIT}

EXT\_ID is the external identification used in the outside world.
This may be the number or code {}``1'' for the veterinarians. Then
EXT\_UNIT would be {}``VET''. If we would merge veterinarians from
Bavaria and Saxony in one system, and the vets were numbered within
each state beginning with {}``1'', the EXT\_UNIT would be {}``VET-SAX''
and {}``VET-BAV''. In this way the old numbering systems can continue
to be used (which is actually the motivation for the EXT\_UNIT/EXT\_ID
setup. Thus, EXT\_UNIT has to be defined such that the EXT\_IDs being
used are made unique.

One other objective is the possibility to reuse external numbers.
Assume that the head veterinarian with EXT\_ID=''1'' retires and
is replaced by his successor. This person should again get the EXT\_ID=''1''.
Then, the same as in TRANSFER, the old EXT\_ID gets closed (CLOSING\_DT
is set) and a new record with the EXT\_=ID=''1'' is created.

The unique index is EXT\_UNIT||EXT\_ID where CLOSING\_DT is NULL.
This means that we are allowing one {}``open'' channel for the combination
of external unit and external identification.


\subsection{Navigation}

Access to ADDRESS and NAMING is straight forward. Let us assume a
new address is to be entered. The following steps are performed:

\begin{enumerate}
\item verify that the address to be entered does not already exist. This
can be done by search the table ADDRESS on any fields with a LIKE
expression.
\item enter a new address. The primary key is simply the sequence DB\_ADDRESS.
Note that EXT\_ADDRESS is just a plain column which may be used for
locating a record in an interactive manner. Notice that COUNTRY needs
to be defined in CODES (Foreign Key).
\end{enumerate}
It is the sole responsibility of the user to ensure that no duplicate
addresses are entered. While this may sound dangerous it is actually
the only logical way. If an address is entered twice there is really
no harm done as this and the other may be used together as DB\_ADRESS
will be used to link the address to people.

NAMING is delt with in a similar manner. The insertion process is
identical to ADDRESS:

\begin{enumerate}
\item verify that the person/organization to be entered does not already
exist by searching on any of the fields interactively.
\item insert the new person/organization. The primary key DB\_NAMING will
automatically come from a the sequence seq\_naming\_\_db\_nam. Notice
that LANGUAGE needs to be defined in CODES (Foreign Key).
\end{enumerate}
UNIT has a somewhat different position in this triplet. It defines
the telefonnumbers, fax, e-mail belonging to a person or organization
for a given role. To create a new external ID, let us say the second
veterinarian, we would do the following:

\begin{enumerate}
\item type EXT\_UNIT='VET' and EXT\_ID='2' into unit

\begin{enumerate}
\item locate the address of this person in ADDRESS (remember DB\_ADDRESS)
\item locate the person in NAMING (remember DB\_NAMING)
\end{enumerate}
\item insert record in UNIT with all data using DB\_ADDRESS and DB\_NAMING,
DB\_UNIT is derived from the sequence, OPENING\_DT is set to the current
date while CLOSING\_DT is NULL.
\end{enumerate}
Thus far, no statemens have been made as to the status of DB\_UNIT.
While in ADDRESS and NAMING the corresponding DB\_ columns function
as primary keys, this is not the case here. Thus, if the user choses
to create another {}``input channel'' for a different EXT\_UNIT/EXT\_ID
with the same DB\_UNIT nothing will prevent her from doing this. What
needs to be considered is however that a select on DB\_UNIT will then
return two records instead of one.
