%%%%%%%%%%%%%%%%%%%%%%%%%%%%%% Textclass specific LaTeX commands.
 \newenvironment{lyxcode}
    {\begin{list}{}{
                           \setlength{\rightmargin}{\leftmargin}
                                \setlength{\listparindent}{0pt}% needed for
                                   AMS classes
                                        \raggedright
                                             \setlength{\itemsep}{0pt}
                                     \setlength{\parsep}{0pt}
                                          \normalfont\ttfamily}%
                                                 \item[]}
                                                    {\end{list}}

\chapter{The business and modify rules}


\section{The business rules}

The currently implemented business rules are shown in table \ref{tab:business rules}
on page \pageref{tab:business rules}.

\begin{table}[htbp]

\caption{Implemented Business Rules\label{tab:business rules}\index{business rules!Range}\index{business rules!List}\index{business rules!IsANumber}\index{business rules!NoNumber}\index{business rules!IsAFloat}\index{business rules!Unique}\index{business rules!ForeignKey}\index{business rules!DateDiff}\index{business rules!LastAction}\index{business rules!IsEqual}\index{business rules!ReservedStrings}}

\begin{center}{\small \vspace*{4mm}}\begin{sideways}
\begin{tabular}{|l|p{100mm}|p{50mm}|}
\hline 
\textbf{\small Method}&
\textbf{\small Description}&
\textbf{\small Example}\tabularnewline
\hline
\hline 
{\small Range}&
{\small value has to be within given range}&
{\small {}``Range 10 33''}\tabularnewline
\hline
{\small List}&
{\small only values from this list are allowed}&
{\small {}``List LR LW PI DU''}\tabularnewline
\hline
{\small NotNull}&
{\small has to have a value}&
{\small {}``NotNull''}\tabularnewline
\hline
{\small IsANumber}&
{\small only a number is accepted (5, -.37, +5.3e-3)}&
{}``IsANumber''\tabularnewline
\hline
{\small NoNumber}&
value must not contain any number &
{}``NoNumber''\tabularnewline
\hline
{\small IsAFloat}&
{\small value must be a floating point number (+4.3, .27, -0.5)}&
{}``IsAFloat''\tabularnewline
\hline
{\small Unique}&
{\small the value is a unique key in the current table}&
{\small {}``Unique''}\tabularnewline
\hline
{\small ForeignKey}&
{\small value has to have an entry in the specified table and column,
possibly with some additional conditions}&
{\small \mbox{{}``ForeignKey animal db\_animal'',}}{\small \par}

{\small ''ForeignKey v\_animal db\_animal ext\_sex=1 ext\_breed=DL''}\tabularnewline
\hline
{\small DateDiff}&
{\small the difference between the content of farrowing date and current
columns of the current record must be with the given range}{\small \par}

{\small DateDiff takes the current value of the record and computes
the difference to date given in the first parameter (compare\_date).
If the difference (in days) between the cure and compare\_date is
in the range given by min\_diff and max\_diff the rule is passed successfully. }{\small \par}

{\small compare\_date can either be a fixed format date like Mar-22-2001,
a column of the current record or a date in a database table/column
with the (hardcoded) db\_animal value of the current record. The syntax
for this format is 'tablename=>columname'. If no compare date comes
from the database the check is also successfully. }&
{\small \mbox{{}``DateDiff farrow\_dt 20 56'',}}{\small \par}

{\small \mbox{{}``DateDiff Mar-22-2001 50 90'',}}{\small \par}

{\small \mbox{{}``DateDiff animal=>birth\_dt 1 65''}}\tabularnewline
\hline
{\small LastAction}&
{\small LastAction is a conditional DateDiff depending on the value
of the last action. For each element of this last action list an allowed
range (in days) has to be specified.}{\small \par}

{\small If last action was SEL{[}ECTION{]} the range is 20 100, if
it was AI the range is 18 30}&
{\small {}``LastAction SEL 30 100 AI 18 30 LITTER 10 34''}\tabularnewline
\hline
IsEqual&
This example is placed in the CHECK attached to column db\_sire e.g.
in service and tests if the animal ID given is indeed a male. Thus,
the second parameter (i.e. \$data\_column) takes its value from the
current column Notice that the constant is specified as external code.&
IsEqual ( animal db\_animal db\_sex 'M'); \tabularnewline
\hline
ReservedStrings&
{\small ReservedStrings checks if the passed \$date contains one of
the reserved strings which are defined in pdblrc.}&
{\small {}``ReservedStrings''}\tabularnewline
\hline
\end{tabular}
\end{sideways}\end{center}
\end{table}



\section{The Modify Rules}

Sometimes it is necessary to modify the incoming value before it is
fed to the business rules. These modify methods are: \textbackslash{}par\textbackslash{}vspace\{4mm\}

\begin{center}\label{encode}\begin{tabular}{|l|p{80mm}|l|}
\hline 
\textbf{Method}&
\textbf{Description}&
\textbf{Example}\tabularnewline
\hline
\hline 
UpperCase&
converts all passed date into uppercase letters&
{}``UpperCase''\tabularnewline
\hline 
LowerCase&
converts all passed date into lowercase letters&
{}``LowerCase''\tabularnewline
\hline 
ConvertBool&
accepts YyJjNn and converts it to the appropriate boolean expression
(true/false)&
{}``ConvertBool''\tabularnewline
\hline 
CommaToDot&
translates all commas , into dots . (mainly used for numerical date)&
{}``CommaToDot''\tabularnewline
\hline 
DotToColon&
translates all dots . into colons : (useful for fast typing of date/time
values (16.34.00 => 16:34:00&
{}``DotToColon''\tabularnewline
\hline 
SetNow&
sets the value to the current time&
{}``SetNow''\tabularnewline
\hline
SetUser&
sets the value to the user who is running this job&
{}``SetUser''\tabularnewline
\hline
\end{tabular}\end{center}


\section{Layering\index{Layering} of Business Rules}

As described above, all business rules are specified as properties
of the the columns nd are thus part of their definition. This results
in one set of rules applied to any database modifications. However,
there may be circumstances that one set of rules is not sufficient
to describe all data coming into the database. For instance, the database
may contain data from the nucleus level of a breeding program and
also data from the production level. Clearly, business rules may be
different for the two types of data. To accommodate this situation
APIIS has implemented sets or layers of business rules. The philosophy
behind this is, that data streams can be subdivided into distinct
classes of data which have their own set of rules. Examples are (as
mentioned above) nucleus versus production level. Others could be
fat breeds versus less fat (like Meishan vs Landrace).

Operationally, business rules layers are defined as additional CHECK
in the model file. They are written as CHECK1\index{CHECK1} for a
layer 1, CHECK2\index{CHECK2} for a layer 2 etc. An example is given
in table \ref{cap:check levels}. The column db\_sex requires is a
foreign key in CODES and must not be NULL in the base (default) level
as indicated by the the key CHECK. If no check\_level (chk\_lvl\index{chk\_lvl})
is specified those given by CHECK apply. Its explicit chk\_lvl is
0. On the other hand the chk\_lvl=1 has only the foreign key requirement.
Thus, at this level NULL values for sex will be allowed. 

The procedure for specifying and using layered business rules is as
follows:

\begin{enumerate}
\item determine the number of layers of business rules required in your
set of data streams. This means that you should group together classes
of records that have similar requirements regarding the business rules.
Examples are: nucleus population, multiplier level, production level.
Or small breeds version large breeds. Also, combinations are possible.
These levels should get entries for documentation purposes in CODES\index{CODES}
under class CK\_LVL.
\item write the basic set of rules in the model file using the key CHECK.
The set of rules specified here will be the basic level that is used.
Thus, it will be used if either CHK\_LVL is set to 0 or not set at
all. Then specify for each check level a corresponding CHECKn rule.
Thus, if you decided to have three check levels you will have CHECK,
CHECK1 and CHECK2 in your model file. While CHECK should be specified
as the base set of rules, the other CHECKs are specified only if the
base CHECK is not applicable. Thus, whenever a CHECK key exists for
a given column corresponding the chk\_lvl specified it will replace
the base CHECK. Then this set of business rules will be executed.
As can be seen in table \ref{cap:check levels} only for column db\_sex
are the business rules modified in level 1. In db\_breed only the
base is specified, thus for all other chk\_lvl that may be specified
only the base set of checks are performed.
\item as has been said above, prior to enforcing the business rules the
programmer needs to specify which level she/he wants to fire. This
is typically done in the load object by calling the routine: \$pdbl->Model->set\_checklevel\index{set\_checklevel}
(1); In this example it would be set to 1. If you specify a check
level that does not exists in the model file at least once the calling
program should stop.
\item With a number of possible check levels, the current level that was
used when the database content was modified (insert or update) needs
to be stored with the record in each table. This is done in the column
CHK\_LVL. This is read and used in the program check\_integrity\index{check\_integrity}
to fire the correct set of business rules.
\end{enumerate}
%
\begin{table}[htbp]

\caption{\label{cap:check levels}Specifying layers of business rules}

\begin{lyxcode}
{\scriptsize col002~=>~\{~DATA~=>~'',~}{\scriptsize \par}

~{\scriptsize ~~~~~~~~~~~DB\_COLUMN~~~=>~'db\_sex',~}{\scriptsize \par}

~{\scriptsize ~~~~~~~~~~~DATATYPE~~~~=>~'BIGINT',~}{\scriptsize \par}

~{\scriptsize ~~~~~~~~~~~LENGTH~~~~~~=>~'1',~}{\scriptsize \par}

~{\scriptsize ~~~~~~~~~~~DESCRIPTION~=>~'sex',~}{\scriptsize \par}

~{\scriptsize ~~~~~~~~~~~DEFAULT~~~~~=>~'',~}{\scriptsize \par}

~{\scriptsize ~~~~~~~~~~~CHECK~~~~~~~=>~{[}'ForeignKey~codes~db\_code',~'NotNull'{]},~}{\scriptsize \par}

~{\scriptsize ~~~~~~~~~~~CHECK1~~~~~~=>~{[}'ForeignKey~codes~db\_code'{]},~}{\scriptsize \par}

~{\scriptsize ~~~~~~~~~~~MODIFY~~~~~~=>~{[}{]},~}{\scriptsize \par}

~{\scriptsize ~~~~~~~~~~~ERROR~~~~~~~=>~{[}{]},~}{\scriptsize \par}

{\scriptsize \},~}{\scriptsize \par}

{\scriptsize col003~=>~\{~~~~~~~~~~}{\scriptsize \par}

~{\scriptsize ~~~~~~~~~~~DATA~~~~~~~~=>~'',~~~~~~~~~~}{\scriptsize \par}

~{\scriptsize ~~~~~~~~~~~DB\_COLUMN~~~=>~'db\_breed',~~~~~~~~~~}{\scriptsize \par}

~{\scriptsize ~~~~~~~~~~~DATATYPE~~~~=>~'BIGINT',~~~~~~~~~~}{\scriptsize \par}

~{\scriptsize ~~~~~~~~~~~LENGTH~~~~~~=>~'2',~~~~~~~~~~}{\scriptsize \par}

~{\scriptsize ~~~~~~~~~~~DESCRIPTION~=>~'breed',~~~~~~~~~~}{\scriptsize \par}

~{\scriptsize ~~~~~~~~~~~DEFAULT~~~~~=>~'',~~~~~~~~~~}{\scriptsize \par}

~{\scriptsize ~~~~~~~~~~~CHECK~~~~~~~=>~{[}'NotNull',~'ForeignKey~codes~db\_code'{]},~~~~~~~~~~}{\scriptsize \par}

~{\scriptsize ~~~~~~~~~~~MODIFY~~~~~~=>~{[}{]},~~~~~~~~~~}{\scriptsize \par}

~{\scriptsize ~~~~~~~~~~~ERROR~~~~~~~=>~{[}{]},~~~}{\scriptsize \par}

{\scriptsize \},~}\end{lyxcode}

\end{table}



