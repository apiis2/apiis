\chapter{Initial Loading of Historic Data}

Very rarely, a new information system can be set up without any previouly
collected data. This implies the necessity to load all previously
collected data into the new database before the routine data flow
can start reaching the database. Historic data will typically be available
from a number of different and often independant source or oven organization.
As a consequence, many logical inconsistancies will be found in the
data once they are loaded into a target structure that defines a possibly
stringent set of business rules. As historic data may come in non
predictable way (many files, any format), a procedure was developed
which is as far as possible generic and thereby adaptable to any type
of animal data recording system. It therefore allows any number of
data files to be involved and furthermore, develops a strategy to
deal with the potentially large number of errors that will become
obvious once loaded into the business rule centered database.


\section{The General Procedure}

Once the structure of the database has been defined it has to be populated
with historic data, i.e. information already available in some computer
compatible form. As has been described above, all external codes or
identifications are translated into internal database codes to allow
for reuse of external codes should the meaning of them change at some
time. 

\begin{enumerate}
\item Operate on codes and external identification

\begin{enumerate}
\item Collect all temporary external codes
\item check external codes and determine their target code to go into the
database
\item determine illegal values and handle them
\item identify and handle duplicate identifications
\item create input channels for corrected external codes in the database.
The external codes can thus also be viewed as foreign keys which need
to be available in the database prior to load the data.
\end{enumerate}
\item Operate on data

\begin{enumerate}
\item load historic data using the channels created in the previous step
\item verify the loaded data against the business rules and flag records
with errors in the database
\end{enumerate}
\item Wrap up

\begin{enumerate}
\item rewrite the temporary external animal identification to the external
IDs reaching the database in routine data flow
\end{enumerate}
\end{enumerate}
Now we shall describe the steps in a little more detail.

\begin{enumerate}
\item Operate on codes and external identification


In this step only external codes and external identifications are
considered, leaving other data on animals like their measurement aside.
It is a step which basically collects, edits and loads Foreign Key
information into the database. From the programming perspective, this
process can be highly generic with little programming required by
the adaptor (once the generic code has been written).

\begin{enumerate}
\item {}``Collect all external codes'': In APIIS we have three groups
of external codes:

\begin{itemize}
\item animal identifications; this can be any combinations of data fields
that identify an animal uniquely in the historic data set. This is
not necessarily the external ID that may be used in later routine
data reporting. Its sole purpose is to identifiy individual animals
correctyl in the historic data sets. The definition of this temporary
external ID can be different from the external IDs used and reported
during routine data flow.
\item an address or partner identification
\item classical codes like the numeric representation of a breed or a code
for a certain abnormality
\end{itemize}
\item {}``check external codes'': like all data also identifications and
codes can be erroneous or inconsistant. As an example the breed {}``Pietrain''
may be coded in some files as {}``PI'' in others as {}``pi''.
Not only should the data made consistant but also a procedure must
exist to handle records that are identified as being incorrect. The
basic principles that we have followed is to provide for each external
code or identication the opportunity to translate it into a new target
value or as illegal by manual intervention. Then in the following
process (like loading of data) any occurence of the original value
will be replaced by its translated value.
\item {}``determine illegal values'': some identifications will be illegal.
For example, unknown sires may be codes as {}``~~~~~~'' or
as {}``999999''. These are not legal external identifications, and
should thus not be used to create a data channel. Also, if these IDs
come up later on in the loading process they should be skipped.
\item {}``duplicates'': The same ID must not be used for different animals.
This can happen when animal IDs are being reused after a number of
years. For some records we know that they should appear only once
for an animal: a field performance test may be an example. Here we
can test for multiple occurences. If duplicates are found then human
interaction is required to resolve the problem. 
\item {}``create input channnels'': Once all the target codes have been
established by manual human intervention, likewise duplicates and
undefined IDs have been dealt with, the external IDs get loaded into
the database. This populate the tables TRANSFER , CODES and UNIT.
\end{enumerate}
\item Operate on data


After the {}``foreign keys'' have been collected, edited and loaded
in the first block, now actual measurements have to be loaded into
the database. The translation tables edited manually in the first
step are also used here as filter.

\begin{enumerate}
\item {}``load historic data'': For each file containing historic data
a parameter file has to be written. This requires knowledge of the structure
of the data files and the database. The definition of the external
codes are identical to those in the first step. They are then passed
through the translation created by manual intervention. However, the
business rule layer is bypassed as time sequence dependant checks
cannot be made because data get processed in random order.
\item {}``Verify the loaded data'': Thus far all database modifications
have bypassed the business rule layer of the model file. The business
rule set can only be used if data come in correct time sequences which
clearly cannot be done when loading historic data. Thus, conflicts
with the business rules can be expected. Rows violating the business
rules will be flagged {}``dirty''. This allows a selection of correct
data during later database operation. Furthermore, {}``dirty'' records
can be cleaned up successively later on.
\end{enumerate}
\item Wrap up


In the previous step all historic data has been loaded, verified against
the business rules and flagged accordingly. Now the routine data flow
can start using the data stream model. But before this can be done,
the data channels may have to be translated from the temporary external
animal ID to the ID that comes in via the routine data streams. The
temporariy external IDs were created for the purpose of uniquely identifying
data records in the historic datasets. Thus, after loading, they have
served their purpose and may be modified. In the historic dataset,
initially there may have been duplicate IDs refering to different
animals. By adding the birth year they would have been made unique:
4711 in 1990 and in 1999 would have been identified as 4711$|$90 and
4711$|$99. The former animal will not be active any more while data
for the latter will still come into the system. However, it will be
identified in the data stream only as 4711. Thus, its external number
can be changed to 4711. Accordingly, incoming data to animal 4711
will find an open data channel and be translated to the correct internal
database number.

\end{enumerate}

\section{Description of the loading process for the reference database}

As indicated above the reference database contains a script that should
carry out all steps to create the actual reference database starting
with the initial ASCII files containing the historic data. Prerequisites
are:

\begin{itemize}
\item set APIIS\_LOCAL to \$APIIS\_HOME/apiis \\
(with bash: export APIIS\_LOCAL=\$APIIS\_HOME/apiis)
\item add bin/ and lib/ to the search path \\
(with bash: export PATH=\$PATH:\$APIIS\_HOME/bin:\$APIIS\_HOME/lib:
...)
\item the postmaster has to be up and running (use the -F option to speedup
loading; possible -i to accept TCP/IP connections)
\item the current user has to have the right to create and destroy a database
\item run runall.pl \index{runall.pl} (the total one may take a couple
of hours or so depending on the speed of the computer, because the
check routines need a lot of time; 7 hours on pIII 256 Mb RAM, until
check\_integrity nearly 3 hours)
\item a log is written as runall.log\index{runall.log} for the succesfull
run and as runall\_long.log\index{runall\_long.log} for statistics
in data loading
\end{itemize}
In the following we shall describe the procedure which is generally
applicable to any loading process of historic data (as we hope). However,
we shall use the reference database for detail description. Table
\ref{runall.pl} gives a listing of the perlscript. To load the complete
reference database no configuration is necessary. However, this script
can also be used as the basis for user specific loading. Then the
following lines need adaptation:

\begin{lyxlist}{00.00.0000}
\item [line~75]for testing purposes it is useful to load only a (consistant)
subset of all historic data. The number of records loaded is set via
\$max\_rec. To load all data use a number larger than the number of
records in the data file accessed in program S6\_collect\_ignore\_animal\index{collect\_ext\_id.pl}
\item [line~78~etc]the array @job holds all the programs that need to
run. These are the same as shown in table \ref{runall.pl}.
\item [line~153]here define the used project. In this setup it is
  possible to run seperate projects side by side. The projectname has
  to be the same as defined in \$APIIS\_HOME/etc/apiisrc section PROJECTS.
\end{lyxlist}
%
\begin{table}[htbp]

\caption{creating the complete reference database (runall.pl)\label{runall.pl}}

\begin{minipage}{1.1\textwidth}
\begin{Verbatim}[fontsize=\scriptsize, frame=single, numbers=left ]
"S1_add_codes()",                                      # step 1
"S2_collect_codes(-m $max_rec)",                       # step 2
                                                       # step 3 manual edit
"S3_load_codes('-f codes.chg.cvs')",                   # step 4
"S4_collect_jobs($max_rec)",                           # step 5
                                                       # step 6 manual edit
"S5_load_jobs('-f job.chg.cvs')",                      # step 7
"S6_collect_ignore_animal('-i -m $max_rec')",          # step 8 find all ext
                                                       # and leave UNDEFINED in file
                                                       # step 9 manual edit
"S7_collect_dup_animal('-d -n hb_not_uniq.txt -m $max_rec')", # step 10 find duplicates
                                                       # step 11 manual edit
"S8_load_ext_animal('-f dup_animal.chg.cvs -m $max_rec -c ignore_animal.ok')", #step 12
"S9_transfer_to_animal()",                             # step 13
"S10_load_data('-p herdbook.par -m $max_rec')",        # step 14 load the data herdbook
"S10_load_data('-p station.par -m $max_rec -f codes.chg.cvs')",    # step 15 station
"S10_load_data('-p field.par -m $max_rec -f codes.chg.cvs')",      # step 16 field
"S10_load_data('-p litter.par -m $max_rec -f codes.chg.cvs')",     # step 17 litter
"S10_load_data('-p address.par -m $max_rec')",                     # step 18 address
"S11_clear_index()",                                   # step 19
"S12_dates_to_transfer_hb('-m $max_rec')",             # step 20
"S13_dates_to_transfer_station('-m $max_rec')",        # step 21
"S14_fill_last_action()",                              # step 22 fill la_rep
"S15_post_initial()",                                  # step 23 rewrite ext_animal
                                                       # step 24 manual edit 'codes_unit.chg'
"S16_collect_codes2('-f codes_unit.chg.cvs')",         # step 25 new codes from post_initial
"S17_clear_transfer.pl", # step 26 if duplicate animals exist after rewriting
"check_integrity -f ../model/breedprg.model -D -g check.erg", # step 28 (set dirty)
\end{Verbatim}
\end{minipage}

\end{table}

Now the specific steps will be described on the basis of the reference
database.

\begin{enumerate}
\item make a list of all files that hold historic data
\item start with identification of adult animals

\begin{enumerate}
\item determine the unique external id of the animal. In the reference dataset
this is:

\begin{enumerate}
\item breed society (there are sires from other breed societys in station
test records)
\item herdbook number
\item sex
\end{enumerate}
\item for each file determine the columns to create the EXT\_ID on the above
basis
\end{enumerate}
\item load all ext\_ids in a vector and note unique (u) and multiple (m)
with each ext\_id:

\begin{lyxlist}{00.00.0000}
\item [m:]multiple occurences to be expected, these are occurences as parents,
or sows with litter records
\item [u:]typically, in herdbook files we should have only one record for
each ext\_id. Thus, if an ext\_id (as defined above) occurs more than
once we know that we have a mistake.
\end{lyxlist}
\end{enumerate}
When developing a new information system on the basis of APIIS normally
historic data exists that has been collected by some other system
or by a number of different systems. The problems and procedures to
integrate all historic data in a new consistent APIIS based system
is described in this chapter.

Because initial loading has to deal with data accumulated over time
no time dependent business rules can be enforced. This means that
during this phase the business rule\index{business rules} enforcement
system will be switched off. However, this means that after this phase
the freshly loaded system has to be verified extensively to match
the accuracy requirements stated in the model file.


\section{Initial Datasets}

Here it is assumed that the initial datasets are available in the
form of flat ASCII files. These files constitute the complete historic
data that are intended to be loaded into the central database. Furthermore,
this implies that never more historic data is to be included. Thus,
there will be only one initial step to load historic data. After this
process has been carried out the newly established database will be
the sole repository of the data. New data will afterwards be entered
into the database via the time dependent stream entry using the model
file.

The files used in the reference database reside in the directory \\
\$APIIS\_HOME/ref\_breedprg/initial and are:

\begin{enumerate}
\item herdbook data \dotfill herdbook.dat
\item field test data \dotfill field.dat
\item station test data \dotfill station.dat
\item reproduction data \dotfill litter.dat
\item address data \dotfill address.dat
\end{enumerate}
Loading historic data requires the following steps:

\begin{enumerate}
\item list all data streams and their content (DS01 -- DSnn)\label{enu:list-all-data-streams}
\item normalize the database structure\label{enu:normalized-the-database}
\item list all files of historic data (in the reference database we have
the 5 files given above)\label{enu:list-all-files-hist}
\item create all other foreign keys such as codes and addresses\label{enu:create-all-other-FK}
\item create external identifications for all animals referd to in the historic
data files\label{enu:create-external-identifications}
\item load historic data\label{enu:load-historic-data}
\item rewrite TRANSFER for currently used external
  identifications\label{enu:rewrite-TRANSFER}
\item check integrity against the defined business rules
\end{enumerate}
\begin{comment}
create conditional index in ENTRY\label{enu:create-conditional-index-ENTRY}
\end{comment}

\section{Step 1: Adding code}

The program S1\_add\_codes.pl\index{add\_codes.pl} is used to fill the
CODES\index{codes} table with the types that are being used in the
later loading process (see \ref{codes}). These are codes which are
not came from data but from datastream and/or position in data, like
type of the weight (station start test or fieldtest...) in table WEIGHT
or role (breeder, owner) in table UNIT, where we have distinct columns
in data. Entry and exit action are further examples. Here we can insert
whole new classes or needed values which will come later in routine
datachannels and are not present in historic data.

\begin{figure}[htbp]

\caption{\label{add_codes}Configuration section for S1\_add\_codes.pl}

\begin{minipage}{1.1\textwidth}
\begin{Verbatim}[fontsize=\scriptsize, frame=single, numbers=left ]
 ## class        ext_code        [ shortname longname description ]
 'SERVICE_TYPE' => {
                 'insem'      => ['insem', 'insemination (fresh semen)' ],
                 'nat'        => ['nat', 'natural (Natursprung)']
                   },
\end{Verbatim}
\end{minipage}                 

\end{figure}

In the example (figure \ref{add_codes}) a new code class
'SERVICE\_TYPE' would defined with the two external codes for
insemination and natural service.


\section{Step 2: Collecting and setting up CODES\label{codes}}

The steps 2 until 7 are made before the external animal
identifications would be collected, because also different codes from
different files (ex. sex from different sources) could be part of the
unique animal identification. 

This block performed by the programs
S2\_collect\_codes\index{collect\_codes1.pl} 
and S3\_load\_codes \index{collect\_codes2.pl}which need to get
adapted to the specific information system under consideration.

The configuration block in S2\_collect\_codes is given in table \ref{config_col_codes1}.
\begin{figure}[htbp]

\caption{Configuration block for S1\_collect\_codes\label{config_col_codes1}}

\begin{Verbatim}[fontsize=\scriptsize, frame=single, numbers=left ]
##############################################################################
# Begin configuration section:
##############################################################################
#                file           column           category
my @total = (
             [ 'herdbook.dat' , '0'  ,              'sex' ],
             [ 'herdbook.dat' , '9'  ,            'breed' ],
             [ 'herdbook.dat' , '20' ,              'mhs' ],
             [ 'station.dat'  , '0'  ,              'sex' ],
             [ 'station.dat'  , '2'  ,            'breed' ],
             [ 'station.dat'  , '32' , 'slaugter_remarks' ],
             [ 'station.dat'  , '16' ,          'l_cause' ],
             [ 'field.dat'    , '0'  ,              'sex' ],
             [ 'field.dat'    , '4'  ,            'breed' ],
             [ 'field.dat'    , '10' ,              'sex' ],
             [ 'litter.dat'   , '4'  ,            'breed' ],
             [ 'litter.dat'   , '8'  ,            'breed' ],
            );
##############################################################################
# End configuration section:
##############################################################################
\end{Verbatim}

\end{figure}
Each line configures one column with codes. The files are again those
that we had listed before. In {}``herdbook.dat'' we have 3 columns
with codes. They represent the sex of the animal, the breeds and the
mhs (maligne hypthermia) status of the animal. Notice, that we
need to specify \bf{all} occurences of codes.

%


In historic data often many different codes are used for the same
object. For instance for male animals we may find M, m and 1. These
would have to be translated into one interal code number. A general
procedure for loading CODES would follow these steps:

\begin{enumerate}
\item count the number of raw tables ($n_{tables}$)
\item for each table determine the columns that hold a code and give each
a category name (column class in CODES)
\item for each table:

\begin{enumerate}
\item column 1 do a select (code), count(code) group by code
\item column 2 do a select (code), count(code) group by code
\end{enumerate}
\item this will result in a table like this:


\begin{tabular}{|c|c|c|c|c|c|c|}
\hline 
\multicolumn{5}{|c|}{}&
\multicolumn{2}{c|}{Names}\tabularnewline
\hline 
RAW table&
column&
content&
n&
Target Code&
short&
long\tabularnewline
\hline
\hline 
herdbook&
sex&
1&
231&
&
&
\tabularnewline
\hline 
herdbook&
sex&
01&
212&
&
&
\tabularnewline
\hline 
herdbook&
sex&
m&
2121&
&
&
\tabularnewline
\hline 
herdbook&
sex&
f&
3212&
&
&
\tabularnewline
\hline 
...&
...&
...&
...&
&
&
\tabularnewline
\hline 
litter&
anomaly&
01&
2123&
&
&
\tabularnewline
\hline 
litter&
anomaly&
ok&
216&
&
&
\tabularnewline
\hline 
...&
...&
...&
...&
&
&
\tabularnewline
\hline
litter&
anomaly&
99&
3221&
&
&
\tabularnewline
\hline
\end{tabular}

\item write this table to a file
\end{enumerate}
The program collect\_codes1.pl produces as out the ASCII file codes.chg.
A few lines from this file are given in figure \ref{codes.chg}. This
file needs to get edited by the developer. 

%
\begin{figure}[htbp]

\caption{\label{codes.chg}Output file codes.chg from S2\_collect\_codes}

\begin{minipage}{1.1\textwidth}
\begin{Verbatim}[fontsize=\scriptsize, frame=single, numbers=left ]
# seperate columns with blancs  
# if you have more than one word in some columns you must insert these between >'< 
# for example: breed raw_herdbook 2  02  8724 DL 'German Landrace' ' another kind of Landrace'
# for undefined TARGET-CODE use NULL
# CATEGORY   TABLE  COLUMN CONTENT    NUMBER  TARGET-CODE   SHORTNAME LONGNAME DESCRIPTION
#------------------------------------------------------------------------
BREED      field.dat    '4'    '10'          1 '10'    
BREED      field.dat    '4'    '11'          1 '11'    
BREED      field.dat    '4'    '2'           1 '2'     
BREED      field.dat    '4'    'DL'     126067 'DL'    
BREED      field.dat    '4'    'LW'         27 'LW'    
BREED      field.dat    '4'    'PI'       1514 'PI'    
BREED      field.dat    '4'    'SH'       2369 'SH'    
BREED      herdbook.dat '9'    'DL'      27473 'DL'    
BREED      herdbook.dat '9'    'HA'         55 'HA'    
BREED      herdbook.dat '9'    'LW'        149 'LW'    
BREED      herdbook.dat '9'    'PI'       1384 'PI'    
BREED      herdbook.dat '9'    'Pi'          7 'Pi'    
\end{Verbatim}
\end{minipage}
\end{figure}



\section{Step 3: Manual edit of codes\label{edit_codes}}

\begin{enumerate}
\item manually, determine the target external code to be used in the system
(this will be done with your favorite editor). This table holds for
each category all the external codes allowed, e.g. category SEX may
be M, F, C for male, female and castrate. Table CODES would translate
SEX/F into an internal code used throughout the database. For unknown
codes it is possible to use 'NULL' in column target code. The table
will then look like this: 


\begin{tabular}{|c|c|c|c|c|}
\hline 
RAW table&
column&
content&
n&
Target Code\tabularnewline
\hline
\hline 
herdbook&
sex&
1&
231&
M\tabularnewline
\hline 
herdbook&
sex&
01&
212&
M\tabularnewline
\hline 
herdbook&
sex&
m&
2121&
M\tabularnewline
\hline 
herdbook&
sex&
f&
3212&
F\tabularnewline
\hline 
...&
...&
...&
...&
...\tabularnewline
\hline 
litter&
anomaly&
01&
2123&
1\tabularnewline
\hline 
litter&
anomaly&
ok&
216&
1\tabularnewline
\hline 
...&
...&
...&
...&
...\tabularnewline
\hline 
litter&
anomaly&
99&
3221&
9\tabularnewline
\hline
\end{tabular}

\end{enumerate}
Furthermore could be inserted the meaning of the additional columns
in table CODES, like short name, long name and description. These
meanings are filled only for one target code and could be also later
filled from other source. The resultant file with the name codes.chg.cvs
(which comes from cvs) is given in the figure \ref{codes.chg.cvs}.

%
\begin{figure}[htbp]


\caption{\label{codes.chg.cvs}Edited output file codes.chg}

\begin{minipage}{1.1\textwidth}
\begin{Verbatim}[fontsize=\scriptsize, frame=single, numbers=left ]
# seperate columns with blancs 
# if you have more than one word in some columns you must insert these between >'< 
# for example: breed raw_herdbook 2  02  8724 DL 'German Landrace' 'another kind of Landrace'
# for undefined TARGET-CODE use NULL
# CATEGORY   TABLE  COLUMN  CONTENT NUMBER TARGET-CODE SHORTNAME LONGNAME DESCRIPTION
#------------------------------------------------------------------------------------
BREED      field.dat    '4'    '10'      1 NULL      
BREED      field.dat    '4'    '11'      1 NULL      
BREED      field.dat    '4'    '2'       1 NULL      
BREED      field.dat    '4'    'DE'   1769 'DE' 'DE' 'German Large White' 'another kind of ...'  
BREED      field.dat    '4'    'DL'    140 'DL' 'DL' 'German Landrace' 'another kind of Landrace'  
BREED      field.dat    '4'    'LW'     27 'LW' 'LW' 'Large White'  
BREED      field.dat    '4'    'PI'   2551 'PI' 'PI' 'German Pietrain'  
BREED      herdbook.dat '9'    'DE'   1403 'DE'       
BREED      herdbook.dat '9'    'DL'    479 'DL'        
BREED      herdbook.dat '9'    'HA'     55 'HA' 'HA' 'Hampshire'  
BREED      herdbook.dat '9'    'SH'      4 'SH' 'SH' 'Schw�bisch H�llische'  
BREED      herdbook.dat '9'    'LW'    149 'LW'        
BREED      herdbook.dat '9'    'PI'   3438 'PI'        
BREED      herdbook.dat '9'    'Pi'      1 'PI'        
BREED      station.dat  '2'    '1'    3271 'DL'        
BREED      station.dat  '2'    '2'     983 'PI'        
BREED      station.dat  '2'    '3'      15 'DS'        
\end{Verbatim}
\end{minipage}
\end{figure}

In this example we have first some unknown codes inside which always
only happens once in the ascii file field.dat and let it out from the
database by changing the target to 'NULL'. The next lines leave the
targets as they are and add here some short and longname and a little
description which is needed only once for each target code. At the end
here are some
changes of target codes from 'Pi' to 'PI' and also for the data source
station from numbers to describing characters.

\section{Step 4: Load edited codes}

Run the program S3\_load\_codes \index{collect\_codes2.pl} which
reads the above edited file and loads it into table CODES.


\section{Step 5: Collect jobs}

The general procedure is the same as in collect codes, see above.
The differences result from the specific table structure for UNIT,
ADDRESS and NAMING. 


\section{Step 6: Manual edit of Jobs}

Here you can specify if you want different db-sequences for the same
person (unit) in different jobs. These are necessary when the same person
can be either breeder or owner. Then you need two different entrys
in UNIT but possible only one entry in ADDRESS and/or NAMING. As an
example see file job.chg.cvs and figure \ref{job.chg}.

%
\begin{figure}[htbp]

\caption{job.chg\label{job.chg}}

\begin{minipage}{1.1\textwidth}
\begin{Verbatim}[fontsize=\scriptsize, frame=single, numbers=left ]
# seperate columns with blancs 
# for undefined TARGET-CODE use NULL
# when  you need distinct sequence for db_name and db_address: see example
# below first 3 jobs get the same db_name but foster breeder get another db_address
# BREEDER   raw_herdbook   hb_zue_nr       208         21  208 
# OWNER     raw_herdbook   owner_nr        208         12  208 BREEDER BREEDER
# F_BREEDER raw_herdbook   hb_auf_nr       208          2  208 BREEDER ''
# SOCIETY   raw_herdbook   society_nr      208        102  208 
# the foster breeder below get the same db_name and db_address as breeder 208
# F_BREEDER raw_herdbook   hb_auf_nr       209          2  209 BREEDER(208) BREEDER(208)
# ext UNIT     TABLE    COLUMN  CONTENT  NUMBER  TARGET-CODE SAME(db_name) SAME(db_address)
#-----------------------------------------------------------------------------
BREEDER     herdbook.dat 18      57        1040  57
BREEDER     herdbook.dat 18      58         128  58
BREEDER     herdbook.dat 18      59           1  59
BREEDER     herdbook.dat 18      5           37  5
BREEDER     herdbook.dat 18      60           8  60
#-----------------------------------------------------------------------------
FBREEDER  herdbook.dat   19      570        171  570   BREEDER(57) BREEDER(57)
FBREEDER  herdbook.dat   19      57           1  570   BREEDER(57) BREEDER(57)
#-----------------------------------------------------------------------------
OWNER     herdbook.dat   11      0         1116  NULL 
OWNER     herdbook.dat   11      10144        2  10144 BREEDER BREEDER
OWNER     herdbook.dat   11      10570        1  10570 BREEDER BREEDER
OWNER     herdbook.dat   11      105         73  105   BREEDER BREEDER
#-----------------------------------------------------------------------------
SOCIETY   herdbook.dat   2       54           2  54   
SOCIETY   herdbook.dat   8       54           2  54   
SOCIETY   herdbook.dat   2       56          77  56   
SOCIETY   herdbook.dat   5       56          61  56   
SOCIETY   herdbook.dat   8       56          54  56   
\end{Verbatim}
\end{minipage}

\end{figure}

In the example the breeder get all unique sequences in table
ADDRESS and NAMING which will updated with the right informations
later. 

In the part foster breeder the two members get the same target as '570' and
then get the same db\_address and db\_naming as the breeder with the
external identification '57' (see the brackets at the end).

The next lines for the owner give then an empty (unknown) target for
the external id '0' and then the other owners get the same sequencies
as the identical numbered breeders. 

At the end the class society get then unique db\_address and db\_unit
for the identical target-code.

\section{Step 7: Load jobs}

Load all present units into the table UNIT and fill the right
sequences to tables NAMING and ADDRESS. 

\section{Step 8: Collecting all external animal identifications}

The step of creating external identifications (\ref{enu:create-external-identifications})
is performed by the program S6\_ignore\_animal\_id.pl\index{collect\_ext\_id.pl}
which resides in \$APIIS\_LOCAL/initial. Prior to being able to load historic
data external identifications in TRANSFER need to be created together
with their internal database numbers.

At this stage EXT\_ID is defined as a concatenation of fields which
is used in the historic dataset to make an identification unique.
This may be society$|$hb\_nr$|$sex in case of the reference. This same
string will be used to load history data into the database in the
following steps. 

This step is done by the program S6\_collect\_ignore\_animal. The program
is parameterized in a configuration block right at its beginning.
The example from the reference database is given in figure \ref{config-block}.

%
\begin{figure}[htbp]

\caption{\label{config-block}Configuration block of S6\_collect\_ignore\_animal}

\begin{minipage}{1.1\textwidth}
\begin{Verbatim}[fontsize=\scriptsize, frame=single, numbers=left ]
##############################################################################
# Begin configuration section:
##############################################################################
#              file             new_nr        old_nr
my @link =  ('herdbook.dat', '2 1 0'        '8 7 c2 12',
              'station.dat', 'cstation 11', '8 9 c2 10');

#              file              nr      unique / multiple
my @total = (
             [ 'herdbook.dat' , '2 1 0' ,      'u' ],
             [ 'herdbook.dat' , '8 7 c2 12' ,  'u' ],
             [ 'herdbook.dat' , '5 4 c1' ,     'm' ],
             [ 'herdbook.dat' , '8 7 c2' ,     'm' ],
             [ 'field.dat'    , '2 1 c2 3' ,   'u' ],
             [ 'field.dat'    , '2 1 c2' ,     'm' ],
             [ 'station.dat'  , 'cstation 11', 'u' ],
             [ 'station.dat'  , '8 9 c2 10' ,  'u' ],
             [ 'litter.dat'   , '2 1 c2' ,     'm' ],
             [ 'litter.dat'   , '6 7 c1' ,     'm' ],
            );

my @regex = (
              [ '\|0$', 'notch nr 0' ],
              [ '\|999999\|', 'inside animalnr 999999' ]
            );
###############################################################################
\end{Verbatim}
\end{minipage}

\end{figure}


Logically it breaks down into the following blocks:

\begin{enumerate}
\item create LINK: in pigs we have different numbers or IDs in the historic
datasets that refer to the same animal. This is usually the ID of
the piglets that is kept through testing. Only after selection this
animal gets a new number which is used for further reporting. Thus,
the two ext\_ids must refer to the same db\_id, which is created as
a counter. In this step two hashes are created from each link record,
one for each ext\_id holding the db\_id. \\
In line 5 those two sets of columns are given that refer to the animal
identification as young and as adult (selected) animals. The columns
chosen are those that make the IDs unique in the complete data set.
The first ID is made up of the columns 2, 1 and 0. This is herdbook
society, herdbook number and sex. The second ID is made up of the
dam herdbook society (column 8), the dam herdbook number (column 7),
the sex of the dam (fix 2) and the running piglet number (column 12).\\
Line 6 describe the link between mother number + notch number with
the internal station number which will refer to the information from
this datasource.
\item for each historic dataset each ext\_id (created as a concatenation
as described above) is read one by one performing the following actions:

\begin{enumerate}
\item does ext\_id exist in LINK(i)? if yes: use its DB\_ID
\item does EXT\_ID exist in TOTAL? if yes: increment COUNT (if UNIQUE print
error)\\
If it does not exist insert and add DB\_ID as an incremented counter
\end{enumerate}
\end{enumerate}
It is configured also in figure \ref{config-block} in lines 10 through
19. Notice the last column consisting of either 'u' or 'm'. This stands
for unique or multiple and means if the ID given should be unique
in the given file or if it can occur more than once. If the field
test, only one record is allowed for each animal, thus, if it occurs
more than once we know that there is a problem. Thus, the animal ID
as given in line 10 must occur only once. On the other hand, the animals
dam as given in line 12 will show up more than once. In this situation
nothing can really be checked.

Note, that ALL datafiles that constitute the historic data set must
be searched for external IDs and be configured in this section. This
clearly includes parents of animals that are tested at the station
and in the field (here, we sometimes have sires from different herdbook
societies which are not part of the breeding program (i.e. herdbook)
under consideration.

There are two problems that need attention at this point:

\begin{itemize}
\item unknown or undefined animals need to be identified. The run with the option
  '-i' create two files which should help you to identify such
  external animal identifications. 
\item then for some external IDs we know that they should occur only once. We
have indicated this with a 'u' in the @total vector \ref{config-block}.
If a this rule is violated we need to have means of resolving this
problem. This will always require manual interference. In the second
step collect\_ext\_id.pl (with option '-d') produces a file with duplicates
with the name of 'dup\_animal.chg'.
\end{itemize}

In the third configuration part (see figure \ref{config-block} at
@regexp) we have an aditional block which should 
help to find unknown or probably incorrect external animal identifications.
Here you can describe simple regular expressions that match the wanted
identifikations. At the end of this operation two files are generated. 

\begin{comment}
The hash TOTAL with elements: EXT\_ID and DB\_ID is then used to insert
records into transfer.
\end{comment}

\section{Step 9: Manual edit }

The file 'ignore\_animal.chg' (see figure \ref{ignore_animal.chg}),
craeted with option '-i' on 'collect\_ext\_id.pl', contain only little
statistic which could help to identify external animals where usually
used as unknown animals. 
The idea behind the creating of this file was that animal
identifications which are only used as dummy number to create a full
pedigree or can insert in the original software only animals which
have parents but you don't know them, than the number of occurrence
are normaly much higher as the normal use of an animal.

To really ignore further these animals you must write these
identifications into the file 'ignore\_animals.ok' or use option '-c
filename' for this file and leave only animals there which should
be ignored. (see file 'ignore\_animal.ok' in directory \$APIIS\_LOCAL/initial
for an example) If you need some more informations about the exact
number of use the external animal identification in each input file
you can use the option '-x' which create a file 'detailed\_use' with
this information.%

\begin{figure}[htbp]

\caption{ignore\_animal.chg\label{ignore_animal.chg}}

\begin{minipage}{1.1\textwidth}
\begin{Verbatim}[fontsize=\scriptsize, frame=single, numbers=left ]
# leave only animals which schould be ignored
# and rename to ignored_animals.ok or use the option c
# ext_animal count
32|400080|1  2563
0|9999999|0  2001
32|400061|1  1686
32|100048|1  1663
32|400167|1  1655
32|253020|1  1605
    :
0|0|1  591
0|0|2  586
0|0|2|0  577
\end{Verbatim}
\end{minipage}

\end{figure}

In this example seems to be that the first animal are an real one and
if we see this was an boar and could be came as sire so often. But
then we can identify such a number wich is probably not a valid
identification of an living animal. unfortunateley in our case we
found other invalid IDs later on the file (0$|$0$|$1 as unknown sire...).

The second file 'ignore\_animal2.chg' cover the identifications from
the defined regular expressions and also the animal identifications
which has uninitialised values inside the concatenation. An example
see in figure \ref{cap:ignore_animal2.chg}. Now you have to decide
which animals are used as unknown or if something wrong with the id.
Also these identifikations has to be added into file
'ignore\_animal.ok'. mostly it would be better to copy these file to
the ignore\_animal.ok and add the other identifications found in
ignore\_animal.chg.

%
\begin{figure}[htbp]

\caption{ignore\_animal2.chg\label{cap:ignore_animal2.chg}}

\begin{minipage}{1.1\textwidth}
\begin{Verbatim}[fontsize=\scriptsize, frame=single, numbers=left ]
# write the animals which schould be ignored
# to file ignored_animals.ok or use the option c with collect_ext_id.pl
# ext_animal  description of reason
32|999999|1  inside 999999  
25|67682|2|0  notch_nr 0  
32|104172|2|0  notch_nr 0  
32|101338|2|0  snotch_nr 0  
  :
||2|  NULL values
||2  NULL values  
\end{Verbatim}
\end{minipage}

\end{figure}

The example show all external identifications which match the given
regular expressions in collect\_ext\_id.pl and additional all
concatenated external IDs where not all elements are known. 

\section{Step 10: Find duplicates}

Use of 'S7\_collect\_dup\_animal' create a output file with
the data of duplicated animals, if defined that these animals schould
be unique in the raw datafile. (see file 'dup\_animal.chg') To define ignored
animals and find duplicates are two steps because animals to be ignored
also could be duplicated...

\section{Step 11: Manual edit of duplicates}

In this step you can change the external animal identification in
dependence of the whole data for these animal (if you know which external
identification is wrong). If the lines are total identically one of
the line will be accepted because there are no other information for
the animal.

As example see file 'dup\_animal.chg.cvs'
(few lines are in figure \ref{dupl1}), where only in the case of
station data it is possible to change the identification. All other
animals in this file will further be ignored in loading data, because
nobody know if the external identification is ok. 

%\begin{SaveVerbatim}[fontsize=\scriptsize, frame=single, numbers=left, commandchars=\\\{\}]{dupanimal}
%#  change only ext_animal                          
%#  for example 32|400723|2|15 =>  32|400723|2|16   
%# file         ext_animal      key            line 
%     :
%station.dat = 32|133575|2|\bf{57} ( 8 9 c2 10 )  =>  
%   ...|133575|57|202349|110|12.40|8.9.99 00:00:00|||30.40|...
%station.dat = 32|133575|2|\bf{58} ( 8 9 c2 10 )  =>
%   ...|133575|57|202355|110|10.60|8.9.99 00:00:00|30.11.99 00:00:00|A9|28.20|...
%\end{SaveVerbatim}
%\fbox{\BUseVerbatim{dupanimal}}

%
\begin{figure}[htbp]

\caption{\label{dupl1}example for changed duplicates}

\begin{minipage}{1.1\textwidth}
\begin{Verbatim}[fontsize=\scriptsize, frame=single, numbers=left, commandchars=\\\{\} ]
#  change only ext_animal                          
#  for example 32|400723|2|15 =>  32|400723|2|16   
# file         ext_animal      key            line 
     :
station.dat = 32|133575|2|{\bf57} ( 8 9 c2 10 )  =>  
   ...|133575|57|202349|110|12.40|8.9.99 00:00:00|||30.40|...
station.dat = 32|133575|2|{\bf58} ( 8 9 c2 10 )  =>
   ...|133575|57|202355|110|10.60|8.9.99 00:00:00|30.11.99 00:00:00|A9|28.20|...
\end{Verbatim}
\end{minipage}

\end{figure}

The example in figure \ref{dupl1} show the changing of a notchnumber
from 57 to 58 for the animal wich would be slaughtered (this
information are on the end of the printed line). this means that the
used boar have the correct notchnumber 57. both lines will be
represented in the database.

\section{Step 12: Load external identifications}

This step (S8\_load\_ext\_aniomal) really load the table TRANSFER with
all external identifications 
and create the proper db\_animal. Not inserted are the identifications
from the file 'ignore\_animal.ok' and the not edited animals from file
'dup\_animal.chg.cvs'.

\textbf{After this, the historic data can be loaded as all verified
codes and units will be in the represent tables.}


\section{Step 13: write all animals to table animal}

The script S9\_transfer\_to\_animal write all animal numbers into table
ANIMAL. The reason is to have, after loading all external animals
into table TRANSFER, you need to have all existing internal animal
numbers also in table ANIMAL. After this step all needed access to
this table could only be updates and no inserts anymore. With this
step you guarantee also that all animals reside in this table and
go down to unknown parents. Later all known parents overwrite this
behavior.

\section{Loading Data\label{load_data_hist}}

The above steps ensure that all data channels are open and that each
external identification is connected to the correct internal database
number. The same holds true for all codes that are to be loaded. Also
here, we have an external representation and an internal equivalent.
In the following steps the tables get filled with information from
the various data files. Clearly, these programs require a lot of adaptation,
since the structure and content of the datafiles will be very different
form one species to the other.

The order of loading the data are verry important which data have the
better meaning if the information come in twice from different
sources. For example use the sex from the herdbook data and only if
there not filled here use other sources.

The configuration sections for each source file are located in a
seperated <name>.par file an will be run as {\it S10\_load\_data('-p <name>.par')}. 

\subsection{Step 14: Loading Herdbook Records}

As an example the parameter file for the loading program (herdbook) is
shown in figures \ref{load_data_herdbook} until
\ref{load_data_herdbook3}. 
The parametrisation require three blocks, first some informations
about the included files, second the definition of animal
identification (also sire and dam) and the definition of traits, and
third the sql-section where defined the actions will taken with the
data.  All information from the source files are located by the
position in it.

In the second section the names \texttt{}ext\_animal, ext\_sire and
ext\_dam are mandatory if exist and should have the same information
as given in S6\_collect\_ignore\_animal. Then follow the definition of the
traits. The function get\_value has in mind if the value is a code or
a unit (defined in S2\_collect\_codes respective S4\_collect\_jobs)
and/or have to be redefined. The last parameter of the function could
be an dateformat 
which have to splitted in the right order or an check for numeric values or upper /
lower case any character (for details see pod from load\_data.pl). If
is no value on this position then the return 
value is 'NULL' (in database sense = absence), because also no values
are allowed (ex: slaughter findings only some time appear).

The last section with the nativ sql-syntax has three
possible parts:

\begin{enumerate}
\item insert,
\item update and
\item any sql which should be executed only one time after all others.
\end{enumerate}

The reason for this is to make the inserts much more quickly, if the
indices are droped and recreated later after this step. On the other
side, for updates it is an advantage if indices exists. The use of
the third part is in this example to drop all records from table GENES
which have no mhs-status because mhs can also be 'NULL'. 
% (you can also do this when adding a where clause to the insert sql) %?

\begin{figure}[htbp]

\caption{first section from herdbook.par\label{load_data_herdbook}}

\begin{minipage}{1.1\textwidth}
\begin{Verbatim}[fontsize=\scriptsize, frame=single, numbers=left ]
#startconf
##############################################################################
# Begin configuration section:
################################### EDIT -> ##################################
$prg_name = 'herdbook';
$infile = 'herdbook.dat';
$sep = '\|';      # Delimiter/separator for input file (pipe | must be masked!)
$chg_file = 'dup_animal.chg.cvs'; # chg_file for duplicated animals
$file_cod = 'codes.chg.cvs';      # chg_file for codes or option -f
$file_job = 'job.chg.cvs';        # chg_file for jobs or option -j
#$file_id = 'keys.chg.cvs';       # chg_file for keys or option -i
################################### EDIT <- ##################################
#endconf
\end{Verbatim}
\end{minipage}

\end{figure}

The first section describe the names of the inputfiles wich should be
used and probably the used delimiter for the raw datafile.


\begin{figure}[htbp]

\caption{second section from herdbook.par\label{load_data_herdbook2}}

\begin{minipage}{1.1\textwidth}
\begin{Verbatim}[fontsize=\scriptsize, frame=single, numbers=left ]
#startval
################################### EDIT -> ##################################
$ext_animal = $line[2] . '|' . $line[1] . '|' . $line[0];
$ext_sire   = $line[5] . '|' . $line[4] . '|' . '1';
$ext_dam    = $line[8] . '|' . $line[7] . '|' . '2';
$sex        = get_value( 0 , \@line );
$breed      = get_value( 9 , \@line );
$litter     = get_value( 13, \@line );
$teats_l_no = get_value( 16, \@line, 'n' );
$teats_r_no = get_value( 17, \@line, 'n' );
$mhs        = get_value( 20, \@line );
$exterior   = get_value( 23, \@line, 'n' );
$birth_dt   = get_value( 14, \@line, 'yyyymmdd' );
$name       = get_value( 10, \@line );
$breeder    = get_value( 18, \@line );
$f_breeder  = get_value( 19, \@line );
$society    = get_value( 2 , \@line );

$soc_fix = get_db_unit( '32', 'society' ); # where isnull
################################### EDIT <- ##################################
#endval
\end{Verbatim}
\end{minipage}

\end{figure}

In this section we have three blocks:

\begin{itemize}
\item animal / sire / dam identification
\item value definition and 
\item some further data manipulation (further example see field.par)
\end{itemize}

The animal identification have the same informations as described in
collect\_ext\_id.pl. Birth date has in our case an content like
'19981207' and wold be described by 'yyyymmdd'. The leading 'n' as
option for get\_value check the content for the teats count that
they are real numbers. Else the content of the field are ignored.

The last line allow to get or manipulate some data. in this example we
need the correct internal db\_unit for the combination external unit and
external id. The reason are that the society would only given if this is
not the own one.

\begin{figure}[htbp]

\caption{third section from herdbook.par\label{load_data_herdbook3}}

\begin{minipage}{1.1\textwidth}
\begin{Verbatim}[fontsize=\scriptsize, frame=single, numbers=left ]
#startsql
@sql_insert=(); 
################################### EDIT -> ##################################
$sql_insert[0]="INSERT INTO genes 
      ( db_animal, db_mhs, last_change_dt, last_change_user )
      VALUES ( $db_animal, '$mhs', '$now', '$user')";
$sql_insert[1]="INSERT INTO exterior
      ( db_animal, teats_l_no, teats_r_no, exterior, last_change_dt,
      last_change_user )
      VALUES ( $db_animal, '$teats_l_no', $teats_r_no, $exterior, '$now', '$user' )";
##################
 @sql_update=();
##################
$sql_update[0]="UPDATE animal SET db_sire = $db_sire, db_dam = $db_dam,
     db_sex = $sex, db_breed = $breed, parity = $litter,
     name = '$name', birth_dt = $birth_dt, db_society = $society,
     db_breeder = $breeder, db_foster_breeder = $f_breeder
  WHERE db_animal = $db_animal";
$sql_update[1]="UPDATE animal SET db_sex = $id_code{ 'SEX' . '#'
  . '1'}
  WHERE db_animal = $db_sire AND db_sex isnull";
$sql_update[2]="UPDATE animal SET db_sex = $id_code{ 'SEX' . '#'
  . '2'}
  WHERE db_animal = $db_dam AND db_sex isnull";
##################
@sql_only_one=();               #for use of sql statement which only one time executed
$sql_only_one[0]="DELETE FROM genes WHERE db_mhs isnull";
$sql_only_one[1]="UPDATE animal SET db_society = $soc_fix where db_society isnull";
## you must "'" the values wich are strings or dates ?? but you can do this for all...
################################### EDIT <- ##################################
#endsql
\end{Verbatim}
\end{minipage}

\end{figure}

The last section describe the native SQL-statements to load the data
in the tables. Inserts are made for tables GENES and EXTERIEUR. 
The variables $now and $user could be used to fill the
mandatory fields last\_change\_user and last\_change\_dt. 

Remember that the filling of table ANIMAL are always have to be
updates after loading table animal via transfer\_to\_animal.pl.
Therefore here we update that table. The next two are examples for
conditional updates only if the sex of the given animal not known. Here
we have also an example for using codes which are not comming from the
data. The string \verb+$id_code{ 'SEX' . '#' . '1'}+ give back  the db\_sex
where the coding class are 'SEX' and the value are '1'. This
information we have from the use of the animal either as sire or dam.

Last we delete all animals from table genes which not have any
information about the mhs status and give to all animals which not
have an society information the right one from the own society (which
is the default)

The further data loading programs have the same structure. There will
only describe some usefull structures which there are different from herdbook.par.

\subsection{Step 15: Loading Station Records}

Here we give an example (figure \ref{ifloops} ) for using 'if loops' to
execute some SQL-statements only if the values are defined. Therefore
we have also to use the push method to fill the array with the 
SQL-statements because we could not define the array element number as
fix. Remember that the function get\_value return 'NULL' (database
empty) if the value are not defined. The delete statement in
@sql\_only\_one (compare also herdbook.par) would then not be used here.

\begin{figure}[htbp]

\caption{third section from herdbook.par\label{ifloops}}

\begin{minipage}{1.1\textwidth}
\begin{Verbatim}[fontsize=\scriptsize, frame=single, numbers=left ]
if ( $test_wt ne 'NULL' and $test_dt ne 'NULL' ) {
  push @sql_insert, "INSERT INTO weight
    ( db_animal, test_dt, test_wt, db_weight_type, last_change_dt, last_change_user )
   VALUES
    ( $db_animal, $buy_dt, $buy_wt,
     '$id_code{ 'WEIGHT_TYPE' . '#' . 'buy_s'}', '$now', '$user')";
}
\end{Verbatim}
\end{minipage}

\end{figure}

\subsection{Step 16: Loading Fieldtest Records}

A new not necessary parameter block are inserted here which will be
run once after database access is created. This could be used if
allways inserted informations are needed for 
the next steps. The example would be shown in figure
\ref{perlrun}. The use of the new informations are used in runing
one insert only if some informations not always was inserted until
here. The problem in the reference data are, that this information
(number of teats) could either came from the herdbook and / or from
the field data.  

\begin{figure}[htbp]
\caption{block perlrun from parameterfile field.par\label{perlrun}}

\begin{minipage}{1.1\textwidth}
\begin{Verbatim}[fontsize=\scriptsize, frame=single, numbers=left ]
#startperl
################################### EDIT -> ##################################
# block have to be run once after db access
##############################################################################
  :
  my $sqltext = "SELECT db_animal, teats_l_no FROM exterior";
  my $sql_ref = $apiis->DataBase->sys_sql($sqltext);
  $apiis->check_status;
  while ( my $ret = $sql_ref->handle->fetch ) {
  :
################################### EDIT <- ##################################
#endperl
\end{Verbatim}
\end{minipage}
\end{figure}

The code in figure \ref{perlrun} have to be only valid perl so the
cryptic example would not be explained.

Further here we found an example for calculating birth date from given testing
date and age of animal (in section values). 

\subsection{Step 17: Loading Litter Records and Step 18: Loading
  Address Records}

nothing special...


%\subsection{Step 18: Loading Address Records}


\section{Step 19: Clear Index}

This script clear such tables which couldn't create unique indices
after loading historic data. This is required because indices are
droped when inserting data there and recreated afterwards. The most
possible reason are if there are the same information come from
different sources.

However there are something wrong with the data. In the reference
we get duplicated records with the same service date for the same
animal. For example in the output you get the message \texttt{\small 'Cannot
create unique index. Table contains...'}, then you can specify the
table, the unique index and the indexname like in figure \ref{clear_index.pl}.%
\begin{figure}[htbp]
\caption{EDIT section in S11\_clear\_index\label{clear_index.pl}}

\begin{minipage}{1.1\textwidth}
\begin{Verbatim}[fontsize=\scriptsize, frame=single, numbers=left ]
################ E D I T ########################################
#                   table ;     unique index     ;   indexname ; where clause
$combination[0] = "service; db_animal, service_dt; uidx_service_2";
# $combination[2] = "weight; db_animal; uidx_weight_9; db_weight_type = 
#   (select db_code from codes where ext_code = 'weaning' and class = 'WEIGHT_TYPE')";
################ E N D E  - E D I T #############################
\end{Verbatim}
\end{minipage}

\end{figure}
 
The programm delete the further oids after the first one - if equal
combinations exists. This means that you can use a hierarchical order
which element should be saved if is one.

You can also use it for cleaning with not unique indices via using the
where clause. The example means that we want to check if there in table weight
more than one weaning weight is included (because only one weaning
weight (see weight\_type) are possible).


\section{Houskeeping}

So far merely data has been loaded into the correct structure. However,
external data channels have been defined for convenience of the loading
process but not as external ids as they will reach the database. Furthermore,
dates in transfer will have to be set.


\subsection{Step 20: Complete transfer with dates from
  Herdbook\label{dates_to_transfer_hb.pl}} 

Complete table TRANSFER with information from file 'herdbook.dat'.
Create proper opening\_dt, entry\_action... for all animals in TRANSFER.
This step is in the reference db a little difficult, because you have
to regard a lot of dates in 'herdbook.dat' which can be filled in
every combination or not. Concrete the following information have
to go to TRANSFER:

\begin{enumerate}
\item breeder
\item owner
\item foster breeder
\item birth date
\item registration date
\item leave date
\item buy date
\end{enumerate}
Therefor you must describe the sequence and the dependencies for each
possible combination manually. For example opening date: first birth
date if exists then buy date if come from other society, then registration
... Also this step require inserts in table TRANSFER because also
buying, renaming (registration) and so on should be reflected here.
:-(


\subsection{Step 21: Complete transfer with dates from Station}

Same procedure as in section \ref{dates_to_transfer_hb.pl} for informations
comming from station.dat but rather simpler. :-)


\subsection{Step 22: Fill last action}

Fill the last reproduction actions in table ANIMAL if possible. Get
simply the ordered actions from the involved tables (service, litter,
animal) and fill the newest one to la\_rep.


\subsection{Step 23: Rewriting TRANSFER}

In the previous step EXT\_ID was defined to make animal ids consistent
over the complete range of years in the historic dataset. They are
often a concatenation of various kinds. Now that all historic data
have been loaded we need to rewrite the EXT\_ID and UNIT to a format
that matches the incoming routine data streams. If there is not an
living database required you have also to rewrite the db\_unit but you
can skip the steps for splitting the concatination.

% The rewriting needs
%to be done only for those animals for which we shall get routine information.
%Thus, really old animals do not need to be rewritten, instead we can
%leave the ext\_ids as they are. 

The program post\_initial.pl does the job under consideration. There
are much specific code, because its verrry different which ids are
needed for incomming datastreams. Also the file codes\_unit.chg will
be created for the new mandatory units to make the animal ids unique
in each reporting unit. The program fill the related tables NAMING
and ADDRESS with the correct sequences.

In general there are three parts to edit in the script:
\begin{enumerate}
\item some global informations, the requirements of split (if use
  APIIS not for further incomming data)  and the unit informations for the
  base parents.
\item create an sql-select which animals have to be rewritten. Here it
  is possible to let some IDs uncahanged and if required drop the
  unique index of table TRANSFER so far (then have a look at
  clear\_transfer.pl).
\item a loop where desribe the roles under which circumstances what
  should happen with the external identification. Mainly here describe
  the new external animal ID and the parts of unit.
\end{enumerate}

For configuration block three in figure \ref{post_ini} we have the
concret example from post\_initial.pl.
\begin{figure}[htbp]
\caption{third EDIT section on post\_initial.pl\label{post_ini}}

\begin{minipage}{1.1\textwidth}
\begin{Verbatim}[fontsize=\scriptsize, frame=single, numbers=left ]
    ########### EDIT ##################################################
    # split here:  EDIT  this loop
    # 32|0815|2|16 => ext_animal 0815|16 (1 3|1) ext_unit 32|2 (0 2)
    # $line represents the concatenated animal identification
    # specific for each case!!
    my @lline = split ( /\|/, $line[0] ); # split the elements
    # station numbers
    if ( $lline[0] =~ /station/ ) { # station|56789
      #  ea == ext_animal, ei == ext_id, eu == ext_unit
      my $ea = $lline[1];
      my $ei = 's57';
      my $eu = 'station32';
      $new_unit{ $eu . $ext_sep . $ei } ++;
      $new_id{ $line[0] } = [$ea, $ei, $eu];
    # herdbooknumbers
    # three elements means with notch number
    } elsif ( $#lline == 3 ) { # 32|0815|2|16
      my $ea = $lline[1] . '|' . $lline[3];
      my $ei = $lline[0] . '|' . $lline[2];
      my $eu = 'society|sex';
      $new_unit{ $eu . $ext_sep . $ei } ++;
      $new_id{ $line[0] } = [$ea, $ei, $eu];
    # here without notch number
    } else { # 32|0815|2
      my $ea = $lline[1];
      my $ei = $lline[0] . '|' . $lline[2];
      my $eu = 'society|sex';
      $new_unit{ $eu . $ext_sep . $ei } ++;
      $new_id{ $line[0] } = [$ea, $ei, $eu];
    }
  }
  ########### END EDIT ###############################################
\end{Verbatim}
\end{minipage}

\end{figure}
In this example we have the three different numbering systems: station
numbers and herdbook numbers either with or without notch number. 

The roles for spliting are:
\begin{enumerate}
\item the word 'station are inside the concatenated number
\item else after splitting the concatenated number we have
  \begin{enumerate}
  \item three data elements which means a notch number is added or
  \item all the rest. Then we have less than three elements an this
    mean it is an real herdbooknumber together with sex and society.
  \end{enumerate}
\end{enumerate}

The target numbers are than:
\begin{enumerate}
\item station numbers
  \begin{enumerate}
  \item incoming animal number are the station number (position 1)
  \item unit are 's57' as ext\_id and ext\_unit is 'station32'
  \end{enumerate}
\item mothernumber plus notch
  \begin{enumerate}
  \item incoming animal are mothernumber plus ($|$) notch number
  \item unit are society plus sex (position 0 and 2 from the splitted
    data) and ext\_unit are 'society$|$sex'
  \end{enumerate}
\item  real herdbooknumber
  \begin{enumerate}
  \item incoming animal are herdbook number
  \item unit are society plus sex (position 0 and 2 from the splitted
    data) and ext\_unit are 'society$|$sex'
  \end{enumerate}
\end{enumerate}

The differences between the last two splits are than only the number
of elements and an existing pipe sign inside the external identification.
This information also could be later used to return the right external id
to the right people (real herdbooknumber if exist else mothernumber
plus notch or station numbers if the station have an request...).

\subsection{Step 24: Manual edit }

These step add the meaning (shortcode, long code, description) of
the new units to table CODES like described in section \ref{edit_codes}.

\subsection{Step 25: Insert new codes from post\_initial}

Fill in the new codes really.

\subsection{Step 26: Close some datachannels}

Close some datachannels if there are now duplicated combinations for
ext\_animal and db\_unit. For example if the split of an on farm used number has
deleted the birth year and would be reused than, now we have the same
external number more than one time and this require a close of the older chanel. 
For this purpose the script clear\_transfer.pl could be used. The
configuration section are described in figure \ref{clear_transfer}.

\begin{figure}[htbp]
\caption{configuration section from script clear\_transfer.pl\label{clear_transfer}}

\begin{minipage}{1.1\textwidth}
\begin{Verbatim}[fontsize=\scriptsize, frame=single, numbers=left ]
################ E D I T ########################################
# where is the used date_information
#                table , date
$used_date[0] = "animal, birth_dt";
################ E N D E  - E D I T #############################
\end{Verbatim}
\end{minipage}

\end{figure}
This script close all datachannels with more than one ext\_animal
db\_unit combination in depense of one date. Here would be used the
birth date from table animal. 

\section{Consistency checking against the business rules}

So far, data has been loaded outside the business rules.

\begin{comment}
Checking TRANSFER\index{ENTRY} for Consistency\index{Consistency}

A Perl program is available for consistency checks on TRANSFER. It
is located in references/reports/check\_TRANSFER.pl. It should run
on each database (there is no need to have a different structure for
TRANSFER) after TRANSFER has been populated. Particular focus should
be on the duplicates in EXT\_ID\index{ID}. In general, the user should
ask herself if the statistics generated are acceptable. %
\marginpar{not further needed because the strategie has changed!%
}
\end{comment}

\subsection{Checking the database content against the business rules}

During the loading process the APIIS busines rule system has not been
in effect. Thus, data will have been loaded that does not comply with
the business rules specified in the model file. The program check\_integrity
\index{verify\_integrity} does this job. 

To this effect each table is accessed in turn then all rows are read
one by one and validated against the model file business rules. Errors
are logged in an ascii file with the name of the table used as file
name and {}``.err'' appended.

One time you must run this programm to fill the mandatory column 'dirty'
in each table. Set 'true' if one busines rule is violated. 


\subsection{Treatment of errors}

It should be the ultimate objective to have no data in the database
that does not comply to the business rules. However, for practical
reason, this status may be unachievable for a number of records. One
reason will be that the original records are no longer available for
correction. On the other hand deleting a record with only some parts
being incorrect may also not be an option -- other information will
be lost. Thus, if the corresponding flag has been set the status column
{}``dirty'' will be set to error for those records not complying
with the business rules. These records can then be skipped on later
processing. Furthermore, at a later stage after loading of historic
data the database can be clean sucessively. The type of error can
be regenerated by {}``verify\_integrity'' for individual records.


\subsection{Fixing errors}

Example:

\begin{itemize}
\item check\_integrity -f model -o 'table date from' \\
(check\_integrity -f ../model/apiis.model -o 'animal birth\_dt 1998-01-01')
\item edit file 'animal.birth\_dt\_since\_1998-01-01.errors' (see '\textasciitilde{}.cvs'
and add in the new column the correct values)
\item update\_from\_error\_file.pl -f animal.birth\_dt\_since\_1998-01-01.errors
\end{itemize}

