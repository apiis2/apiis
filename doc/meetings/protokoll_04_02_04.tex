\documentclass[12pt,a4paper,DIV12]{scrartcl}
\usepackage{german}
\usepackage{umlaut}
\usepackage{longtable}
\usepackage{fancyvrb}

\begin{document}
\begin{center}{ \Large \bf Protokoll der APIIS-Entwicklerzusammenkunft\\ 4.~und
  5.~Februar 2004 in K�llitsch }
\end{center}

\section{Teilnehmer}
\begin{itemize}
\item Eildert Groeneveld (eg)
\item Helmut Lichtenberg (hl)
\item Hartmut B�rner (hb)
\item Detlef Schulze (ds)
\item Zhivko Duchev (zd)
\item Marek Imialek (mi)
\item Ulf M�ller (um)
\item Ralf Fischer (rf)
\end{itemize}

\section{Ergebnisse}
\begin{enumerate}
\item alle CVS Dateien werden mit einem 'Tag' f�r die
  Versionskontrolle versehen
\begin{itemize}
\item hl setzt diese und erstellt ein Beispiel f�r die Nutzung
\item rf kl�rt das Auschecken und Pflegen f�r die MINIPIGS (G�ttingen)
\end{itemize}
\item[$\Rightarrow$] hat sich inzwischen erledigt, die neuen Kernstrukturen werden in einem
neuen Modul (apiis) abgebildet, pdbl bleibt so wie bisher
\item Umstellung interner Strukturen -- hb, um und rf machen einen Vorschlag f�r
  die zus�tzliche Anforderungen an Input/Output Objekte basierend auf
  den Strukturen von Recordset (hl), dabei sollen auch beliebige
  SQL-Statements m�glich sein 
\item Dokumentation
\begin{itemize}
\item die Quellprograme und Libraries werden am Ort mittels POD dokumentiert
\item hl erstellt ein Makefile welches diese Informationen
  zusammensucht und daraus eine Dokumentation erstellt
\item die Anwenderdokumentation ist zu Modularisieren und mittels
  \verb/\input/ in einem \LaTeX--Dokument zusammenzufassen
\item ds macht hierf�r einen Vorschlag f�r eine generische
  Dokumentstruktur und passt die bisherige Doku daran an
\item innerhalb der doc-Pfade in jedem Projekt sind die
  Unterverzeichnisse 'developer', 'implementer' und 'user' zu
  erstellen und zu nutzen
\end{itemize}
\item keine \verb/die/ und \verb/print/ Statements in Programmen
  welche auch in Verbindung mit HTML genutzt wefrden sollen
  (z.\,B.~pdf-Reports)
\item zd stellt den Patch f�r DBD bereit (f�r K�llitsch)
\item f�r die Graphische Darstellung von Abl�ufen u.\,�.~ist auf die
  Mithilfe von Jutta Moosdorf zur�ckzugreifen, daf�r sind
  aussagekr�ftige Skizzen zu erstellen
\end{enumerate}

\section{Zus�tze}
\begin{enumerate}
\item eg
\begin{itemize}
\item ds �bernimmt die Koordinierung und �berwachung der Doku als
  'documentation officer'
\item bei Umstellung interner Strukturen m�ssen die jetzigen
  'Implementer Interfaces' in Forms und LO weiter m�glich sein
\end{itemize}
\end{enumerate}

\end{document}





