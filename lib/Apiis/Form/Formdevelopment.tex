%%% $Id: Formdevelopment.tex,v 1.3 2005/12/29 08:16:49 heli Exp $
\documentclass[a4paper,11pt]{scrartcl}
\usepackage[T1]{fontenc}
\title{Hints for the development of \\ \apiis XML Forms} 
\author{Helmut Lichtenberg}
\date{CVS $Revision: 1.3 $}
\parindent0pt \setlength{\parskip}{.8ex plus .1em minus .1em}
\tolerance=5000
\newcommand{\apiis}{\mbox{\textsc{Apiis} }}

\begin{document}
\maketitle
% \tableofcontents
\thispagestyle{empty}

The configuration in the new \apiis forms development is done with the
eXtended Markup Language (XML). The main rules for these XML-files are
defined in \$APIIS\_HOME/etc/form.dtd.

General guidelines are:
\begin{itemize}
   \item All important elements of the XML file must have a unique Name.
         This is enforced by the dtd-entry:

         \verb+   Name     ID      #REQUIRED+

         This unique Name is the base of accessing the element values.

   \item XML elements without a Name attribute are broken down into their
         parent elements. Their attributes become attributes of the parent
         element. This setup is chosen to hide complex definitions (e.g.
         for color and text attributes).
\end{itemize}

After parsing the form configuration file through Perl's libxml module, an
own datastructure is built.

All attributes and auxiliary data for a block with the name \verb+Block_0+
can be accessed via its name. The main method for this is

\verb+   $form_object->GetValue( <element name>, <attribute name> );+

Example:

\verb+   my $description = $form_object->GetValue( 'Block_0', 'Description' );+

This will give you the describing text about this block.

In addition to the XML attributes, some auxiliary data structures for
development purposes are created. Some of these structures, which are often
used, got their own methods to access them (e.g. \verb+$self->top+), but
most of them are made accessible via the method \verb+GetValue+.

For the method names see the POD docs.
The structural elements, you get via \verb+GetValue+ are described here:


\begin{itemize}
   \item Global Items
      \begin{description}
         \item[Type] Some pseudo elements (pseudo, because they have no Name
            attribute and become flatted into their parent element)
            determine the \verb+Type+ of the element.
            
            DataSource can be of Type \verb+Record+, \verb+Sql+,
            \verb+Function+, \verb+none+.

            Column can be of Type \verb+DB+, \verb+Related+,
            \verb+HashRef+, \verb+ArrayRef+.
            
            Field can be of Type \verb+FileField+, \verb+Button+,
            \verb+Link+, \verb+ScrollingList+, \verb+BrowseEntry+,
            \verb+PopupMenue+, \verb+TextField+, \verb+TextBlock+,
            \verb+RadioGroup+, \verb+CheckBoxGroup+, \verb+CheckBox+,
            \verb+Calendar+.

            They can be accessed by

            \verb+my $type = $self->GetValue( $element, 'Type');+

            All other element are of the Type, how this element is named.
            A Block is of Type \verb+Block+, a Frame of
            Type \verb+Frame+, etc.
         \item[\_parent]
            You can get the parent of each element (except the root
            element) with:

            \verb+my $datasource = $self->GetValue( $column, '_parent');+

      \end{description}

   \item Block Element
      \begin{description}
         \item[\_datasource\_list]
            As there could be only one \verb+DataSource+ on block level,
            you only need the first element of this list of block datasources.

            \verb+my ($ds_name) =+ \\
            \verb+   @{ $self->GetValue( $blockname, '_datasource_list') };+

         \item[\_field\_list]
            Get all \verb+Field+ elements of this block:

            \verb+my @fields = @{ $self->GetValue( $blockname, '_field_list') };+

         \item[\_misc\_blockelement\_list]
            Other non-\verb+Field+ elements of this block (like
            \verb+Label+, \verb+Tabular+, etc.) can be retrieved by:

            \verb+my @misc_fields =+ \\
            \verb+   @{ $self->GetValue( $blockname, '_misc_blockelement_list') };+

         \item[\_all\_field\_list]
            \_field\_list and \_misc\_blockelement\_list entries are summed
            up together in:

            \verb+my @all_fields =+ \\
            \verb+   @{ $self->GetValue( $blockname, '_all_field_list') };+

            \_all\_field\_list also contains those fields and elements,
            which aren't directly on the block level, but deeper down in
            the hierarchy like in a \verb+Tabular+ or \verb+Frame+ element.

            XML Example:
            \begin{verbatim}
            <Block Name=``Block_0''>
               <Field Name=``Field_0'' />
               <Tabular Name=``Tabular_0''>
                  <Field Name=``Field_1'' />
                  <Field Name=``Field_2'' />
               <Tabular />
            <Block />
            \end{verbatim}

            \verb+$self->GetValue( 'Block_0', '_field_list' )+ will
            return only \verb+Field_0+, whilst
            \verb+$self->GetValue( 'Block_0', '_all_field_list' )+
            returns \\
            \verb+Field_0 Field_1 Field_2+.


         \item[\_frame\_list]
            Retrieve the defined \verb+Frame+ elements of this block.
         \item[\_tabular\_list]
            Retrieve the defined \verb+Tabular+ elements of this block.
         \item[\_is\_detailblock]
            This boolean switch is set to true, if this block is a
            detailblock:

            \verb+if ( $self->GetValue( $blockname, '_is_detailblock') ){+ \\
            \verb+   ...+ \\
            \verb+}+

         \item[\_is\_masterblock]
            dito for a masterblock
         \item[\_detailblocks]
            Get all the detail blocks of a master block:

            \verb+my @detail_blocks =+ \\
            \verb+   @{ $self->GetValue( $blockname, '_detailblocks' ) };+

      \end{description}

   \item DataSource Element
      \begin{description}
         \item[\_column\_list]
            List of columns, belonging to this datasource.
         \item[\_\_curr\_index]
            Stores and returns the index of the current row/record within a
            query.
         \item[\_\_max\_index]
            Stores and returns the index of the last row/record within a
            query. 
         \item[\_\_query\_records]
            \_\_query\_records is a bucket to store (references to) the Record objects
            of a query.

            Example:

            \verb+my @rec_objs =+ \\
            \verb+   @{ $self->GetValue( $datasource, '__query_records') };+

         \item[\_\_rowid]
            Stores and returns the rowid (guid) of the current row/record within a
            query.
      \end{description}

   \item Column Element
      \begin{description}
         \item[\_field]
            Contains the name of the Field, that is bound to this Column
            (via DSColumn).
         \item[\_related\_from]
            If a e.g. Column\_1 is related to Column\_2 (via RelatedColumn),
            this relation is also stored in Column\_2 as \_related\_from
            (Column\_1).
      \end{description}

   \item Field Element
      \begin{description}
         \item[\_my\_block]
            As Fields could be hidden deeper in the hierarchy of a block
            (like as part of a Frame or Tabular), you cannot get the Block
            name easily with \_parent. \_my\_block stores the name of the
            surrounding Block of this Field.
         \item[\_my\_datasource]
            The same like \_my\_block, but for the DataSource of the
            surrounding Block.
         \item[\_my\_own\_datasource]
            As a Field may have its own DataSource for filling the
            \_list\_ref with non-standard values, this field-specific
            datasource must be separated from the block-level DataSource:

            \verb+my $block_ds = $self->GetValue( $field, '_my_datasource');+ \\
            \verb+my $field_ds = $self->GetValue( $field, '_my_field_datasource' );+

         \item[\_data\_ref]
            This is a reference to the variable, that contains the data of
            this Field.
         \item[\_data\_refs]
            If a query for instance results in 4 records, the data for a
            certain field is stored in \_data\_refs[0] to \_data\_refs[3].
            For the current record, the entry for the appropriate index is
            linked to \_data\_ref.
         \item[\_list\_ref]
            This is a reference to a an array to provide list fields (like a
            scrolling list) with the allowed values.
         \item[\_displays\_intdata]
            If some field exceptionally displays internal data instead of
            external one, it is flagged with this boolean switch.

      \end{description}

\end{itemize}

\end{document}
